\section*{مقدمه تدوین} 
کتاب قابوس‌نامه نیاز به معرفی ندارد که مقدمه‌ای از تدوینگران لازم داشته باشد. صرفا در این چند سطر، چندُ‌و‌چون شکل‌گیری این تدوین و خاطرات خواندن اثر را به یادگار مینویسم. در جمعی از دوستان، معروف به کوهنوردان سابق\footnote{همگی عزیزان این گروه به خاطر علاقه به کوهنوردی دور هم جمع شدند. شگفت است که جز یک برنامه کوه، آن هم در ظل آفتاب و در چله تابستان نرفتند. ولی خب مراودات همان تک برنامه، اینان را بس که سال‌های سال یاد همان تک برنامه کذایی کردند و دور هم مجازی گرد آمدند. این گروه سالین سال است که مجازی فعالیت فرهنگی میکند. } این اثر هفنگی، به صورت مجازی، در دوران پندمیک کویید۱۹ خوانده شد. 
در ابتدای پندمیک، گلستان مقبول نظر جمع بود و خوانده شد. اگر خواننده این سطور تا کنون گلستان را نخوانده، اکیدا توصیه میشود که ابتدای امر آن کتاب حضرت عجل خوانده شود. بعد از خوانش گلستان سعدی، این اثر مقبول نظر جمع کوهنورد افتاد. از آنجا که دسترسی به این کتاب به سادگی میسر نبود و از آنجا که نویسنده سال‌های سال بود که دار فانی را وداع گفته بود، نظر جمع بر آن گشت که تلاشی در جهت جمع‌آوری این مطالب و تدوین آن به اندازه تلاش 
خود شود. تدینگران اثر بر اساس اندک سواد و ذوقی که داشتند اثر را جمع‌آوری و تدوین کردند. خواننده محترم برای نظر کارشناسی پیرامون مدخلات به ادیبان حاذق ارجاع داده میشود. فلذا این اثر، ذوق چند تنی دوست‌دار ادبیات میباشد که در دور‌همی‌های مجازی پدید آمده است. 

در خلال این جلسات، متاسفانه دوستی از حلقه خوانش دچار سرطانی لاعلاج گردید و همه را در تاثر و شگفتی از حکمت روزگار گذاشت. شوربختانه، عزیز ما درگذشت و همه ما را در غم دوری خود گذاشت. این تدوین ابتدای امر دور هم برای ذوق خود خوانندگان گردآوری شده بود و قصد انتشاری وجود نداشت. ولی برای زنده نگه داشتن یاد و خاطره آن عزیز رفته هم که شده،‌ تصمیم ما به انتشار این تدوین شد. فلذا تقدیم به شما دوست عزیز و با یاد آن دوست رفته.

\hfill{تقدیم به میثم حقدان}

\hfill{از طرف کوهنوردان سابق}

\hfill{کمبریج، بوستن - سنه۲۰۲۱ بعد از میلاد}

\newpage