\addcontentsline{toc}{section}{باب اول - در شناختِ ایزد تعالی}
\section*{باب اول - در شناختِ ایزد تعالی}


آگاه باش اى پسر! كه هيچ چيز نيست از بودنى و نابودنى و شايدْ بود، كه آن شناخته‌ی مردم نَگَشت چنانكه اوست، جز آفريدگارِ جلَّ جَلالُه، كه شناخت را در او راه نيست، و جز او همه شناخته گشت، چه شناسنده‌ی خداى آنگه باشى كه ناشناس شوى، و مثالِ شناخته چون منقوش است و شناسنده نقاش و گمان نقش؛ تا در منقوش [قبولِ] نقش نباشد، هيچ نقاش بَر وى نقش نَكُند. نبينى كه چون مومْ نقش‌پذيرتَر از سَنگَست از موم مُهر سازند و از سنگ نسازند؟ پس در همه شناخته‌اى قبولِ شِناس است و آفريدگار قابل نيست؛ و تو بگمان در خود نگر، در آفريدگار مَنگر؛ و در ساز نِگر، و سازنده را بشناس؛ و نگر تا درنگ، ساخته سازنده از دستِ تو نَرُبايد، كه همه درنگى از زمان بود، و زمان گذرنده است، و گذرنده را آغاز و انجام بُوَد. و اين جهان را كه بسته همى بينى، بندِ او خيره مدان، و بى‌گمان باش كه بندِ او ناگشاده نماند؛ و در آلاء و نعمآىِ آفريدگار انديشه كن، و در آفريدگار انديشه مكن كه بیراه‌تر كسی، آن بُود كه جايى كه راه نبود راه جويد؛ چنانكه پيغامبر (صلى اللّه عليه و سلم) گفت: «تَفَكّروا فِى آلاءِ اللّه و نِعَمآٓئِه و لا تَتَفَكَّروا فى اللّه». و اگر كردگار بر زفان\footnote{زبان} بندگانِ خداوندِ شرعْ را گستاخیِ شناختن راه خويشْ ندادى، هرگز كس را دليرى آن نبودى كه در شناختنِ راهِ خداىِ عزّوجل سخن گفتى، كه به هر نامى و هر صفتى، كه خداى را عز‌ّو‌جل بِدان بخوانى، بر موجبِ عجز و بيچارگیِ خود دان، نه بر موجبِ الوهيت و ربوبيت، كه تو خداى را تعالى را سزاىِ او، نتوانى ستودن. پس چون بسزايیِ او، او را نتوانى ستودن، شناختن چون توانى‌؟ اگر حقيقت توحيد خواهى، بدان كه هر چه در تو محالست، در ربوبيت صدقست؛ چون يكى‌اى، كه هرکه يكى را به حقيقت بدانست، از شرك برى گشت؛ يكى بر حقيقت خداست عزّوجل، و جز او همه دُواَند. هر چه به صفت دو گردد، يا به تركيب دو بُوَد، چون جسم؛ يا به ترتيب، چون عدد؛ و يا به جمع دو بود، چون صفات؛ و يا به صفت دو بود، چون مبسوطات؛ يا به اتصال دو بود، چون طبع و صورت؛ يا در مقابله‌ی چيزى دو بود، چون جوهر [و عرض]؛ يا به تولد دو بود، چون اصل و فرع؛ يا به امكان دو بود، چون مثل و شبه؛ يا از هَرْسان چيزى را دو بود، چون هيولى و عنصر؛ يا از روى عدد دو بود، چون مكان؛ يا از روى مدد دو بود، چون زمان؛ يا از روى حد دو بود، چون گمان و نشان؛ يا از روى قبولِ چيزى دو بود، چون خاصيت و بيش و كم بود، چون مسكوك؛ و با هستى و نيستى جز او بود، چون ضدّ و فرق؛ و هر چه جز او چگونگى دارد، چون قياس؛ اين همه نشان دُوئیست و هر چه نشان دُوئی دارد، جز از خداى.
حقيقتِ توحيد آنست كه بدانى كه هر چه اندر دلِ تو آيد، نه‌ خداى بود، چه خداى تعالى آفريدگارِ آن چيزْ بُوَد، برى از شركْ و شبه، جلّ‌وجلاله و تَقَدَّسَتْ اَسْماؤُه.



\newpage
