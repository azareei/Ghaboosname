\addcontentsline{toc}{section}{باب چهلم
در آيين [و] شرط وزارت}
\section*{باب چهلم
در آيين [و] شرط وزارت}

اگر چنان بود كه بوزارت افتى محاسبت و معاملت‌شناس باش و با خداوند خويش راستى و انصاف كن و همه طريق راستى نگه‌دار و همه خود را مخواه كه گفته‌اند: «من طلب الكلّ فاته الكلّ» كه همه بتو نه‌دهند اگر دهند بعد از ان آن را خواستار بود، اگر اول فراز گذارند آخر بنه گذارند. پس چيز خداوند نگاه دار و اگر بخورى بدو انگشت خور تا در گلو بنه ماند. اما يكباره دست عمّال فرو مبند كه چون چربو از پانه دريغ دارى كبابت خام آيد، تا دانگى بديگران بنگذارى درمى بنتوانى خوردن و اگر بخورى آن محرومان خموش نه‌باشند و نگذارند كه پنهان بماند. و نيز همچنان كه با ولى نعمت خويش منصف باشى با لشكر نيز منصف باش، توفيرهاى حقير مكن كه گوشت كز دندان بيرون كنى شكم را سود ندارد كه زيان آن توفير بزرگتر از سود باشد بدان كم مايه

توفير لشكرى را دشمن خويش كنى و دشمن خداوند خويش. و اگر خواهى كه كفايتى بنمايى (79 پ) و مال جمع كنى و بحاصل آرى ويرانيهاى مملكت را آبادان گردان تا ده چندان توفير پديد آيد و خلقان خداى تعالى را بى‌روزى نكرده باشى.

حكايت بدان كه چنانكه شنودم كه ملكى از ملكان پارس با وزير خويش متغيّر شد وى را معزول كرد، وزارت را كسى ديگر نام‌زد كرد و اين معزول را گفت:

خويشتن را جايى اختيار كن كه بتو دهم تا تو با نعمت و قوم خويش آنجا روى و مقام كنى. وزير گفت: مرا نعمت نه‌مى‌بايد هر چه مرا هست ترا دادم و هيچ جاى آبادان نخواهم كه مرا بخشد، اگر بر من  همى رحمت كند از مملكت خويش دهى ويران بمن دهى تا من بحق الملك با مرّقعى بروم و آن ده آبادان كنم و آنجا بنشينم. اين ملك فرمود كه چندان ده ويران كه خواهد وى را دهيد. اندر همه مملكت بگرديدند يك بدست زمين جز آبادان نيافتند كه بوى دهند، تا خبر دادند كه در همه مملكت ويرانى نيست و بدست همى‌نيايد. وى ملك را گفت: اى خداوند من خود دانستم كه در تصرف من ويران نيست امّا اين ولايت كه از من بازگرفتى بكسى ده كه اگر وقتى ازو باز خواهى هم چنين بتو سپارد كه من بتو سپردم.

چون اين سخن معلوم ملك شد از ان وزير معزول عذرها خواست و وى را خلعت داد و وزارت بوى باز داد.

پس اندر وزارت معمار و دادگر باش تا زبان تو هميشه دراز باشد و زندگانى تو بى‌بيم بود كه اگر لشكر بر تو بشورند خداوند [را] ناچاره دست تو كوتاه بايد كرد تا دست خداوند تو كوتاه نه‌كنند، پس آن بيداد نه بر لشكر كرده باشى و آن توفير تقصير كار تو گردد. پس خداوند را بعث كن بر نيكويى كردن با لشكر كه پادشاه بلشكر آبادان باشد و ده بدهقان، پس در آبادانى كوش و جهان‌دارى كن و بدان كه جهان‌دارى بلشكر توان كردن و لشكر بزر توان داشت و زر بعمارت كردن بدست توان آوردن و عمارت بداد و انصاف توان كرد، از عدل و انصاف غافل مباش. و اگر چه [بى] خيانت و صاين باشى هميشه از پادشاه ترسان باش كه كسى را از خداوند ترسيدن.