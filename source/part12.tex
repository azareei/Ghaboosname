 \addcontentsline{toc}{section}{باب دوازدهم - در مهمان كردن و مهمان شدن}
\section*{باب دوازدهم- 
در مهمان كردن و مهمان شدن}

اما مردمانِ بيگانه را هر روز مهمان مكن كه هر روز به‌ سزا به حق مهمان نتوانى رسيد. بنگر تا به يك ماه چند بار ميزبانى خواهى كردن، آنكه سه بار خواهى كردن يك بار كن و نفقاتى كه در آن سه مهمانى خواهى كردن در اين يك مهمانى كن تا خوان تو از همه عيبى برى بُوَد و زبان عيب‌جويان بر تو بسته بُوَد. و چون ميهمانان در خانه‌ی تو آيند هر كسى را پيش‌باز همى رو و تقرّبى همى كن اندر خور ايشان و تيمار هر كسى به‌سزا همى دار چنان‌كه بوشكور بلخى گويد:


\begin{quote}
\centering
كه را دوست مهمان بود ورنه‌دوست\footnote{هرکه دوست و نه‌دوست مهمانش بودند}  \quad \quad 
شب و روز تيمار مهمان بروست\footnote{نیکوست}
\end{quote}

اگر وقت ميوه بُوَد پيش از نان خوردن ميوه‌هاى تر و خشك پيش ايشان نِهْ تا بخورند. و يك زمان توقف كن آنگاه مردمان را بنان بر و تو منشين تا آنگاه كه مهمانانت بگويند، چون يك‌بار بگويند: بنشين و با ما مساعدت كن، تو گوى: شايد بنشينم‌؟ بگذاريت تا خدمت كنم. و چون يك‌بار ديگر تكرار كنند بنشين و با ايشان نان خور. اما فرود\footnote{پایین،زیر} همه كس نشين مگر مهمانى سخت بزرگ بُوَد كه نشستن ممكن نباشد.

و عذر مخواه از مهمان كه عذر خواستن طبع بازاريان بُوَد، هر ساعت مگوى كه: «اى فلان، نان نيك بخور، هيچ نمى‌خورى، به جان تو كه شرم ندارى، من خود سزاى تو چيزى نتوانستم كردن، ان شا  اللّه بار ديگر عذر اين باز‌خواهم»؛ كه اين نه سخنان محتشمان باشد، لفظِ كسى بود كه به سالها مهمانى يك‌بار كند از جمله‌ی بازاريان، كه از چنين گفتارْ مردم خود شرم‌زده گردد و نان نتواند خوردن و نيم‌سير از نان برخيزد. و ما را به گيلان رسميست خوب، چون مهمانى را به خوان برند كوزهه‌هاى آب [و] خوردنى در ميان خوان بنهند و مهمانْ‌خداى\footnote{میزبان} و پيوستگانِ او از آنجا بروند، مگر يك كس از دور بيايد از بهر كاسه نهادن را تا مهمانان چنانكه خواهند نان بخورند، آنگه ميزبان پيش آيد، و رسم عرب هم اينست. و چون مهمانان نان خورده باشند، بعد از دست شستن، گلاب و عطر فرماى و چاكران و بندگانِ مهمانان را نيكو تعهد كن كه نام و ننگْ ايشان بيرون برند.

و اندر مجلس نُقْلْ و اَسْپَرْغَم\footnote{گل، هر گیاهی خوش‌بو، ریحان و نیز ریحانی را گویند که آن را شاه‌اَسْپَرْم خوانند} بسيار فرماى نهادن، و مطربان خوش فرماى آوردن. و تا نبيدِ خوش نبُوَد مهمان مكن، كه همه روز خود مردمان نان خورند، سيكى خوش و سماع خوش بايد تا اگر در خوان و كاسه تقصيرى افتد، عيبِ خوانِ تو بدان پوشيده گردد. و نيز سيكى خوردن بزه است، چون بزه خواهى كردن، بارى، بزه‌ی بى‌مزه مكن، سيكى كه خورى خوش‌ترين خور، و سماع كه شنوى خوش‌ترين شنو، و اگر حرامى كنى با كسى نيكو كن، تا اگر اندر آن جهان مأخوذ باشى، بدين جهان معيوب و مذموم نباشى. پس چون اين همه كه گفتم، كرده باشى، خود را بر مهمانان حقى مشناس؛ ايشان را بر خويشتنْ حَقْ واجب دان.

% \HekaiatBegin
 چنين شنيدم كه پسر مقله، نصر بن منصور تميمى را عَمَلِ\footnote{کار دولتی، ماموریت} بصره داد.
سال ديگر باز كرد\footnote{معذور شد} و حسابش همى‌كرد\footnote{بازخواست کردند}. و مردى منعم بود و خليفه را بر او طمعى افتاده بود. حسابش بكردند و مال بسيار بر او فرو آوردند. پسر مقله گفت: اين مال بگزار يا به زندان رو. نصر گفت: اى مولانا، مرا مال هست و ليكن اينجا حاضر نيست؛ يك ماه مرا زمان ده كه بدين مقدار مرا به زندان نبايد رفت. پسر مقله دانست كه مرد را طاقت اين مال گزاردن هست و راست همى‌گويد. گفت: از امير المؤمنين دستورى نيست كه باز جاى رَوى تا اين مال نگزارى اما اينجا در سراى من در حجره‌اى بنشين و يك ماه مهمان من باش. نصر گفت: فرمان بُردارم. در سراى پسر مقله محبوس بنشست. و اتفاق را اول ماه رمضان بود چون شب اندر آمد پسر مقله گفت: فلان را بياريد تا هر شب روزه با ما بگشايد. اين نصر يك ماه رمضان روزه با وى همى‌گشاد. چون عيد كردند روزى چند بر آمد پسر مقله بدو كس فرستاد كه اين مال همى دير آورند، تدبير اين كار چيست‌؟ نصر گفت: من اين زر گزاردم، پسر مقله گفت: به كه دادى‌؟ گفت: تو را دادم. پسر مقله طيره\footnote{خفت، شرمندگی، خشم} گشت و نصر را بخواند و گفت: اى خواجه اين مال كِى به من دادى‌؟ نصر گفت: من زر به تو ندادم و لكن اين يك ماه نان تو را رايگانى بخوردم. ماهى بر خوان تو روزه‌ی خويش گشادم [و مهمان تو بودم] اكنون كه عيد آمد، حق من اينست كه از من زر خواهى‌؟ پسر مقله بخنديد و گفت: خط و برات\footnote{رسید پرداخت} بستان و برو به سلامت كه اين زر به دندان مزد به تو دادم و من از بهر تو بگزارم و نصر بدين سبب از مصادره برست.
% \HekaiatEnd


پس از مردم منت پذير و تازه روى باش. و لكن نبيد كم خور و پيش از مهمانان مست مشو، چون دانى كه مردمان نيم مست شدند آنگاه از خويشتن سكرى همى نماى و ياد مردم همى‌گير و نوش خور همى ده به حدّ و اندازه. و پيوسته تازه روى و خنده ناك همى باش اما بيهوده خنده مباش، كه بيهوده خنديدن دَوْمْ ديوانگيست؛ چنانكه كم خنديدن دَوْمْ سياست است و خويشتن‌داريست. چه گفته‌اند كه خنده‌ی بيهوده و بى‌وقت گريه بود. و چون مهمان مست شود و بخواهد رفتن، يك‌بار و دوبار خواهش كن و تواضع نماى، مگذار كه بِرَوَد، بارِسوم در وى ميآويز، به تلطف به راهش بكن تا برود. و اگر چاكران تو خطائى كنند از ايشان در گذار و پيش مهمان روى ترش مكن و با ايشان جنگ مكن كه: اين نيكست و آن نه نيكست. اگر چيزى تو را ناپسنديده آيد بارِ ديگر مفرماى كردن و اين يك‌بار صبر كن. و اگر مهمان تو هزار مَحال بگويد يا بكند از وى بردار و حُرمتِ وى بزرگ دار.

 چنان شنيدم كه: وقتى معتصم مجرمى را پيش خويش گردن همی‌فرمود زدن؛ اين مرد گفت: اى امير المؤمنين به حق خداى تعالى و به حق رسول عليه السلام كه نخست مرا به يك شربت آب مهمان دار وآنگه هر چه خواهى بفرماى كه سخت تشنه شده‌ام. معتصم بر حكم سوگند فرمود كه او را آب دهيد. آب به وى دادند. مَرْد آب بخورد و به رسم عرب گفت: «كثّرهم اللّه خيرا» يا امير المؤمنين مهمانِ تو بودم بدين يك شربت آب؛ اكنون اگر به طريق مردمى مهمان كشتن واجب كند تو مرا بفرماى كشتن و اگر نه عفو كن تا بر دست تو توبه كنم. معتصم گفت: راست گفتى، حق مهمان بزرگست، تو را عفو كردم، توبه كن كه پس ازين چنين حركت خطا نكنى.

اما بدان كه حقِ مهمان نگاه داشتن واجبست، و لكن حق آن مهمان كه به حق شناسى ارزد، نه چنان كه هر قلاّشى\footnote{بیکاره، ولکرد} را به خانه برى وآنگه چندين تواضع فرمايى كه اين مهمان منست، بدان كه اين تقرّب با كه بايد كردن.

فَصْلْ، پس اگر مهمان شوى مهمانِ هر كس مشو، كه حشمت را زيان دارد. و چون شوى سخت گرسنه مشو و سير نيز مشو كه اگر نان نتوانى خوردن ميزبان بيازارد و اگر به افراط خورى زشت باشد. و چون در خانه‌ی ميزبان شوى جايى نِشين كه جاى تو باشد. و اگر خانه‌ی آشنايان تو باشد و تو را ولايتى باشد در آن خانه؛ بر سرِ نان و بر سرِ نبيد كارافزايى مكن، با چاكران ميزبان مگوى كه: اى فلان، اين طبق بدان جاى نه و اين كاسه فلان جاى نه يعنى كه من ازين خانه‌ام. مهمان فضولى مباش و به نان و كاسه‌ی ديگران را تقرب مكن. و چاكر خويش را زلّه مده، كه گفته‌اند كه «الزّلة زلّة.» و مست خراب مشو، چنان برخيز كه اندر راه اثر مستى بر تو پيدا نبود. مستى مشو كه از چهرۀ آدميان بگردى. تمامى مستى به خانه‌ی خويش كن. و اگر به مثل يك قدح نبيد خورده باشى و كهتران تو صد گناه بكنند كس را ادب مفرماى كردن اگر چه مستوجب ادب باشد كه هيچ كس آن از روى ادب نشمارد و گويند: عربده همى‌كند. هر چه خواهى كردن نبيد ناخورده كن تا دانند كه آن قصد ادبست نه معربدى كه از مست همه چيزى بعربده شمرند همچنان كه گفته‌اند: «الجنون فنون»، ديوانگى گونه گونه است، عربده نيز هم گونه گونه است كه مستى هم نوعى از ديوانگيست. و بدان كه در مستى بسيار گفتن عربده است، و نقل بسيار خوردن عربده است، و بسيار دست زدن و پاى كوفتن عربده است، و نقل بسيار كردن هم عربده است، و پيوسته سرود گفتن خارج و باز خواستن عربده است، و بسيار تقرب كردن بنا واجب هم عربده است، و بسيار خنديدن و بسيار گريستن هم عربده است، در مستى و در هشيارى ديوانگى است. پس ازين همه هر چه گفتم پرهيز كن كه اين هر چه گفتم يا جنونست يا عربده [كه نه همه عربده] و جنون مردم را زدن باشد.

و پيش هر بيگانه‌اى مست خراب مشو مگر پيش عيالان و بندگان خويش. و اگر از مطربان سماعى خواهى همه [راههاى] سبك\ مخواه تا به رعنائى\footnote{احمقی} و سبكى منسوب نباشى، هر چند بيشتر جوانان راه‌هاى سبك خواهند.



\newpage








































