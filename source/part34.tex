\addcontentsline{toc}{section}{باب سى و چهارم - 
در علم نجوم}
\section*{باب سى و چهارم - 
در علم نجوم}

و اگر منجم باشى، جهد كن تا بيشتر رنجِ خويش در علمِ رياضى برى، كه علمِ احكامِ علمى وافرست و  دادِ او به تمامى نتوان دادن. بى‌خطائى از آن‌كه كس چنان مصيب\footnote{درستکار} نباشد، كه بر وى خطائى نرود. اما به همه حال ثمره‌ی نجوم احكام‌ست، كه چون تقويم كردى فايده از تقويم احكام‌ست، پس چون از احكامى نمى‌گزيرد، جهد كن تا اصولش نيكو بدانى و بر مقومى\footnote{تقویم نویسی} قادر باشى، كه اصلِ حكم آنگه راست آيد كه تقويم سيارگان راست بود و طالع درست بود. نِگر كه بر طالع تخمينى اعتماد نكنى، الا كه به استقصاى سخت به حساب و نمودارات ممهّد\footnote{آماده} كن و چون حساب و نمودارات راست آيد، آنگه حكمى كه از آنجا كنى راست بود و به هر حكمى كه كنى مولدى و ضميرى تا از حالات كواكب آگاه نگردى و از طالع و درجه‌ی طالع و خداوند طالع و قمر و برج قمر و خداوند برج و مزاج بروج‌ها كه در هر برجى چون باشد و از خداوند خانه‌ی حاجت و آن كوكب كه ماه ازو برگشته باشد و آن كوكب كه بدو خواهد پيوست و آن كوكب كه مستولى بود بر درجه‌ی طالع و خانه‌ی حاجت و آن كوكب كه مستولى بود بر درجه‌ی سير كواكب و آن كوكب ثابته كه مسير بدو رسيده باشد و از درجه‌ی مبتز مسعود و درجه‌ی مظلمه و درجه‌ی آبار و حصار و از درجه‌ی محترقه كه در حرم آفتاب بود و صاعد. و ازين هيچ غافل مباش و سهم‌ها و اثنا عشريات و دريجان\footnote{قانونی در علم هیئت که در آن صور و اشکال فلکی را به سه طبقه تقسیم کرده اند. (ناظم الاطباء). ادرجان. وجه. دهج. صورت} و جانبات و نهبهر و ارباب مثلثات و حد و صورت و شرف و هبوط و خانه‌ی وبال و حزن 
و فرح و اوج و حضيض. و آنگه بنگر در حالات قمر و كواكب چون اقبال و ادبار و نيز نظر، مقارنه و اتصال و انصراف و بعد‌النور و بعد‌الاتصال و خالى السير و وحشى‌السير، نقل و جمع و منع و رد، دفع‌التدبير، دفع‌القوه، دفع‌الطبيعه، انتكاث، اعراض، قوت، مكافات، قبول، تشريق و تغريب، اجتماعى و استقبالى، معرفت هيلاج و كدخدا و عطيت دادن و كم كردن و زيادت كردن عمر و راندن سيرهاى پنج‌گانه. چون ازين همه آگاه گردى، آنگاه در احكام سخن گوى تا حكم تو راست آيد و حكم از تقويمى معتمد كن چنانكه حكيم آن تقويم را زيجى كرده باشد كه به خطى معروف بود و در اوساط وى نگاه كرده و مجموعه و مبسوطه‌ی وى نيكو ديده و مكرّر كرده و تعديلهاى تأمل كرده، با اين همه احتراز كنىِ، از سهو و از خطا تا غلطى نيفتد. و چون اين همه احتياط كرده باشى، بايد كه تو را اعتقاد بود كه هر حكمى كه من كردم چنان خواهد بودن، اگر بر آن قول معتقد نباشى، هيچ اصابت نيفتد. و در مسئله‌اى كه پرسند ضميرى هر چه گويى توان گفت چنانكه بيشتر حكم تو راست آيد، اما حديث مولدها، من از استاد خويش چنان شنودم كه مولد مردم نه آنست به حقيقت كه از مادر جدا شود كه مولد اصلى طالع زرع‌ست، وقت مسقطالنطفه، آن طالع كه آب مرد اندر رحم زن افتد و قبول كند آن طالع مولد اصلي‌ست، نيك و بد همه بدان پيوسته است. اما آن ساعت كه از مادر جدا شود آن طالع را تحويل كبرى خوانند و تحويل سال كه بيوفتد آن را تحويل وسطى خوانند و تحويل شهور را تحويل صغرى و بر سر مردم آن گذرد كه در طالع مسقط‌النطفه بود و دليل خبر رسول عليه السلام: «السعيد من سعد فى بطن امّه و الشقى من شقى فى بطن امه» و سيد عالم عليه السلام اين سخن ازين گفته است كه من تو را گفتم. اما تو را در طالع زرع سخن نيست كه نه‌به پاى چون تويى بافته‌اند، اما اينكه از طالع تحويل كبرى بجويى، طريق استادان گذشته نگاه دار اندر هر حكمى كه كنى چنانكه پيش ازين فرمودم. و اگر مسئله‌اى وقتى پرسند اول به طالع وقت نگر و به صاحب طالع و پس به قمر و برج قمر و خداوندش و بدان كواكب كه قمر بدو خواهد پيوست و بدان كوكب كه قمر ازو باز گشته بود و بدان كوكب كه در طالع يابى يا در وتدى و اگر در وتدى بيش از كوكب يابى بنگر كه مستولى كيست و شهادت بيشتر كه راست سخن از آن كوكب گوى تا مصيب باشى.

فصل، آنچه شرط احكامست لختى گفتم اكنون اگر چنانكه مهندس باشى ومّاح در حساب قادر باش، زينهار كه يك ساعت بى‌تكرار حساب نباشى كه علم حساب علم وحشى است. پس اگر زمينى پيمايى نخست بايد كه زوايا بشناسى و شكلهاى مختلف الاضلاع را خوار نگيرى و نگويى كه اين را بر يك مساحت بكنم و باقى به تخمين كه حساب مساحت تفاوت بسيار آورد و جهد كن تا زوايا را نيك بشناسى كه استاد من رحمه اللّه پيوسته مرا گفتى: هان اى فلان تا از زوايا غافل نباشى، در حساب مساحت كه بسيار ذوات\footnote{جمع ذات،‌حقایق،‌ ماهیات} الاضلاع بود، كه در وى زاويه‌ی قوسى به حاده ماند برين مثال. يا برين مثال و بسيار حادّه باشد كه به منفرجه ماند و اينجا بسيار تفاوت افتد. و اگر شكلى بود كه بر تو مشكل شود و معلوم تو نگردد مساحت او به تخمين مكن، همه را مثلث كن يا مربع، كه هيچ شكلى نبود كه برين گونه نتوان كردن و آن وقت هر يكى را بپيماى كه راست آيد. و اگر همچنين درين باب همى گويم بسيار توان گفت اما كتاب از حال خود بگردد و اين قدر كه گفتم ناگزير بود از آنچه سخن نجومى گفتم خواستم كه درين باب سخنى بگويم تا نبذها از هر چيزى تمام بود تا باشد كه به يكى كار كند و رغبت نمايد.

\newpage
