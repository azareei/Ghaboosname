\addcontentsline{toc}{section}{باب سى و ششم - 
در آيين و رسم خنياگرى}
\section*{باب سى و ششم - 
در آيين و رسم خنياگرى}

و اگر خنياگر\footnote{نوازنده، آوازه‌خوان} باشى، سبكْ‌روح و خوش‌ْگوى باش و خوىِ نيكو دار و خود را به طاقت خويش هميشه پاك‌ْجامه دار و مطيّب و چرب زبان باش، چون به سرا كارى شوى، رویْ‌گرفته مباش و همه راه‌هاى گران مزن و نيز همه راه‌هاى سبك مزن، كه نيز همه از يك نوع زدن شرط نَبُوَد كه آدمى همه از يك طبع نباشد، همچنان كه همه خلق مختلف‌اند، خلق نيز مختلف است؛ و زين سبب استادان اهل ملاهى اين صناعت را ترتيبى نهادند: اول دستان خسروانى زنند و آن از بهر مجلس ملوك ساختند.

بعد از آن طريق‌ها به وزن كم از آن بنهادند، چنانكه بدو سرود توان گفت و آن را راه نام كردند و آن راهى بود كه به طبع پيران و خداوندان جدّ نزديك بود، پس اين راه را گبران\footnote{آتش پرست، مجاور آتشکده} از بهر اين قوم ساختند، آنگاه چون اين قوم را ديدند كه خلق همه پيرو اهل جد نباشند گفتند: از بهر پيران طريقى نهاديم از بهر جوانان نيز طريقى نهيم، پس بجستند شعر هايى كه به وزن سبك‌تر بود بر وى راههاى سبك ساختند و خفيف نام كردند، تا از پس اين هر راهى گران ازين خفيفى بزنند و بگويند تا در نوبتى مطربى هم پيران را و هم جوانان را نصيب بود. پس كودكان و زنان لطيف‌ طبع‌تر باشند، بى‌بهره ماندند تا آنگه كه تو رانگفتند و پديدار آمد، آنگه اين ترانه را نصيب اين قوم كردند تا اين قوم نيز راحت يابند ازين لذت، از آنچه اندر وزنها هيچ وزنى لطيف‌تر از ترانه نيست پس همه از يك نوع مزن و مگوى چنين كه ياد كردم همى زن و همى گوى تا همه كس از استماع تو بهره يابند. و در مجلسى كه بنشينى نگه كن اگر مستمع سرخ روى باشد و دموى روى بود بيشتر بر دورود زن و اگر زرد روى و صفراوى بود بيشتر زير زن و اگر سياه گونه بود و نحيف سوداوى بود، بيشتر بر سه تار زن و اگر سپيد پوست و فربه بود، مرطوب بود بيشتر بر بم زن كه اين رودها را بر چهار طبع ساختند و اهل علم موسيقى اين صناعت را هم بر چهار طبع مردم ساختند. و هر چند اين كه گفتم در شرط و آيين مطربى نيست خواستم كه ازين معنى نيز تو را آگاه كنم تا معلوم تو باشد. ديگر جهد كن كه محاكى\footnote{حکایت کننده} باشى كه به مقدار حكايت و مزاح و مطايبه كردن\footnote{شوخی کردن} تو از رنج خنياگرى تو بكاهد . ديگر آنكه اگر خنياگرى باشى و شاعرى نيز دانى، عاشقِ شعرِ خويش مباش و همه روايت شعر خويش مكن، كه چنانكه تو را با شعر خويش خوش باشد، مگر آن قوم را خوش نباشد كه خنياگران راويان شاعرانند نه راويان خويش. و ديگر آنكه اگر نردباز باشى، چون در سراكارى\ شوى اگر دو كس به هم نرد همى بازند، خنياگرى خويش باطل مكن و به تعليم كردن نرد منشين و يا خود به نرد و شطرنج باختن مشغول مشو كه تو را به خنياگرى خوانده‌اند نه به مقامرى\footnote{قماربازی}. و نيز سرودى كه آموزى ذوق نگاه‌دار و غزل و ترانه‌ی بى‌وزن مگوى و مياموز كه آنگه سرودت جاى گير نبود، بلكه سرودت جاى ديگر بود و زخمه جاى ديگر. و اگر چنانكه بر كسى عاشق باشى همه روز حسب حال خويش مگوى، كه مگر آنكه تو را خوش آيد ديگران را خوش نیايد. و بهر سرودى در معنى ديگر گوى و شعر و غزل بسيار ياد گير چون: فراقى و وصالى و توبيخ و ملامت و عتاب ورد و منع و قبول و جفا و وفا و عطا و احسان و خشنودى و گله و حسب حال‌هاى وقتى و فصلى چون: سرودهاى بهارى و خزانى و زمستانى و تابستانى بايد كه بدانى كه به هر وقت چه بايد گفت، نبايد كه اندر بهار خزانى و اندر خزان بهارى گويى و اندر تابستان زمستانى و اندر زمستان تابستانى نگويى، وقت هر سرودى بايد كه بدانى اگر چه استاد بى‌نظير باشى. در سراكار حريفان را همى نگر، اگر قومى مردمان خاص بينى سزاى عقل كه صرف مطربى دانند پس مطربى كن و راه‌هاى نيك و نواهاى نيك همى زن، اما سرود بيشتر اندر پيرى گوى و اندر مذمت دنيا و اگر قومى جوانان و كودكان بينى نشسته، بيشتر طريق‌هاى سبك گوى و سرودهايى گوى كه بيشتر در زنان گفته باشند يا در ستايش نبيد\footnote{شراب خرما} و نبيد خوارگان. و اگر قومى سپاهى و عيار پيشگان را بينى دو بيتي‌هاى ماوراءالنهرى گوى، در حرب كردن\footnote{جنگ کردن} و خون ريختن و ستودن عياران. و جگر خواره مباش و همه نواهاى خسروانى مزن و مگوى كه شرط مطربى نگاه همى دارم، نخست بر پرده‌ی راست چيزى بگوى پس بر رسم بر هر پرده‌اى چون پرده‌ی باده و پرده‌ی عراق و پرده‌ی عشاق و پرده‌ی زير افگند و پرده‌ی بوسليك و پرده‌ی سپاهان و پرده‌ی نوا و پرده‌ی گذشته و پرده‌ی راهوى و شرط مطربى به جاى آر، كه تا تو شرط مطربى به جاى آرى مردمان ملول شده باشند و يا مست شده و رفته؛ بنگر كه هر كس چه راه دوست‌تر دارد و چه مى‌خواهد، چون قدح بدو رسد آن گوى كه او خواهد، كه تا تو آن نگويى كه او خواهد او تو را آن ندهد كه تو خواهى، خنياگر را بزرگترين هنرى آنست كه بر طبع مستمع رود. و در مجلس كه باشى پيش دستى مكن بياد گرفتن و سيكى بزرگ خواستن؛ و نبيد كم كن تا سيم به حاصل كنى، چون مقصود خويش به حاصل كردى و سيم يافتى آنگه تن اندر نبيد ده. و در سراكار با مستان ستيزه مكن به سرودى كه خواهند يا نقدى كه كنند گر چه محال گويند تو زان مينديش، بگذار تا بگويند. چون نبيد بخورى و مردمان مست شوند تو با هم‌كاران در مناظره و محاكات مشو، به مستان مشغول باش كه بخانه‌ی خداوندگار به محاكا كردن سيم به حاصل نيايد و مردمان را ضجر كردن باشد. و نگر تا خنياگر معربد نباشى كه از عربده‌ی تو سيم خنياگرى از ميان برود و سر و روى شكسته و جامه دريده باز خانه روى كه خنياگران مزدور مستان باشند و مزدور معربد را دانى كه مزد ندهند.

و اگر در مجلسى تو را كسى مى‌ستايد آن كس را تواضع همى نماى و چيزى كه وى خواهد بيشتر گوى تا ديگران نيز بستايند. اول به هشيارى ستودن بى‌سيم بود چون مست شوند سيم از پس آن ستودن باشد و گر مستان به راهى يا به سرودى در سخت شوند چنانكه عادت مستانست تو از گفتن و زدن آن ملول مشو، همى گوى و همى زن تا آنگه كه از ان ميان غرض تو حاصل شود كه مطربان را بزرگترين هنرى صبوريست كه با مستان كنند كه اگر صبور نباشند هميشه محروم مانند. و نيز گفته‌اند كه: خنياگر كور و كر و لال بايد كه بود يعنى كه گوش به جايى ندارد كه نبايد داشت و جايى ننگرد كه نبايد نگريست و چون از ان سرا برود و آنجا چيزى كه ديده باشد و شنيده، جاى ديگر نگويد. چنين مطرب هميشه با ميزبان بود و شغلش روان بود. «و اللّه الهادى الى سواء السّبيل.» 

\newpage
