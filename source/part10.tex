\addcontentsline{toc}{section}{باب دهم - در خویشتن‌داری و ترتیب خوردن و آیین آن}
\section*{باب دهم - در خویشتن‌داری و ترتیب خوردن و آیین آن}

بدان ای  پسر! كه مردمانِ عامه را در شغل‌هاىِ خويش ترتيب و اوقات پديد نيست، به وقت و ناوقت به ننگرند. و بزرگان و خردمندان هركارى را از آنِ خويش وقتى پديد كردند، بيست و چهار ساعت شبانروزى بر كارهاى خويش ببخشيدند، ميان هر كارى تا هر كارى را فرقى و وقتى نهادند و حدى و اندازه‌اى پديد كردند تا كارهاى ايشان به يكديگر اندر نياميزد و خدمتگارانِ ايشان را نيز معلوم بُوَد كه به هر وقتى به چه كار مشغول بايد بودن تا شغل‌هاى ايشان همه بر نظام باشد.

امّا اول به حديث طعام خوردن: بدان كه عادت مردمانِ بازارى چنان رفته است كه بيشتر طعام به شب خورند و آن سخت زيان‌كارست، دايم با تخمه باشند. و مردمان سپاهی‌پيشه را عادت چنانست كه وقت و ناوقت ننگرند، هر وقت كه يابند بخورند و بدان مشغول باشند و اين عادتِ ستوران باشد كه هرگه كه علف يابند همى‌خورند. و مردمان خاص و محتشمان به شبان‌روزى اندر يك‌بار نان خورند و اين اندر طريقِ خويشتن‌دارى نيكوست و ليكن تَنْ ضعيف گردد و مرد بى‌قوت بُوَد. پس چنان [صواب‌تر] كه مردمِ محتشم بامداد به خلوت مسكنه بكند و آنگاه بيرون آيد و به كدخدايى خويش مشغول شود تا نمازِ پيشين بكند، آن‌قدر كه تو را بود نيز رسيده باشد؛ و آن كسانى كه با تو نان خورند حاضر فرماى كردن تا با تو نان خورند. اما نان به شتاب مخور و آهسته باش؛ و بر سر نان با مردمان حديث همى‌كن چنانكه در شرط اسلامست و لكن سر در پيش افگنده دار و در لقمه‌ی مردمان منگر.

% \HekaiatBegin
شنودم كه وقتى صاحب عبّاد نان همى خورد با نديمان و كسان خويش، مردى لقمه از كاسه برداشت، مويى در لقمه‌ی او بود، مرد همى‌نديد. صاحب او را گفت: اى فلان، موى از لقمه بردار. مرد لقمه از دست فرونهاد و برخاست و برفت. صاحب فرمود كه باز آريدش و پرسيد كه اى فلان چرا نان نيم‌خورده از خوانِ ما برخاستى‌؟ اين مرد گفت: مرا نانِ آن‌كس نبايد خورد كه تارِ موى در لقمه‌ی من بيند. صاحب سخت خجل شد از آن حديث.
% \HekaiatEnd

اما تو به خويشتن مشغول باش نخست بر بَوارِد\footnote{ترشی باشد که در برابر شیرینی است} خوردن درنگ همى‌كن، آنگه بعد از آن، كاسه فرماى نهادن. و رسم محتشمان دو گونه است: بعضى نخست كاسه‌ی خويش فرمايند نهادن، آن وقت از آنِ قوم، و بعضى نخست آنِ قوم فرمايند نهادن آنگه آنِ خويش و اين نيكوتر، كه اين طريق كَرَمَست و آن طريق سياست. اما بفرمايند تا چون كاسه آرند، از لونى به لونى روزگار برند، كه همه شكم‌ها يكسان نباشد، چنان كن كه چون از خوان برخيزى كم‌خوار و بسيارخوار هر دو سير باشند. و اگر پيش تو خوردنى بُوَد [كه] پيش ديگران نَبُوَد، ديگران را از آن نصيبى همى‌كن. و بر سرِ نان بر ترش‌روى مباش و بر خوانِ سلار بر خيره جنگ مكن كه: فلان خوردنى نيكست يا فلان خوردنى بد است و اين سخن خود به بابى ديگر گفته آيد. و چون ترتيب نان خوردن بدانستى ترتيب شراب خوردن بدان كه آن [را] نيز هم نهادى و رسميست.


\newpage

