\addcontentsline{toc}{section}{باب دوم - در آفرینش و ستایش پیغامبر علیه‌السلام}
\section*{باب دوم - در آفرینش و ستایش پیغامبر علیه‌السلام}



و بدان اى پسر! كه ايزد تعالى جهان را نه از بهرِ نيازِ خويش آفريد و نه بر خيره آفريد؛ چه بر موجب عدل آفريد، بيافريد بر موجبِ عدلْ و بياراست بر موجبِ حكمت؛ چون دانست كه هستى بِهْ كه نيستى، كُوْن بِه كه فساد، زيادت بِهْ از نقصان، خوب بِهْ كه زشت، و بر هر دو توانا بود و دانا بود. آنچه بِهْ بود بِكَرد و خلافِ دانشِ خود نكرد و به هنگام كرد. و آنچه بر موجبِ عدلْ بود، بر موجبِ جهل و گزاف نشايد، كه نهادش برْ موجبِ حكمت آمد؛ چنانكه زيباتر بود بِنِگاشت؛ چنانكه توانا بود، كه بى‌ آفتاب روشنى دهد، و بى‌ ابر باران دهد، و بى‌طبايعْ\footnote{ طبایع اربع؛ چهار عنصر آب، آتش، باد، و خاک} تركيب كند، و بى ‌ستاره تأثيرِ نيک و بد در عالم پديد كند؛ بلى، چون كار بر موجبِ حكمت بود، بى‌ واسطه هيچ پيدا نكرد و واسطه را سبب كُوْن و فساد كرد، زيرا كه چون واسطه برخيزد، شرف و منزلت ترتيب برخيزد؛ و چون ترتيب و منزلت نبود، نظام نبود؛ و فعل را از نظام لابدْ بود، پس واسطه نيز لابدْ بود؛ و واسطه پديد كرد، تا يكى قاهر بود و يكى مقهور، و يكى روزى‌خوار بود و يكى روزى پَرْوَر، و اين دُوئى بر يكى [اى] ايزد تعالى گواهست. پس چون تو واسطه بينى و غرض نبينى، نگر تا به واسطه ننگرى؛ و كم و بيش از واسطه نبينى، از خداوند واسطه بينى. اگر زمين بر ندهد، تاوان بر زمين منه؛ و اگر ستاره داد ندهد، تاوان بر ستاره منه؛ ستاره از داد و بیداد همچنان بى‌آگاه است، كه زمين از بَرْ دادن. چون زمين را آن توانايى نيست كه نوش درافگنى ز هر بار آرد، ستاره هم ايدونست\footnote{ایدون: اینچنین ، بدین طریق، همچنین}، نيكى و بدى نتواند نمودن. چون جهان به حكمت آراسته شد، آراسته را از بَر دادن و زينت لابُدْ بود؛ پس درنگر بدين جهان، تا زينتِ او و بَرِ او ببينى از نبات و حيوان و خورِش‌ها و پرورش‌ها و پوشيدنى.


و انواع خوبى، كه همه زينتى است، از موجب حكمت پديد كرده؛ چنانكه در محكم تنزيل خود همى‌گويد: «وَ ما خَلَقْنَا السَّماوات وَ الْأَرض وَ ما بَيْنَهُما لللاعِبِین»، «ما خلقنا هما الا بالحق». چون دانستى كه ايزد در جهان هيچ نعمتى بيهوده نيافريد، بيهوده بُوَد كه دادِ نعمتْ و روزى ناداده ماند؛ و دادِ روزى آنست كه به روزى‌خواره دهى تا بخورد. چون داد چنين بود، مردم آفريد تا روزى خورد. چون مردم پديد كرد و تمامیِ نعمت به مردم بود و مردم را لابُدْ بود از سياست و ترتيب؛ و ترتيب و سياست بى‌ رهنماى خام بود، كه هر روزى‌خواره كه روزى بى‌ترتيب و عدل خورد، سپاس روزى‌دهنده [نداند و اين عيب روزى‌دهنده] را بُوَد كه روزىِ بى‌دانشان و ناسپاسان را دهد؛ و چون روزى‌ده بى‌عيبْ بُوَد، روزى‌خوار را بى‌دانش نگذاشت؛ چنانكه اندر كتابِ خويش ياد كرد: «وَ فِي السّماءِ رِزْقُكُمْ وَ ما تُوعَدُونَ». در ميان مردمان، پيغامبران فرستاد تا رَهِ داد و دانش و ترتيبِ روزی‌خوردن و شكرِ روزى‌ده گزاردن  به مردم آموختند؛ تا آفرينش جهان به عدل بُوَد، و تمامیِ عدلْ به حكمت، و اثرِ حكمتْ نعمتْ، و تمامى نعمتْ به روزى‌خوار، و تمامى روزى‌خوار به پيغامبر ره‌نماى، كه از اين ترتيب هيچ كم نشايد كه باشد، تا به حقيقت پيغامبر راهنماى را بر روزى‌خوار خداى تعالى چندانْ فضلْ آنست كه روزى‌خوار را بر روزى. پس چون از خِرَد نِگَرى، چندان حرمت و شفقت و آرزو كه روزى‌خوار را بر نعمت و روزی است، واجب كند كه حقِ راه‌نماىِ خويش بشناسد، و روزى‌دهِ خويش را منّت دارد، و فريشتگان او را حق‌شناس باشد، و همه پيغامبران را راست‌گوى دارد، از آدم تا به پيغامبر ما عليهم السلام، و فرمانبردار باشد در دين، و در شكر منعم تقصير نكند، و حق فرايض دين نگاه دارد تا نيك‌نام و ستوده باشد.
\newpage