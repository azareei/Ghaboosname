\addcontentsline{toc}{section}{باب بیست‌و‌پنجم - 
اندر خریدن اسب}
\section*{باب بیست‌و‌پنجم - 
اندر خریدن اسب}

اما اگر اسب خرى هشيار باش تا بر تو غلط نرود كه جوهر اسب و آدمى يكيست.  اسب‌ِ نيك و مردِ نيك را هر قيمتى كه كنى برتابد، چنانكه اسبِ بد و مردِ بد را چندانكه بتوان نكوهيدن شايد. و حكيمان گفته‌اند كه: جهان به مردم بپاست و مردم به حيوان، و نيكوتر حيوانى از حيوانات اسبست، كه داشتن او هم از كدخدايى است و هم از مروت. و در مثل گويند كه: اسب و جامه را نيكو دار تا اسب و جامه تو را نيكو دارد. و معرفت نيك و بد ايشان دشوارتر از آن مردم است كه مردم را با دعوى معنى بود و اسب را نبود بلكه دعوى اسب ديدارست تا از معنى خبر يافتن، اول به ديدار نگر كه اگر بهنر غلط كند به ديدار نكند كه اغلب اسب نيك را صورت نيكو بود و بد را بد بود. پس نيكوتر صورتى چنانكه استادان بيطره\footnote{دام‌پزشکی} گفته‌اند: بايد كه دندان باريك و پيوسته و سپيد بود و لب زيرين درازتر و بينى بلند و فراخ و كشيده و پهن پيشانى و املس\footnote{صاف، برّاق} بناگوش و دراز گوش و ميان گوش ها گشاده، آهخته\footnote{برآورده} گردن، باريك تنگ‌گاه، بُن‌ِ‌گردن سطبر و سطبر خرده‌گاه\footnote{سر دست و پای اسب و استر و خر و امثال آن باشدکه چدار و بخاو بر آن نهند و ریسمان بر آن بندند.} و زيرين قصبه\footnote{دسته موی پیچیده} كوتاه‌تر از زورين، خرد موى و سُم‌هاى وى دراز و سياه و گِردْ پاشنه و بلند پشت، كوتاه تهى‌گاه، فراخ سينه، ميان دست و پاي‌هاى او گشاده، دم او گشن\footnote{انبوه} و دراز، پره‌ی دم او باريك و كوتاه، سياه خايه و سياه چشم و مژه، و اندر راه رفتن هشيار، ماليده خردگاه، كوتاه پشت، معلق سرين، عريض كفل\footnote{میان دو ران}، درون سوى ران او پر گوشت و به هم در رسته؛ و چون سوار بر خويشتن حركت كند بايد كه از حركت مرد آگاه باشد.

و اين هنرها كه گفتم بايد كه على‌الاطلاق\footnote{بدون قید و شرط} در هر اسبى بُوَد تا نيك بود. و آنچه در اسبى بود و در دگرى نبود هنر رنگهاست. و از همه رنگ‌ها بهتر كُمِيت\footnote{اسب} بهيم\footnote{اسب یکرنگ} خرما گونست كه هم نيكو بود و هم در سرما و در گرما صبور و رنج‌كش باشد. اما اسب چرمه\footnote{اسب سفیدی موی} ضعيف بود، لكن اگر خايه و ميان ران‌ها و كون و دم و دست [و] پاى و برش و ناصيه\footnote{ موی پیش سر} و دم سياه باشد نيك بود. و اسب زرده آن جنس [نيك بود] كه به غايت زرده بود و بر وى درم درم و بش و ناصيه و دم و خايه و كون و ميان ران و چشم و لب او اين همه سياه بود. و اسب سمند بايد كه چنين بود. و گلگون بايد كه يك رنگ بود و هيچ به ابلقى\footnote{دو رنگی} نزند. و ادهم\footnote{اسپ خاکسترگون که سیاهی آن بر سپیدی غالب باشد} بايد كه سياهى بريق بود و نه بايد كه سرخ چشم بود كه بيشتر اسب سرخ [چشم] ديوانه بود و معيوب. و اسب پور\footnote{فرزند} كم بود كه نيك بود. و ابرش\footnote{اسب که نقطه های خرد سفید دارد} بد بود خاصّه كه چشم و كون و خايه و سم او سپيد بود.

و اسب ديزه\footnote{ اسب  را گویند که از کاکل تا دمش خط سیاه کشیده شده باشد} كه سياه قوايم بود بر آن صفت كه زرده را گفتيم نيك بود. و اسب ابلق ناستوده است و نيك خود كم بود. و چون هنرهاى اسبان بدانستى عيب‌ها نيز بدان كه در اسبان چند گونه عيب است. عيبى [كه] به كار زيان دارد و به ديدار زشت بود، و باشد كه نه چنين بود و لكن ميشوم\footnote{ناخجسته} و صاحب‌كُش [بود]، و باشد كه با علت‌ها و با خوي‌هاى بد بود كه بعضى بتوان برد و بعضى نتوان. و هر عيبى و هر علتى را ناميست كه بدان نام بتوان دانستن چنانكه ياد كنم. بدان كه عيب است اسب يكى آنست كه گنگ بود و اسب گنگ راه بسيار گم كند و علامتش آنست كه چون ماديانى بيند اگر چه نر فروهلد بانگ ندارد. و اسب اعشى، يعنى شب‌كور، بد بود و علامتش آن بود كه به شب از چيزى كه ديگر اسبان برمند نرمد و هر جاى بد كه برانى برود و پرهيز نكند. و اسب كر بد بود و علامتش آنست كه چون بانگ اسبان شنود، جواب ندهد و مادام گوش باز پس افگنده بود.

و اسب چپ بد بود و خطا بسيار كند و علامتش آنست كه چون او را به دهليزى\footnote{میان در و خانه} اندر كشى، نُخُست دست چپ اندر نهد. و اسب اعمش\footnote{کسی که به سبب مرض ، آب از چشمش جاری شود} آن بود كه روز بد بيند و علامتش آنست كه حدقه‌ی چشم وى سياهى بود كه به سبزى زند و مادام چشم گشاده دارد چنانكه مژه بر هم نزند و اين عيب باشد [كه در يك چشم باشد] و بود كه هر دو چشم بود. و هر چند به ظاهر اسب احول\footnote{لوچ ، دو بین ، کسی که همه چیز را دوتایی می بیند}  معيوب بود اما عرب و عجم متفق‌اند كه مبارك بود و چنين شنيدم كه دُلْدُلْ\footnote{اسب پیامبر} احول بوده است. و اسب ارجل\footnote{اسب با یک پای سفید} و اعصم\footnote{اسبی با یک پای سفید} يعنى پاى سپيد شوم بود و اگر به پاى چپ يا به دست چپ سپيد بود شوم‌تر بود. و اسب ازرق\footnote{کبود چشم} اگر به هر دو چشم ازرق بود روا بود، اما اگر به يك چشم ازرق بود خاصه به چپ بد بود. و اسب مَغرِب بد بود يعنى سپيد چشم. و اسب پوزه نيز بد بود [و اسب اقود نيز بد بود] يعنى راست‌گردن و چنين اسب اندر وحل نيك برنگردد. و اسب خول هم بد بود آنكه هر دو پايش كژ بود به پارسى كمان‌پاى خوانند بسيار بيوفتد. و اسب قالع شوم بود آنكه بالاى كاهل گردباى موى دارد. و مهقوع\footnote{صاحب هَقعَه(لکه سفید) }  همچنين آنكه گردبا زير بغلش بود، اگر به هر دو جانب بود شوم‌تر بود. و اسب فرستون\footnote{قپان که بارها بدان سنجند و آن را کپان گویند یعنی بزرگ و قپان معرب آن شده} هم شوم بود كه گردباى سم دارد، از درون سو و از برون روا بود. و اسدف\footnote{ اسب که رانها نزدیک و سمها دوردور نهد و در هر دو بند دست وی اندک پیچیدگی بود و جانب راست سم آن به بیرون رویه مایل باشد. و اگر جانب چپ باشد آنرا اقفد نامند} نيز بد بود يعنى كه سم در نوشته و آن را احنف نيز گويند. و آنكه دستش درازتر بود يا پايش، هم بد بود به نشيب و فراز و آن را اقرن خوانند و اسب اعزل هم [بد] بود يعنى كژ دم و وى را اكشف خوانند از آنچه مادام عورتش پيدا بود و اسب سگ دم نيز بد بود و اسب افحج نيز بد بود آنكه پاى بر جاى دست خويش نتواند نهادن و اسب اسبق بد بود دايم لنگ بود و آن آن بود كه بر مفاصل غدد دارد و اسب عرون هم بد بود و آن آن بود كه استخوان در مفاصل دست دارد و اگر در مفاصل پاى دارد اقرن خوانند و هم بد بود و مانع الركاب و سركش و شموس و گزنده و بسيار بانگ و ضرّاط\footnote{تیز آواز} و لگدزن و آنكه سرگين افگند درنگ كند و آنكه نر خويش بسيار فروهلد اين همه بد بود و اسب زاغ چشم شب كور بود.

شنيدم در حكايتى كه چوپان احمد فريغون روز نوروز پيش وى رفت بى‌هديه‌ی نوروزى و گفت: زندگانى خداوند دراز باد! هديه‌ی نوروزى نياورده‌ام از آنچه بشارتى به از هديه دارم. فريغون گفت: بگوى، چوپان گفت: تو را دوش هزار ماديان كره‌ی زاغ چشم بزاده است. احمد وى را صد چوب بفرمود زدن، گفت: اين چه بشارت بود كه مرا آوردى كه تو را هزار كره‌ی شب كور بزاد؟ اكنون چون اين كه گفتم بدانستى از علت‌هاى اسبان نيز آگاه باش كه هر يك را ناميست: نام رنج‌هاى اسبان انتشار، كعاب، ودحنن، و مشش، و عرن، و شقاق، و جمع و قمع، و ناصور، و جذام، و برص، و جرد، و نمله، و ملح، و نفخه، و فقد، و ارتهاش، و سرطان، و فتق، و مكتاف، و فعاس، و خناق، و ديوه، و معل، و عصاص، و ستل، و شفشى، و سعار، و رهصه و بره.

اين علتها را مجمل گفتم اگر همه را تفسير كنم دراز گردد و اين [همه كه گفتم عيب است و پيرى از همه عيب‌ها بَتَر كه هر عيبى كه بود] بتوان برد مگر عيب پيرى كه نتوان برد. اما اسب بزرگ خر تا پنج دانگ كه اگر چه مرد بهى و منظرانى\footnote{دیدنی} باشد بر اسب كوچك حقير نمايد. و بدان كه پهلوى اسبان بيشتر از جانب راست يك استخوان زيادت باشد بشمار اگر هر دو با يكديگر راست باشد از انچه ارزد زيادت بخر كه كم اسبى ازو سبق بتواند برد. هر چه خرى از چهار پاى و ضياع و عقار و غير آن چنان خر كه تا تو زنده‌اى منافع آن به تو مى‌رسد، بى‌شك آخر تو را هم زن و فرزند بود روزى، چنانكه كسى گويد: هر كه او مردست جفت او زن بايد.



\newpage
