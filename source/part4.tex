\addcontentsline{toc}{section}{باب چهارم - در فزونی طاعت از راه توانش}
\section*{باب چهارم - در فزونی طاعت از راه توانش}


بدان اى پسر! كه خداى عزّوجل دو فريضه پيدا كرد، از بَهرِ مُنعمان و بندگان خاص، و آن حجّ‌ است و زكات؛ و فرمود تا هر كه را سازِ بُوَد، خانۀ او زيارت كند؛ و آنان [را] كه ساز ندارند نفرمود؛ نبينى كه در دنيا معاملۀ درگاهِ پادشاه، هم خداوندانِ ساز توانند كرد؛ و ديگر كه اعتمادِ حج بر سفر است؛ و بى‌سازان را سفر فرمودن نه از دانشْ بُوَد، و بى‌ساز سفر كردن اندر تهلكه\footnote{هلاک شدن، نابودشدن} بود؛ و چون ساز باشد و سفر نكنى، خوشی و لذت و نعمتِ دنيا به تمامى نيافته باشى؛ كه تمامیِ خوشى و لذت نعمت جهان در آن است كه ناديده ببينى و ناخورده بِخورى و نايافته بيابى؛ و اين جز در سفر نَبُوَد كه مردم سفرى، جهان ديده و كارآزموده و روزبه باشد، و به همه كارها ناديده ديده‌باشد و ناشنيده شنيده؛ چنانكه بتازى گفته‌اند: «ليس الخبر كالمعاينة» و به پارسى گفته‌اند:
\begin{quote}
\centering
جهان ديدگان را و ناديدگان \quad \quad
نكردند يكسان پسنديدگان
\end{quote}

پس آفريدگار تقدير سفر كرد بر خداوندانِ نعمت، تا دادِ نعمت بدهند و بَرْ از نعمت بخورند و فرمانِ خداى تعالى به جاى آرند و خانۀ او را زيارت كنند؛ و درويش و بى‌ساز را نفرمود، چنانكه در دو بيت من گويم:

\begin{quote}
\centering
گر يار مرا نخواند [و] با خود ننشاند\quad \quad
وز درويشى مرا چنين خوار بماند


معذورست او كه خالق هر دو جهان \quad \quad 
درويشان را بخانۀ خويش نخواند
\end{quote}

چه درويش اگر قصدِ حج كند، خود را در تهلكه افگنده باشد؛ چه هر درويش كه كارِ توانگران كند، چون بيمارى بُوَد كه كارِ تن‌درستان كند، و داستان او راست بدان داستان ماند كه آورده‌اند:

% حکایت
% \HekaiatBegin
 شنيدم كه درويشى و توانگرى وقتى قصد خانۀ خداىْ كردند؛ و گويندْ كه آن توانگرْ رئيسِ بخارا بود، و مردى سخت منعم بود؛ و در ان قافله از او مُنْعِم‌تر كَسْ نبود، و فزون از صد تا اشتر زير بار او بود؛ و اندر عمارى نشسته بود، خرامانْ و نازانْ در باديه همى شد، با ساز و آلتى كه در حَضَرْ باشد؛ و بسيار قوم از درويش و توانگر با او هم‌راه بودند. چون نزديك عرفات رسيدند، درويشى همى آمد؛ برهنه پاى و آبله كرده و تشنه و گرسنه؛ وى را ديد بدان ناز و تن آسانى؛ روى بدو كرد و گفت: «وقت مكافات جزاىِ من و تو هر دو يكى خواهد بود؟ تو در ان نعمت همى روى و من درين شدّت همى رَوَم!» رئيس بخارا روى بدو كرد و گفت: «حاشا كه خداى تعالى جزاىِ منْ چونْ جزاىِ تو دهد. اگر من دانستمى كه تُرا و مرا يك پايگاه خواهد بودن، هرگز در باديه نيامدمى». درويش گفت: «چرا؟». گفت:«من فرمان خداى تعالى را ميكنم و تو خلافِ فرمانِ خداى.مرا خوانده‌اند و من ميهمانم و تو طُفِيْلى؛ حشمتِ طفيلىْ چون حشمتِ مهمانْ نباشد؛ خداى تعالى حَجِّ توانگران را فرمود، و درويشان را گفت «وَ لا تُلْقُوا بِأَيْدِيكُمْ إِلَى التَّهْلُكَةِ»؛ و تو بى‌فرمان خداى تعالى به بيچارگى و گرسنگى در باديه آمدى، و خود را در تهلكه انداختى، و فرمان خداى تعالى را كار نبستى، با فرمانْ بردارى چرا برابرى جويى‌؟».
% \HekaiatEnd


هر كس كه استطاعت دارد و باستطاعت حجْ كند، همچنان باشد كه دادِ نعمتْ داده باشد، و فرمان خداى تعالى عزّوجل به جاى آورده؛ پس چون تُرا سازِ حج باشد، در طاعت تقصير مكن؛ و ساز حج پنج چيزست: مِكنت\footnote{توانگری ، نیرو، ثروت}
 و نعمت و مدت و داد حرمت و امن و راحت. چون ازين بهره يافتى، جَهْد كن بر تمامى؛ و بدان كه حج طاعتيست كه بَرْدايم چون ساز بُوَد؛ اگر نيّت قهر خود در سال مستقبل معلّق كنى، نيّت قهر امام ازو منقطع كند؛ و لكن زكات طاعتى است كه به هيچ گونه، چون مكنت بود، نادادنْ عُذرى نيست؛ و خداى تعالى زكاتْ دهندگان را از مقربان خود خواند؛ و مثال مردم زكات دهنده در ميانِ مردم ديگر چون مثال پادشاه است در ميان رعيّت؛ كه روزى‌ده او بود و ديگران روزى‌خوار؛ و خداى تعالى تقدير كرد تا گروهى درويش باشند و گروهى توانگر؛ و توانا بُوَد بدان كه همه را توانگر كردى؛ و لكن دو گروه از ان كرد تا منزلت  و شرف بندگان پديد شود، و بَرتَران از فروتَران پيدا شوند؛ چون پادشاهى كه يك رَهى را روزى ده قومى كند؛ پس [اگر] اين رهى كه روزى‌ده باشد روزى خورد و ندهد، از خشمِ پادشاه ايمن نباشد؛ نيز اگر منعمْ روزى خورد و زكات ندهد، از خشمِ خداى تعالى ايمن نباشد. اما زكات در سالى يك‌بار بر تو فريضه است. لكن صدقه اگر چه فريضه نيست، در مُروَت و مردمى است؛ چندانكه مى‌توانى هَمى دِه و تقصير مَكُن كه مردمِ صدقه‌ده پيوسته در امنِ خداى تعالى باشند؛ و ايمنى از خداى تعالى به غنيمت بايد داشت. و زنهار باد بر تو كه در نهادِ حج و زكات، دل با شك ندارى؛ و كارِ بيهوده نسگالى\footnote{سگالیدن: اندیشه نمودن، اندیشیدن} و نگويى كه دويدن و برهنه كردن خويشتن را و ناخن و موى ناچيدن چراست‌؟ و ز بيست دينار چرا نيم دينار ببايد داد، و از گاو و گوسفند و اشتر چه مى‌خواهند و چرا قربان كنند؟ درين جمله دلْ پاك‌ دار و گمانْ مَبَر كه آنچه تو دانى، چيزيست كه چيزِ خودْ آنست كه ما ندانيم. تو به فرمان‌بردارىِ خداوند تعالى مشغول باش؛ ترا با چون و چرا كار نيست؛ و چون اين فرمان خداى تعالى به جاى آوردى، حقِ پدر و مادر بشناس، كه حقْ شناختنِ پدر و مادر، هم از فرمانِ خداى تعالى است.


\newpage










