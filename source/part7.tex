\addcontentsline{toc}{section}{باب هفتم - 
در پيشى جستن از سخن‌دانى}
\section*{باب هفتم - 
در پيشى جستن از سخن‌دانى}

بايد كه مردمْ سخن‌گوى و سخن‌دان باشد؛ اما تو اى پسر! سخن‌گوى باش و دروغ‌گوى مباش. خويشتن را به راست‌گويى معروف كن، تا اگر وقتى به ضرورت دروغ گويى، از تو بپذيرند؛ و هر چه گويى، راستگوى؛ ولكن راستِ به دروغْ‌مانندْ مگوى، كه دروغِ به راستْ، همانا بِهْ از راستِ به دروغ، همانا كه آن دروغْ مقبول بُوَد و آن راستْ نامقبول. پس از راست‌گفتنِ نامقبول بپرهيز، تا چنان نيوفتد كه مرا با اميرْ ابُوالْسَوارْ شاوربْن‌ِالفضل -رحمةاللّه‌عليه- افتاد.

% حكايت: \\

% \HekaiatBegin
بدان كه من به روزگارِ اميرْابُوالْسَوار، آن سال كه از حجِ اسلام باز آمدم، به غزا\footnote{ جنگ کردن ، جنگ کردن با دشمن دین}
 رفتم به گنجه؛ كه غزاىِ هندوستان خود بسيار كرده بودم، خواستم كه غزاى روم نيز كرده شود. و ابُوالسَوار مردى [بر جاى] و خردمند بود، و پادشاهى بزرگ و سايس\footnote{سیاستمدار، باسیاست، باکیاست}
 و عادل و شجاع و فصيح و متكلّم و پاك‌دين و پيش‌بين، چنانكه ملوك پسنديده باشند، و همه جِدّْ بودى بى‌هزل. چون مرا بديد، بسيار حشمت كرد و با من در سخن آمد؛ و زِ هَرْ نوعى همى‌گفتم و همى‌پرسيد و من همى شنودم و جواب همى‌دادم؛ و سخن‌هاى من او را پسنديده آمد؛ با من بسيار كرامت‌ها كرد و نگذاشت كه بازگردم؛ و زِ بَسْ احسان‌ها كه همى كرد با من، من نيز دل بنهادم و چند سال به گنجه مقيم شدم؛ و پيوسته به طعام و شراب، در مجلسِ او حاضر بودمى؛ و از هر گونه سخن‌ها از من همى پرسيدى، از حالِ عالَمْ و ملوكِ گذشته. تا روزى از ولايتِ ما سخن همى‌رفت و از حال ناحيتِ گرگان از من همى‌پرسيد. من گفتم: به روستاىِ گرگان، اندر كوه، دِهي‌ست؛ و چشمه‌ی آب از ده دورَست؛ و زنان كه آب آرند، گروهى گِرد آيند و هر كسى با سبويى؛ و از آن چشمه آب بردارند و سبوى بَرْ سَرْ نهند و جمله باز گردند. يكى از ايشان بى‌سبوىْ از پيش ايشان همى‌آيد و به راه اَندر همى‌نگرد؛ و كِرْمي‌ست سَبزْ اندر زمين‌هاى آن دِهْ، هر كجا كه آن كِرم همى‌يابد، از راه يكسو همى‌افگند، تا اين زنان پاى بر آن كرم ننهند. چه اگر كسى از ايشان پاى بر آن كرم نهد و كرم زير پاى او بميرد، اين آب كه اندر سبوىِ بَرْ سَرْ دارد، دَرْ وَقتْ گَنده شَوَد صَعْب، چنانكه بِبايد ريختن و باز بايد گشتن و سبوى شُستن و ديگر باره آب از چشمه برداشتن. چون من اين سخن بگفتم، امير ابُوالسَوار روىْ تُرش كرد و سر بگردانيد و [چند روز] با من [نه] بر آن حال بود كه پيش از آن بوده بود. تا پيروزان ديلم با من بگفت كه: امير گله‌ی تو كرد و گفت: فلان مردى برجايست چرا بايد كه [با] من سخن چنان گويد كه با كودكان گويند؛ چونْ او مردى را پيشِ منْ، دروغ چرا بايد گفت‌؟ من دَرْحال\footnote{فوراً} قاصدى را از گنجه به گرگان فرستادم و محضرى فرمودم كردن به شهادت رئيس و قاضى و خطيب و جمله‌ی عدول و علما و اشرافِ گرگان كه: اين ده برجاست و حالِ اين كِرم برين جمله است. و به چهار ماه اين درستى بياوردم و محضر پيش اميرْابُوالْسَوار نهادم و بديد و بخواند و تبسم كرد و گفت: من خود دانم كه از چونْ تويى دروغ نيايد، خاصّه پيش چون منى؛ اما خودْ آن راست چه بايد گفتن كه چهار ماه روزگار بايد و محضرى به گوايىِ دويست مرد عدول تا آن راست از تو قبول كنند؟ 
 % \HekaiatEnd
 
 
 اما بدان كه سخن از چهارْ نوع است: يكى نه دانستنى است و نه گفتنى، و يكى [هم] دانستنى است و هم گفتنى، و يكى گفتنى است و نادانستنى، و يكى دانستني‌ است و ناگفتنى. اما ناگفتنى و نادانستنى: سخنى است كه دين را زيان دارد؛ و آنكه گفتنى است و نادانستنى: سخنى است كه در كتابِ خداى‌تعالى و در اخبارِ رسول -صلى‌اللّه‌عليه‌و‌‌سلم- باشد، و اندر كتاب‌هاى علوم و علما كه در تفسير او تقليد بود و در تأويل او تعصب و اختلاف چون يد و وجه و نزول و مانند اين. پس اگر كسى دل در تأويل آن بندد خداى عز و جل او را بدان بگيرد؛ و آنكه هم گفتنى است و هم دانستنيست: سخنى بُوَد كه صلاحِ دين و دنيا در آن بُوَد و هم بدين جهان به كار آيد و هم بدان جهان؛ و از گفتن و شنودنِ آن گوينده [و شنونده] را نفع بُوَد؛ و آنكه دانستنى است و ناگفتنى: چنان بُوَد كه عيبِ محتشمى يا عيبِ دوستى تُرا معلوم شود، يا از طريقِ عقل يا از كارِ جهان تو را تخايلى\footnote{تکبّر و تبختر} بندد، كه آن نه شرعى بُوَد؛ چون بگويى يا خشمِ آن محتشمْ تو را حاصل آيد، يا آزارِ دوستْ حاصل شود، يا بيمِ شوريدنِ غوغا [و] عامّه باشد بر تو، پس اين سخن دانستنى بُوَد و ناگفتنى. اما ازين چهارْ نوع كه گفتم، بهترينْ آن سخن است كه هم دانستنى است و هم گفتنى. اما اين چهار نوع سخن، هر يكى را دو رويست: يكى نيكو و يكى زشت؛ سخن كه به مردم نمايى، بر روىِ نيكوترين نِماى، تا مقبول بُوَد و مردمان درجه‌ي تو بشناسند؛ كه بزرگان و خردمندان را به سخن  دانند، نه سخن را به مردم؛ كه مردم نهانست زيرِ سخنِ خويش، چنانكه به تازى گويند: «المرء مخبوء تحت لسانه»؛ و سخن بُوَد كه بگويند به عبارتى، كه از شنيدنِ آن روح تازه گردد، و همان سخن به عبارتىْ ديگر توان‌ْ گفتن، كه روح تيره گردد.




% حكايت:\\
% \HekaiatBegin
 شنيدم كه هارون‌الرشيد خوابى ديد، بر آن جمله كه پنداشتى كه همه دندانه‌اى او از دهن بيرون افتادى به يكبار. بامداد معبرى\footnote{تعبیر کننده} را بياورد و پرسيد كه: تعبير اين خواب چيست‌؟ معبر گفت: زندگانى امير‌المؤمنين دراز باد، همه اقرباىِ تو، پيش از تو بميرند، چنانكه كس از تو باز نماند. هارون گفت: اين مرد را صد چوب بزنيد، كه بدين دردناكى سخنى در روىِ من بگفت؛ چون همه قراباتِ من، پيش از من جمله بميرند، پس آنگه من كه باشم‌؟ خواب‌گزارى ديگر بياوردند و همين خواب با وى بگفت. خواب‌گزار گفت: بدين خواب كه امير‌المؤمنين ديد، دليل كند كه خداوندْ درازْ زندگانى‌تَر بُوَد از همه قراباتِ خويش. هارون گفت: «طريق العقل واحد». تعبير از آن بيرون نشد، اما از عبارت تا عبارت بسيار فرق است؛ اين مرد را صد دينار بدهيد.
 % \HekaiatEnd
 
 و حكايتى ديگرم ياد آمد اگر چه نه حكايت كتابست و لكن «النادرة لاترد». 

%  \HekaiatBegin
 شنودم كه مردى با غلام خويش خفته بود؛ غلام را گفت: كون\footnote{ماتحت، باسن} زين سو كن. غلام گفت: اى خواجه اين سخن نيكوتر ازين بِتَوانْ گفت. مرد گفت:
چون گويم‌؟ غلام گفت: بگوى كه روىْ از آن سو كن، كه اندر هر دو سخنْ غرض يكي است، اما تا به عبارت زشت نگفته باشى. مرد گفت: شنودم و بياموختم و به جرمِ آن ناشايست كه گفتم، تُرا آزاد كردم. پس پشت‌و‌روىِ سخنْ نگاه بايد داشت و هر چه گويى بر روىِ نيكوترْ بايد گفتن، تا هم سخن‌گوى باشى و هم سخن‌دان. اگر گويى و ندانى، چه تو و [چه] آن مُرغَك كه او را طوطى خوانند، كه وى نيز سخن گوي‌ است اما نه سخن‌دان است. و سخن‌گوى و سخن‌دان آن بود، كه هر چه گويد مردمان را معلوم شود تا از جملۀ عاقلان بُوَد و اگر [نه] چنين باشد بهيمه‌\footnote{چهارپا}اى باشد مردمْ‌پيكر.
% \HekaiatEnd

اما سخن را بزرگ دان كه از آسمان سخن آمد و هر سخن كه بدانى از جايگاه سخن دريغ مدار و به ناجايگاه ضايع مكن، تا بر دانش ستم نكرده باشى. اما هر چه گويى، راست گوى؛ دعوى كننده‌ي بى‌معنى مباش و اندر همه دعوي‌ها بُرهانِ كمتر شناس و دعوى بيشتر و به علمى كه ندانى دعوى مكن و از آن عِلْم نان مَطَلَب كه غرضِ خويش از آن علم و هنر به حاصل توانى كردن كه معلومِ تو باشد و به چيزى كه ندانى به هيچ‌ چيز نرسى.

% حكايت:\\

% \HekaiatBegin
 چنانكه گويند كه به روزگار خسرو زنى پيش بزرجمهر آمد و از وى مسئله‌اى پرسيد و در آن حال بزرجمهر سر آن سخن نداشت گفت: اى زن اين كه تو همى‌پرسى من ندانم. اين زن گفت. پس تو كه اين ندانى اين نعمت خدايگان ما به چه چيز مى‌خورى‌؟ بزرجمهر گفت: بدان چيز كه دانم؛ و بدان كه ندانم مَلِك مرا چيزى نمى‌دهد ور باور ندارى بيا و از مَلِك بپرس تا خود بدانچه ندانم مرا چيزى همى دهد يا نه‌؟
% \HekaiatEnd


اما اى پسر! اندر كارها افراط مكن و افراط را شوم دان و اندر همه شغلى ميانه باش كه صاحب شريعت ما -صلى‌اللّه‌عليه‌و‌سلم- گفت: «خير الأمور اوساطها». و در سخن گفتن و شغل گذاردن، گران‌سنگى\footnote{سنگینی، وقار داشتن، آهستگی، متانت} عادت كن و اگر از گران‌سنگى و آهستگى نكوهيده گردى، دوست‌تر دار از آنكه از سبكسارى و شتاب‌زدگى ستوده گردى؛ و به دانستنِ رازى كه به تو تعلّق ندارد، رغبت مكن؛ و جز با خودْ رازِ خويش مگوى، اگر بگويى آن سخن را زان پَسْ رازْ مخوان؛ و پيشِ مردمان، با كسْ رازْ مگوى، كه اگر چه درون‌سو سخن نيكو بُوَد، از بيرون‌سو گمان به زشتى برند كه آدميان بيشتر به يك‌ديگر بَدْگمان باشند. و در هر كارى سخن و همت و حال به اندازه‌ی مال دار. هر چه گويى [آن گوى] كه به راستیِ سخنِ تو گوايى دهد و اگر چه به نزديك مردمان سخن گوى [صادق] باشى؛ و اگر نخواهى كه بستم خود را معيوب كنى بر هيچ چيز گوا\footnote{مُؤَیِّد} مشو، پس اگر شَوى به وقت گوايى دادن احتراز كن؛ پس اگر گواهى دهى به ميل مده. و هر سخنى كه بگويند بشنو ولكن به‌كار‌بستن شتاب‌زده مباش. و هر چه بگويى نا‌انديشيده مگوى و هميشه انديشه را مُقَدّمِ گفتار دار، تا بر گفته پشيمان نشوى؛ كه پيش‌انديشى دوم كفايت است. و از شنودنِ هيچْ سخن ملول مباش، اگرت به كار آيد و اگر نه بشنو تا دَرِ سخنْ بر تو بسته نَبُوَد و فايده‌ي سخن فوت نگردد. و سرد سخن مباش، كه سخنِ سَرْد تخميست كه ازو دشمنى رويد. و اگر [چه] دانا باشى، خود را نادان شمر، تا دَرِ آموختن بر تو گشاده گردد. و هيچ سخن را مشكن و مستاى، تا نخست عيب و هنر آن تُرا معلوم نگردد. و سخن يك‌گونه گوى با خاص، خاص و با عام، عام تا از حدِّ حكمت بيرون نباشى و بر مستمع وبال نگردد. مگر در جايى كه از تو دَرْ سخن گفتن دليل و حجت نشنوند، آنگه سخن بر مرادِ ايشان همى گوى تا به سلامت از ميانِ قوم بيرون آيى. و اگر چه سخن‌دان باشى از خويشتن كمترْ آن نماى كه دانى، تا به وقت گفتار و كردار پياده نمانى. و بسيار دان و كم‌گوى باش، نه كم‌دانِ بسيارْ‌گوى، كه گفته‌اند كه: خاموشى دَوْمِ\footnote{دوام، همیشگی} سلامت است و بسيار گفتن دَوْمِ بى‌خردى؛ از آنكه بسيار‌ْگوى، اگرچه خردمند باشد، مردمان عامه او را از جمله‌یِ بى‌خردان شناسند، و اگر چه بى‌خرد كسى باشد چون خاموش باشد، مردمان خاموشى او از جمله‌یِ عقل دانند. و هر چند پاكْ‌رَوِش و پارسا باشى خويشتن‌‌ستاى مباش، كه گواهىِ تو برْ تو كَسْ نَشنَوَد و بكوش تا ستوده‌یِ مردمان باشى نه ستوده‌ی خويش. و اگر چه بسيار دانى، آن گوى كه به كار آيد، تا آن سخن بر تو وِبال نگردد، چنانكه بر آن علوىِ زنگانى. حكايت شنيدم كه به روزگار صاحب، پيرى بود به زنگان، فقيه و محتشم، و از اصحاب شافعى مطلبى بود، رحمة‌اللّه‌عليه، مفتى و مزكّى و مذكّرِ زنگان بود. و جوانى [علوى] بود پسرِ رئيس زنگان، فقيه بود و هم مذكّرى كردى. و پيوسته اين هر دو را با يك ديگر مكاشفت بودى و بر سر كرسى، يك ديگر را طعن‌ها زدندى. اين علوى، روزى بر سر كرسى اين پير را كافر خواند؛ خبر بدين شيخ رسيد، وى نيز اين علوى را بر سر كرسى حرام‌زاده خواند. خبر به علوى بردند، سخت از جاى بشد. در وقت بر نشست و به شهر رى رفت و پيش صاحب، از آن پير گله كرد و بگريست و گفت: شايد كه به روزگار تو كسى فرزند رسول را حرام‌زاده خواند؟ صاحب ازين خبر در خشم شد و قاصد فرستاد و آن پير را به رِى خواند؛ و به مظالم بنشست با فقها و سادات رى، و اين پير را بفرمود آوردن و گفت: اى شيخ تو مردى از جمله‌ی امامانِ اصحابِ شافعى باشى، مردى عالِم و پيرُ و به لبِ‌ گور رسيده، شايد كه فرزند پيغامبر را حرام‌زاده خوانى‌؟ اكنون اين كه گفتى درست كن يا نه تُرا عقوبتى هر كدام سخت‌تر بكنم تا خلق از تو عبرت گيرد و ديگر كس اين بى‌ادبى و بى‌حرمتى نكند، چنانكه اندر شرع واجبست. آن پير گفت: برين سخن درستى، گواى من خود اين علويست، بر نفس او خود [به از] او گواه مخواه. اما به قول من او حلال زاده‌ی پاكست و به قول او حرام‌زاده است. صاحب گفت: به چه معنى‌؟ آن پير گفت كه: همه زنگان دانند كه نكاح مادر او با پدر او من بسته‌ام و وى بر سرِ مِنبر مرا كافر خوانده است؛ اگر اين سخن از اعتقاد گفته است پس نكاحى كه كافر بندد درست نباشد پس به قول او بى‌شك حرام‌زاده بُوَد. پس اگر نه به اعتقاد گفت، دروغ‌زَنْ است [و حدّ بر وى لازم آيد. اكنون به همه حال يا دروغ‌زن ست] يا حرام‌زاده و فرزند پيغامبر -عليه‌السلام- دروغ‌زن نباشد چنانكه شما خواهيد او را همى‌خوانيد كه بى‌شك ازين دو گانه بيك چيزش ببايد استادن. آن علوى سخت خجل شد و هيچ جواب نداشت و آن سخن ناانديشيده بر وى وبال شد.

پس سخن‌گوى باش، نه يافه گوى؛ كه يافه‌گويى دَوْم ديوانگيست. و با هر كه سخن گويى، همى نگر تا سخن تو را خريدار هست يا نه‌؟ اگر مشترى چرب يابى، همى فروش؛ و اگر نه آن سخن بگذار و آن گوى كه او را خوش آيد تا خريدار تو باشد.

و لكن با مردمان مردم باش و با آدميان آدمى كه مردم ديگرست و آدمى ديگر. و هر كسى كه از خواب غفلت بيدار گشت با خلق چنين زيَد كه من گفتم. و تا توانى از سخن شنيدن نفور مشو كه مردم از سخن شنيدن سخن گوى شوند، دليل بر ان كه اگر كودكى را كه از مادر جدا شود در زير زمين برند و شير همى‌دهند و همان جاى همى‌پرورند، مادر و دايه با وى سخن نگويند و ننوازند و سخنِ كس نَشنَوَد، چون بزرگ شود، لال بُوَد و هيچ سخن نداند گفتن تا به روزگار كه همى‌شنود و بياموزد. دليل بر آن كه هر كرى كه مادرزاد بُوَد لال بُوَد و ازين سببست كه همه لالان كر باشند. پس سخن‌ها بشنو و قبول كن، خاصه سخن‌ها و پندهاى ملوك و حكيمان كه گفته‌اند كه: پند حكما و ملوك شنيدن ديده‌ی خِرَد را روشن كند، كه توتياى چِشمِ خِرَد حكمت است، پس سخنِ اين قوم را بگوشِ دلْ بايد شنودن و اعتماد كردن. و ازين سخن‌ها اندرين وقت چند سخنِ نغز و نُكَت‌ْهاى بديع يادم آمد از قول نوشروان عادل، ملك‌ملوك‌العجم، اندرين كتاب ياد كردم تا تو نيز بخوانى و بدانى و ياد‌گيرى و كارْبَندْ باشى؛ كه كار بستن سخن‌ها و پندهاى آن پادشاه ما را واجب‌تر باشد كه ما از تخمه‌ی آن مِلكيم.

بدان كه چنين خوانده‌ام از اخبار خلفاى گذشته كه مأمونِ خليفه -رحمه‌اللّه- به تربت نوشروان عادل شد، آنجا كه دخمه‌ی او بُوَد، و آن قصه درازست. اما مقصود اينست كه مأمون در دخمه‌ی او رفت اعضاهاى او را يافت بر تختى پوسيده و خاك شده، و بر فرازِ تختِ وى بَرْ ديوارِ دخمه خطى چند به زر نبشته بود به خط پهلوى. مأمون بفرمود تا دبيران پهلوى را حاضر كردند و آن نبشت‌ها را بخواندند و ترجمه كردند به تازى؛ پس از تازى در عجم معروف شد. اول گفته بود كه: تا من زنده بودم همه بندگانِ خداى تعالى از عدل من بهره‌ور بودند و هرگز هيچ كس به خدمت پيش من نيامد كه از رحمت من بهره نيافت؛ اكنون چون عاجزى آمد هيچ چاره ندانستم جزين كه اين سخن‌ها برين ديوار نبشتم تا اگر وقتى به زيارت من كسى بيايد، اين لفظ‌ها بخواند و بداند، او نيز از من محروم نمانده باشد، اين سخن‌ها و پندهاى من پاىِ مُزدِ آن كس باشد و آن پندها اينست كه نبشته آمدست.


\newpage
