\addcontentsline{toc}{section}{باب سیوم - در سپاس داشتن از خداوند نعمت بر توان و ناگزیر}
\section*{باب سیوم - در سپاس داشتن از خداوند نعمت بر توان و ناگزیر}




بدان اى پسر! كه سپاسِ خداوندِ نعمتْ واجب است بر همه كسْ بَر اندازۀ فرمانْ، نه بَرْ اندازۀ استحقاق؛ كه اگر هَمِگیِ خويشْ شُكر سازد، هنوز حقِ شكرِ يك جزء
 از هزار جزء نگزارده باشد؛ جز كه بر اندازۀ [فرمان]، اگر خداوندِ نعمت، اندك شكر خواهد، بسيار بود؛ چنانكه اندازۀ طاعت در دين اسلام پنج است: دو از او خاصِ مُنْعِمان راست و سه از او عمومِ خلايق را؛ يكى از او اقرار به زبان و تصديق به دل؛ و ديگر نماز پنج‌گانه؛ و سوم روزۀ سى روز. امّا شهادت، دليلِ نفىْ است بَرْ حقيقتِ هرْ چه جُزْ از حَقْ است، و نمازِ صدقِ قول [و] اقرارِ بندگيست، و روزه تصديق قول [و] اقرار دادن به خدايى خداست؛ چون گفتى كه من بنده‌ام، در بَنْدِ بندگى بايد بود؛ و چون گفتى كه او خداوندست، در زيرِ حكمِ خداوندْ بايد بود؛ و اگر خواهى كه بندۀ تو تُرا اطاعت دارد، تو از طاعتِ خداوندِ خويشْ مگريز؛ و اگر بگريزى، از بندۀ خويشْ طاعت چشم مدار، كه نيكیِ تو بر كهتر تو، نه بيش از انست كه نيكى خداوند تو بر تو. و بندۀ بى‌طاعت مباش، كه بندۀ بى‌طاعت، خداوندْ‌خوى بُوَد، و بندۀ خداوندْخو زود هلاك شود، چنانكه شاعر گويد:

\begin{quote}
\centering
سزد گر برى بنده را تو گلو \qquad \qquad چو آيد خداونديش آرزو
\end{quote}

و آگاه باش كه نماز و روزه خاصِ خداوندْ راست؛ در او تقصير مكن، كه چون در خاصِ خداى تقصير كنى، از عامْ همچنان بازمانى؛ و بدان كه نماز را خداوند شريعت ما برابر كرد با همۀ دين. هر آن‌كس كه نماز [را] دست بازداشت، دين را دست بازداشت؛ و بى‌دين را در اين جهان، جزا كشتن است و بدنامى و به آن جهان عقوبتِ خداىِ عزّوجل. زينهار اى پسر! كه بر دل نگذارى بيهودگى و نگويى كه در نماز تقصير رواست؛ كه اگر از روىِ دينْ ياد نگيرى، از روىِ خردْ ياد گير، كه فايدۀ نماز چند چيزست: اول آنست كه هر كه نماز فريضه بجاى آرد، مادام تن و جامۀ او پاك بُوَد و به همه حالْ پاكى بِهْ كه پليدى؛ و ديگر فايدۀ نماز گزاردن آنست كه از متكبرى خالى باشى، زيرا كه اصلِ نماز بر تواضع نهاده‌اند؛ چون طبع را بر تواضع آرامست، چون طبع را بر تواضع عادت كنى، تن نيز متابع عادت گردد؛ و ديگر معلوم همه دانا آنست كه هر كس كه خواهد كه هم طبع گروهى گردد، صحبت با آن گروه بايد كردن؛ چون كسى خواهد كه بدبخت و شقى گردد، با بدبختان و شقيان صحبت كند؛ و آن‌كس كه نيك‌بختى و دولت جويد، متابع دولت خدا باشد و به اجماع همه خردمندان. نه دولتى است قوى‌تر از دولت اسلام و نه امرى است روان‌تر از امر اسلام. پس گر تو خواهى كه مادام با دولت و نعمت و راحت باشى، صحبتِ خداوندِ دولت‌ْ جوى، و فرمان‌بردارِ دولتيان باش، و خلاف اين مجوى تا بدبخت و شقى نباشى؛ و زنهار اى پسر كه اندر نماز سبكى و استهزا نكنى، بر ناتمامى ركوع و سجود و مطايبه\footnote{شوخی، مزاح، لودگی}
 كردن اندر نماز [نکنی]، كه اين عادت هلاك دين و دنيا بود.

فَصْلْ\footnote{آخر}، اما بدان كه روزه طاعتى است كه به سالى يك بار باشد، نامردمى بُوَد تقصيرْكردن، و خردمندان چنين تقصير از خويشتن روا ندارند؛ و نگر كه گرد تعصب نگردى، از انچه ماهِ روزه بى‌تعصب نبود؛ و اندر گرفتن روزه و گشادن تعصب مكن، هر گَه كه دانى كه پنج عالِم [از] تقیِ نَقى و معتقد و پرهيزگار و قاضى و خطيب و مفتى شهر روزه گرفتند، با ايشان بگير و با ايشان بگشاى و در گفتار جهال دل مبند؛ و آگاه باش كه ايزد مستغنى است، از سيرى و گرسنگى تو؛ و لكن غرض در روزه، مِهْرى است از خداوندِ مَلِك بر مُلْكِ خويش؛ و اين مِهْر نه بر بعضى از مُلْكَت است، چه بر همه تن است؛ و در روزه چون دهان را مُهر كردى، دست و پاى و چشم و گوش و زبان را به مُهر كن، و عورت را بمهر كن؛ چنانكه در شرط است منزه دارى اين اندام‌ها را از فُجور و ناشايست تا دادِ مُهر روزه بداده باشى؛ و بدان كه بزرگترينْ كارى در روزه آنست كه چون نانِ روزْ به شبْ افگنى، آن نان را كه نصيبِ روزِ خود داشتى به نيازمندان دهى، تا فايدۀ رنج تو پديد آيد؛ و آن رنج براى آن بُوَد كه منفعت آن به مستحقى رسد؛ و نگر تا در اين سه طاعت كه عامِ همه جهانست تقصير روا ندارى، كه بتقصير اين [سه] طاعت هيچ عذرى نيست؛ اما آن دو طاعت كه مخصوص است توانگران را، تقصير با عذر روا بُوَد؛ و اما اندرين باب سخن بسيارست، ولكن ما آنچه ناگزير بود اندرين باب گفتيم، اميدست كه فايده حاصل آيد، و السلام.

\newpage
