\section*{مقدمه} 

بسم اللّه الرحمن الرحيم، الحمد لالله رب العالمین، و الصلوة على رسوله محمد و آله اجمعين. چنين گويد جمع‌كنندۀ اين كتاب پندها، الامير عنصرُ المعالى كيكاوسِ بن اسكندرِ بنِ قابوسِ بن وشمگير مولى اميرالمؤمنين، با فرزندِ خويش، گيلانشاه.


بدان اى پسر! كه من پير شدم و ضعيفى و بى‌نيرويى و بى‌توشى بر من چيره شد، و منشورِ عزلِ زندگانى [را] از موىِ خويش، بر روىِ خويش كتابتى همى‌ بينم كه اين كتابت را دستِ چاره‌جويان بِسِتُردنْ نتواند. پس اى پسر! چون من نامِ خويش را در دايره‌ی گذشتگان يافتم، روىْ چنان ديدم كه پيش از آنكه نامۀ عزلْ به من رسد، نامه‌اى ديگر در نكوهش روزگار و سازش كار و بيش بهرگى جستن از نيك‌نامى ياد كنم؛ و تو را از آن بهره كنم بَر موجبِ مهرِ خويش، تا پيش از آنكه دستِ زمانه تو را نرم كند، تو خود به چشمِ عقل در [سخن] من نگرى، فزونى یابى و نيك‌نامى در دو جهان؛ و مبادا كه دلِ تو از كاربستن باز ماند، كه آنگه از من شرطِ پدرى آمده باشد؛ اگر تو از گفتارِ من بهره‌ی نيكى نجويى، جويندگان ديگر باشند كه شنودنْ و كار بستنِ نيكى غنيمت دارند؛ و اگرچه سرشتِ روزگار بر آنست كه هيچ پسر پندِ پدرِ خويش را كاربند نباشد؛ چه آتش در دل جوانان است از روىِ غفلت، پنداشتِ خويش، ايشان را بر آن نهد كه دانش خويش برتر از دانش پيران بينند؛ و اگر چه اين سخن مرا معلوم بود، مهرِ پدرى و دل‌سوزگیِ پدران مرا نگذاشت كه خاموش باشم؛ پس آنچه از موجب طبعِ خويش يافتم، در هر بابى سخنى چندْ جمع كردم و آنچه بايسته‌تر بود و مختصرتر، در اين نامه نِبشتم. اگر از تو كاربستن خيزد، خود پسنديده آمد؛ و الاّ من آنچه شرطِ پدرى بود، بجاى آورده باشم؛ كه گفته‌اند كه: بر گوينده جز گفتار نيست، چون شنونده خريدار نيست، جاىِ آزار نيست.

و بدان اى پسر! كه سرشتِ مردم چنان آمد كه تكاپوى كند، تا از دنيا آنچه نصيبِ [او] آمده باشد، به گرامی‌تَرْ كَس‌ خويش بماند؛ و نصيبِ من از دنيا اين سخن‌گفتن آمده و گرامى‌تر كس بر من تويى. چون سازِ رحيل كردم، آنچه نصيبِ من بود، پيشِ تو فرستادم تا خودكامه نباشى و پرهيز كنى از ناشايست؛ و چنان زندگانى كنى كه سزاى تخمه‌ی پاك توست، كه تو را اى پسر! تخمه بزرگ و شريفست؛ و زِ هر دو طرف كريم‌ُالطرفينى و پيوسته‌ی ملوكِ جهانى: جدّت ملك شمس المعالى قابوس بن وشمگير بود كه نبيره‌ی آغش و هادان بود، و آغش و هادان ملكِ گيلان بود به روزگار كيخسرو، و ابوالمؤيّد بلخى، ذكرِ او در شاهنامه آورده است و ملكِ گيلان از ايشان بجدّان تو يادگار بماند، و جدّه‌ی تو، مادرم، دختر ملك‌زاده المرزبان بن رستم بن شروين بود كه مصنّف مرزبان نامه است؛ سيزدهم پدرش كابوس بن قباد بود، برادرِ ملكِ انوشروان عادل؛ و مادر تو فرزند ملكِ غازىْ محمودِ بنِ ناصر الدين بود، و جدّه‌ی من فرزندِ ملكِ پيروزان ملكِ ديلمان بود. پس اى پسر! هشيار باش و قدر و قيمت نژاد خود بِشناس و از كم‌بودگان مباش؛ هرچند من نشانِ خوبى و روزبهى اندر تو همى بينم، اين گفتار بر شرط تكثر واجب ديدم.

آگاه باش اى پسر! كه روز رفتنِ من نزديكست، و آمدنِ تو بر اثرِ من زود باشد، چه امروز تا دَرين سراىِ ‌سِپَنْجى\footnote{خانه‌ی عاریتی، منزل یک‌شبه} بايد كه بر كار باشى، و زادى و پرورشى را كه سراى جاودان را شايد، بردارى؛ و سراىِ جاودانى، برتر از سراىِ سِپَنْجى است؛ و زادِ او ازين سراى بايد جست، كه اين جهان چون كشتزاريست كه ازو كارى و ازو دِرَوى، از بد و نيك؛ و كس دروده‌ی خويش در كشتزار نخورد، بلكه در آبادانى خورد؛ و آبادانیِ اين سراى، سراىِ باقيست؛ و نيكمردان، درين سراى، همتِ شيران دارند و بدمردان، همت سَگان، و سگ همانجا كه نخجير\footnote{شکار، صید} گيرد، بخورد؛ و شير چون بگيرد، به جاى ديگر خورد. و نخجيرگاهِ تو اين سراىِ سِپَنْجى است و نخجيرِ تو دانش و نيكيست. پس نخجيرْ ايدر\footnote{اینجا، اکنون}
 كن، تا وقتِ خوردن، به سراىِ باقى، آسانْ توانى خوردن، كه طريق سزاىِ ما بندگان طاعتِ خدايَست عزّوجل؛ و مانند آن كس كه راهِ خداى تعالى جويد و طاعتِ خداىِ تعالى جويد، چون آتشى بوَد، كه هرچند سرنگونش كنى، برترى و فزونى جويد؛ و مانند آن كس كه از راهِ خداىِ تعالى و طاعتِ او دور باشد، چون آبى بوَد، كه هر چند بالاش دهى، فروترى و نگونى جويد؛ پس بر خويشتن واجب دان شناختنِ راه ايزد تعالى. «واللّه ولى التوفيق». [و اين كتاب را چهل ‌و‌ چهار باب نهاده آمد]:

\newpage