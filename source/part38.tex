\addcontentsline{toc}{section}{باب سى و هشتم -
در آداب نديمى پادشاه}
\section*{باب سى و هشتم - 
در آداب نديمى پادشاه}

و اگر پادشاه تو را نديمى خويش فرمايد، اگر آلت منادمت پادشاه ندارى، مپذير، كه هر كه نديمى پادشاه كند چند خصلت اندر وى ببايد كه بود، چنانكه اگر در مجلس آن خداوند را از جلوس او زينتى نبود شينى نبود. اول بايد كه هر پنج حواس وى به فرمان وى بود و نيز بايد كه لقائى\footnote{صورت} دارد كه مردمان را از ديدن وى كراهيتى نباشد، تا اين ولى نعمت از ديدن وى ملول نشود. بايد كه اين كس دبيرى داند، تازى و پارسى، تا اگر وقتى به خلوت اندر ملك را به دبيرى حاجت بود، به خواندن يا نبشتن و دبير حاضر نبود، اين پادشاه تو را خواندن فرمايد يا نبشتن، عاجز نباشى. ديگر بايد كه اگر نديم شاعر نباشد در شعر بداند، نيك و بد نظم بر وى پوشيده نباشد و اشعار تازى و پارسى بسيار داند تا اگر اين خداوند را به گاه و بي‌گاه به بيتى حاجت افتد، شاعرى را طلب نبايد كرد يا خود بگويد يا خود روايت كند. و هم‌چنين اندر طب و علم نجوم بداند تا اگر ازين صناعت سخنى رود، يا بدين حاجت آيد، تا آنگه كه طبيب و منجم آيند تو آنچه دانى بگويى تا شرط منادمت به جاى آورده باشى، تا اين پادشاه را به هر علمى بر تو اعتماد افتد و به خدمت تو راغب‌تر بود. بايد كه اندر ملاهى تو را دست بود، چيزى بدانى زدن تا مگر خلوتى بود كه مطرب را جاى نبود تا بدانچه دانى وقت او خوش همى‌دارى تا وى را بدان سبب بر تو ولوعى ديگر باشد. و نيز چنان بايد محاكى باشى و بسيار حكايت‌هاى مضاحك\footnote{سخن خنده‌آور} و سخن مسكته\footnote{سخنی که سبب خاموشی و بروز حالت استماع گردد} و نوادرهاى بديع ياد دارى كه نديمى بى‌حكايات و نوادر ناتمام باشد. و نيز بايد كه نرد و شطرنج‌باز باشى و لكن نه چنانكه مقامر باشى، كه هر كه به طبع مقامر باشد نديمى ملوك را نشايد. و بايد كه قرآن دانى و از ظاهر تفسير خبر دارى و طرفى از فقه و اخبار رسول عليه‌السلام بدانى و از علم شرع از هر چيزى خبر دارى تا اگر در مجلس پادشاه ازين معنى سخنى رود، جواب دانى دادن و طلب قاضى و فقيه نبايد گرفتن. و نيز بايد كه بسيار سير ملوك خوانده باشى و به دانسته و بتن خويش خدمت پادشاهى كرده باشى تا پيش خداوند خصلت‌هاى ملوك ستوده‌ی ملوك گذشته همى‌گويى تا آن اندر دل پادشاه كار كند و بندگان خداى تعالى را اندران نفعى و تفرّجى همى‌باشد. و بايد كه اندر تو هم جدّ باشد و هم هزل، اما بايد كه وقت استعمال سخن بدانى كه كِى باشد و وقت جدّ، هزل نگويى و وقت هزل، جدّ نگويى، كه هر علمى كه بدانى و استعمال آن ندانى كردن، دانستن و نادانستن آن به نزديك مردمان يكى باشد. و با اين همه كه گفتم بايد كه اندر تو رجوليّتى باشد، كه اين ملك نه همه به عشرت مشغول بود، چون وقت مَردى بود، بايد كه مَردى نمايى و تو را توانايى آن بود كه با مردى و دو مرد بزنى مگر و العياذباللّه اندر ميانه‌ی خلوتى در ميان قصف و نشاط كسى خيانت انديشد برين پادشا و از جمله‌ی حوادث‌ها حادثه‌اى بيوفتد تو آنچه شرط مردى و مردمى بود، به جاى آر، تا آن ولى نعمت به سبب تو رستگارى يابد و اگر كشته شوى حق نعمت خداوند خويش به جاى آورده باشى و به نام نيكو رفته و حق فرزندان تو بر او واجب شود و اگر برهى خود نام نيكو و نان يافته باشى تا عمر تو بود.

پس اينكه گفتم اگر جمله در تو موجود نباشد، بايد كه بيشترى باشد تا تو نديمى پادشاه را شايى و اگر چنانكه از نديمى پادشاه به نان خوردن و نبيد\footnote{شراب خرما} خوردن و هزل گفتن بسنده كنى، آن لييمى بود نه نديمى، تدبير نديمى عام كن تا آن خدمت بر تو وبال نگردد. و نيز هرگز تا تو باشى پيش خداوندانِ خويش از خويشتن غافل مباش و در مجلسِ پادشاه در بندگان او منگر، چون ساقى نبيد به تو دهد هر گه كه قدح بستانى در روى وى منگر، سر پيش فگنده دار، بستان و بخور و قدح بازده چنانكه اندر وى ننگرى، تا خداوندان را از تو به خيالى صورت نبندد و خويشتن را نگاه دار تا تو را چنان نيفتد كه قاضى عبد‌الملك عكبرى را افتاد.

شنيدم كه قاضى عبدالملك عكبرى را مأمون خليفه نديمى خويش داد كه عبدالملك نبيد خواره بود و از قضا معزول بدين سبب. روزى در مجلس شراب غلامى ساقى نبيد بدين عبدالملك داد، چون نبيد بستد، به غلام اندر نگريد، به چشم به وِى اشارتى كرد، يك چشم را لختى فرو گرفت. مأمون بديد، عبدالملك بدانست كه مأمون آن بديد و آن اشارت بدانست، همچنان چشم نيم گرفته همى داشت. مأمون بعد از ساعتى پرسيد كه: اى قاضى! چشم تو را چه افتاد؟ عبدالملك گفت: ندانم يا اميرالمؤمنين! اندرين ساعت به هم فراز آمد. و بعد از آن تا وى زنده بود، در سفر و در حضر و در ملا و در خلا، هرگز چشم تمام باز نگشاد، تا آن تهمت از دل مأمون ببرد.

پس آن كس كه او نديم پادشاه يا هم‌نشين بزرگان بود چنين بود و چنين كفايتى بايد تا دانسته باشى كه نديمى پادشاه كردن نه از گزافست.



\newpage


