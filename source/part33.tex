\addcontentsline{toc}{section}{باب سى و سوم - 
اندر ترتيب علم طب}
\section*{باب سى و سوم - 
اندر ترتيب علم طب}

اگر طبيب باشى بايد كه علم اصول طب بدانى نيك، چه اقسام عِلمى و چه اقسام عَمَلى. و بدانى كه آنچه در تنِ مردم موجودست يا طبيعي‌ است يا خارج از طبيعت.

و طبيعى بر سه قسمست: يكى قِسْم از وِى، آنست كه قوام و ثباتِ تَنْ بِدوست\footnote{به آن است} و يك قسم توابع‌ است اين چيزها را كه قوام و ثبات بِدوست و يك قسم آنست كه تن را از حال به حال بگرداند. و آنكه خارج از طبيعت است يا به فعل مَضّرَت\footnote{ضرر،‌آسیب} رساند يا به واسطه يا بى واسطه، يا خود نفس ضَرَرِ فعل بود. اما آن قسم كه ثبات و قوام تن بدوست، يا از جنس مادت است يا از جنس صورت، آنكه از جنس مادت است يا سخت دورست چون استقصات\footnote{جِ اُستُقُص یا اسطقس . این کلمه در اصل یونانی است و به معنی ماده و اصل هر چیزی} و عددش چهارست: آتش و هوا و آب و خاك، يا نزديك‌ترست از استقصات چون امزجه\footnote{جمعِ مزاج ؛ طبیعت ها، سرشت ها} و عددش نُه است: يكى معتدل و هشت نامعتدل: چهار مفرد و چهار مركّب. يا نزديك‌تر از امزجه است چون اخلاط\footnote{جمعِ خِلط؛ چیزهای درهم آمیخته؛ در طب قدیم صفرا وخون و بلغم و سودا} و عددش چهارست: چون صفرا و سودا و بلغم و خون، يا نزديك‌تر از اخلاط چون اعضا و عددش بر يك وجه چهارست و بر يك وجه دو. و معنى اين سخن كه گفتم آنست كه تركيب اعضا از اخلاطست و تركيب اخلاط از مزاج و تركيب مزاج از استقصات، و استقصات بزرگترين مادتى است مر تن مردم را و اخلاط نزديك‌ترين مادتى است. اما آنچه از جنس صورت است بر سه قسمست: قواي‌\footnote{جمعِ قوّه} است و افعال و ارواح، و قوا بر سه قسمست: نفسانى و حيوانى و طبيعى. و نفسانى قوت ‌است و حس و اين بر پنج قسمست: بصر و سمع و شم و ذوق و لمس، و قوت و حركت و عدد و اقسام وى بر حسب عدد و اقسام اعضائى است كه آن را حركت است و قوت سياست است و اين بر سه قسم است: تخيل و فكر و ذكر. و حيوانى بر دو قسمست: فاعل و منفعل، و طبيعى بر سه قسمست: مولده و مربيّه و غاذيه. و افعال بر عدد قواست: نفسانى و حيوانى و طبيعى، از بهر آن را كه قوت مبدأ فعل است و فعل تأثير قوت است و چون برين جمله باشد عدد افعال راست بر عدد قوى\footnote{عربیِ قوا} باشد.

و ارواح بر سه قسمست: نفسانى و حيوانى و طبيعى، از بهر آن را كه روح خادم قوت است، چون برين جمله باشد عدد وى راست بر عدد قوى باشد. و آنچه توابع‌ست مر چيزها را كه قوام و ثبات بدوست چون فربهى كه تابع سردى مزاج‌ست و چون لاغرى كه تابع گرمى مزاج‌ست و چون سرخى گونه كه تابع خون‌ست و چون زردى كه تابع صفراست و چون حركت نبض كه تابع قوّت فاعله‌ی حيواني‌ست و چون شجاعت كه تابع اعتدال قوت حيواني‌ست و چون خشم كه تابع قوت منفعله‌ی حيواني‌ست و چون عفت كه تابع اعتدال قوت شهوانيست و چون حكمت كه تابع اعتدال نفس ناطقه است و جمله چون عرض‌ها و كيفيات كه تابع مادت باشد يا تابع صورت. و آنكه تن را از حال بگرداند، آن را اسباب ضرورتى خوانند و اين شش قسمست: يكى هوا و دُوُم طعام و سِوُم حركت و سكون و چهارم بيدارى و خواب و پنجم گشادگىِِ طبيعت و بستگى، ششم احداث\footnote{ساخت، پیدایش} نفسانى چون اندوه و خشم و بيم و ماننده‌ی اين. و اين‌ها را ضرورتى بهر آن خوانند، كه مردم را چاره نيست از هر يك و هر يك را ازين جمله اندر تن مردم تأثيرست و هر كدام تمام‌تر. چون هر يك ازين جمله بر صورت و بر وجهِ اعتدال بود، استعمال مردم برين جمله برابر صواب و بر وجه اعتدال باشد و چون بعضى را ازين جمله از حال اعتدال تغير اوفتد يا استعمال مردم بعضى را ازين جمله بر وجهِ خطا باشد و بيمارى و علت پديدار آيد بر موجب افراط رفته باشد. و آنچه خارج طبيعت است بر سه قسم است: سبب و مرض و عرض و سبب بر سه قسم است: يا به سبب بيمارى اعضاهاى متشابه باشد يا به سبب بيمارى اعضاهاى رئيسه و تفرق‌الاتصال\footnote{جدا شدن پیوند}، اما به سبب بيمارى اعضاهاى متشابه يا به سبب بيمارى گرم باشد و آن پنج قسمست يا سبب بيمارى سرد و آن بر هشت قسمت است يا سبب بيمارى تر يا سبب بيمارى خشك. و هر يك ازين‌ها بر چهار قسمت‌ست به سبب بيماري‌های اعضاهاى آلى يا سبب بيماريى باشد كه در خلقت افتد، يا اندر مقدار يا اندر وضع يا اندر عدد. و سبب بيماري‌هاى خلقت يا سبب بيمارى شكل بود يا سبب بيمارى تقعير\footnote{گود کردن} و تجويف\footnote{میان تهی کردن} و اين بر هفت قسمت: يا به سبب خشونت باشد و اين بدو قسمت باشد يا سبب ملاست باشد و اين بر دو قسمت باشد.

و سبب بيماريهاى وضع و بيماريهاى عدد هر يك دو نوع است. و مرض بر سه قسمت است: بيماري‌هاى اعضاى متشابه و بيماري‌هاى اعضاى آلى و تفرق‌الاتصال كه آن را مرض مشترك گويند اندر اعضاهاى متشابه اوفتد و هم اندر اعضاى آلى. و بيماري‌هاى اعضاى آلى و بيماري‌هاى اعضاى متشابه بر هشت قسم‌ست چهار فرد: گرم و سرد و تر و خشك و چهار مركب: گرم و تر، و گرم و خشك، سرد و تر، و سرد و خشك. و بيماري‌هاى اعضاى آلى چهار نوع‌ست: بيماري‌هايى كه اندر خلقت اوفتد و اندر مقدار و اندر وضع و اندر عدد؛ و بيماري‌هاى خلقت چهار قسمست آنكه اندر شكل افتد و آنكه اندر تقعير و آنكه بر طريق خشونت اوفتد و آنكه بر طريق ملاست. و بيماري‌هاى مقدار بر دو گونه بود: آنكه از طريق زيادت اوفتد و آنكه از طريق نقصان. و بيماري‌هاى وضع هم بر دو گونه است: يا عضو از جايگاه خويش زايل شود يا پيوند با ديگر اعضا به فساد آرد. بيماريهاى عدد هم بر دو گونه است: يا بر طريق زيادت بود يا بر طريق نقصان. و تفرق‌الاتصال يا اندر اعضاى متشابه افتد يا اندر اعضاى آلى يا اندر هر دو. عرض بر سه قسمست يا بود كه تعلق به افعال دارد و آن بر سه قسمت است و آنكه تعلق به احوال دارد بر چهار قسمت است و آنچه تعلق به استفراغات دارد بر سه قسمت است.

و بايد كه بدانى كه طب بر دو قسمت است: علم است و عمل و قسم علمى آن‌ست كه تو را باز نمودم، بگويم ازين هر علمى كه تو را گفتم از كجا طلب بايد كرد، تا هر يك را به شرح و استقصا بدانى كه از كدام جاى بايد طلبيدن كه اين علم‌ها كه ياد كردم جالينوس\footnote{طبیب معروف یونانی} به شرح و استقصا ياد كرده است در سته عشر\footnote{سته‌عشر یعنی شانزده، منظور مجموعه‌ای است شامل شانزده کتاب از جالینوس، اطبای قدیم اساس طب را بر این شانزده کتاب گذاشته اند} و برخى بيرون از سته عشر. اما علم استقصات آن مقدار كه طبيب را به كار آيد از كتاب استقصات\footnote{یا کتاب‌الاصول، از اقلیدس مهندس یونانی؛ ترجمه حنین بن اسحق طبیب از یونانی به عربی و اصلاح ثابت بن قره بن مروان حرانی}، طلب كن از جمله‌ی سته عشر. و علم مزاج از كتاب‌المزاج طلب كن از جمله‌ی سته عشر. و علم اخلاط از مقالت دوم طلب كن، از كتاب القوى‌الطبيعه از جمله‌ی سته عشر.

و علم اعضاى متشابهه از تشريح كوچك طلب كن هم از سته عشر. و علم اعضاى آلى از تشريح بزرگ طلب كن كه بيرون سته عشر است. و علم قوى طبيعى از كتاب قوى‌الطبيعه طلب كن از جمله‌ی سته عشر. و قوى حيوانى از كتاب النبض طلب كن هم از جمله‌ی سته عشر. و قوى نفسانى از راى بقراط و افلاطون طلب كن و اين كتابي‌ست هم تصنيف جالينوس بيرون سته عشر. و اگر خواهى كه اندرين متبحر شوى و از پايگاه طلب بگذرى علم استقصات و علم مزاج از كتاب الكون و الفساد\footnote{رساله‌ای فلسفی از ارسطو است، ارسطو در این کتاب به بررسی هست شدن اشیاء پیشتر ناموجود (کون) و نیست شدن اشیاء پیشتر موجود (فساد) می‌پردازد} و از كتاب السما و العالم\footnote{جلد چهاردهم بحارالانوار، تالیف محمد باقر مجلسی} طلب كن و علم قوى و افعال از كتاب النفس\footnote{کتابی از ارسطو، از مهترین آثار فلسفی یونان است اسحق بن حنین آن را به عربی ترجمه کرده و افضل الدین کاشانی تلخیص یکی از شروع آنرا به فارسی ترجمه نموده است.  در اروپا تا قرون جدید بزرگترین مرجع مطالعه در علم‌النفس به شمار می رفته و در عالم اسلام نیز هر گونه تحقیقی درباره نفس، لااقل از لحاظ رئوس مطالب، بر آن مبتنی بوده است.} و كتاب الحس و المحسوس\footnote{کتاب ارسطو} و علم اعضا از كتاب الحيوان و اقسام الامراض از مقالت نخستين از كتاب علل و الاعراض\footnote{اثر جالینوس و جزیی از مجموعه شانزده کتاب} طلب كن از جمله‌ی سته عشر. و اسباب اعراض از مقالت دوم هم ازين كتاب طلب كن و اسباب امراض از مقالت چهارم و پنجم و ششم طلب كن ازين كتاب كه گفتم.

و چون قسم علمى ياد كردم ناچار شمتى\footnote{پاره ای و جزیی و بخشی از چیزی} از قسم عملى ياد بايد كردن اگر چه سخن دراز همى‌شود ازيرا كه علم و عمل چون جسم و روح‌اند هر دو به هم و جسم بى‌ روح و روح بى‌ جسم تمام نبود. و چون معالجت خواهى كردن، انديشه كن از خورش‌هاى پيران و كودكان طفل و بيمار كه معالجت بيماران بر دو گونه باشد و معالج بايد كه به هيچ حال ابتدا نكند، معالجتى تا نخست آگاه نگردد از وقت بيمارى و از وقت علت و سبب علت و مزاج سال و صنعت بيمار و تجسس طبيعتش و طبع و جايگاه و حال و مزاج و آب و جنس و عرض ظاهر و علامت‌هاى بيماران و علامت‌هاى نيك و علامت‌هاى بد و انواع رسوب و علامت‌هاى بيماري‌ها كه در باطن بود و نشان‌هاى بحران شناخته باشد و اجناس حميات\footnote{جوانمردی، مردانگی} معلوم گردانيده باشد، كه تدبير امراض بر چه سان باشد. و بر تركيب ادويه ماهر باشد بر مذهب اصحاب‌القياس و قوانين معالجات بدانسته باشد.

و اگر اين هر يكى را شرحى كنم قصه دراز كنم، اما بگويم كه علم هر يك از كدام كتاب بايد طلبيدن تا به وقت حاجت تو را معلوم باشد. اما حفظ صحت از تدبير‌الاصحا بايد طلبيدن، از جمله‌ی سته عشر و معالجت بيماران و قوانين علاج از كتاب حيلة‌البرء طلب بايد كرد از جملۀ سته عشر و علامت‌هاى نيك و بد از تقدمة المعرفة طلب بايد كردن و از فصول بقراط و علم نبض از نبض الكبير و نبض الصغير و علم بول از مقاله‌ی نخستين طلب بايد كردن از كتاب‌البحران از جمله‌ی سته عشر و از كتاب البول جالينوس كه بيرون سته عشر است. و نشان بيماري‌ها كه اندر باطنِ تن باشد، از اعضاى آلى طلب بايد كردن هم از سته عشر و علم ايام‌البحران هم از كتاب ايام‌البحران از سته عشر و علم حميات از كتاب الحميات از سته عشر طلب بايد كرد و تدبير امراض حاده از كتاب ماء‌الشعير طلب بايد كردن از جمله‌ی تصانيف بقراط از جمله اعضاى آلمه و حيلة البرء و تركيب ادويه از ادويه‌ی جالينوس.

و معالج بايد كه تجربت بسيار كند و تجربت بر مردم معروف و مشهور نكند و بايد كه خدمت بيمارستان‌ها كرده باشد و بيماران بسيار ديده و معالجت بسيار كرده باشد تا علت‌هاى غريب بر وى مشكل نشود و اعلال اعضا و احشا بر وى نپوشد و آنچه اندر كتب خوانده باشد به رأى‌العين همى‌بيند و به معالجت اندر نماند. و بايد كه وصاياى بقراط خوانده باشد، تا اندر معالجت بيماران شرط امانت و راستى به جاى تواند آورد و پيوسته خويشتن پاك و جامه پاك و مطيّب دارد. و چون بر سر بيمار شود با بيمار تازه‌روى و خوش سخن باشد و بيمار را دل‌گرمى دهد كه تقويت طبيب بيمار را قوت حرارت غريزى بيفزايد.
 
فصل، اگر پندارى كه به خواب اندرست، چون بخوانى پاسخ همى‌دهد ولكن تو را نشناسد، چشم همى بازگشايد و باز همى بغنود\footnote{آرمیدن و آسودن. استراحت و آسایش} علامت بد بود. و نيز اگر مدهوش بينى و دست هر جاى همى نهد و خود را و جاى خود را همى شوراند علامت بد بود. و نيز اگر مدهوش بود و هر وقتى بانگى بزند و دست و انگشتان خويش همى گيرد و همى فشارد هم علامت بد بود. و اگر سپيدى چشم بيمار سپيدتر از عادت خود بود و سياهى سياه‌تر و زبان نگردد و دهان همى‌گرداند و دم از پس همى كشد هم علامت بد بود. و اگر بيمار قى\footnote{استفراغ کردن} پيوسته همى‌كند لون لون، سرخ و زرد و سياه و سپيد، تا قى باز نیايستد هم مخوف بود. و اگر بيمار را كاهش و سرفه بود خدوى\footnote{آب دهان} او بر گوى بگير و خشك كن و آنگه رگوى را بشوى اگر نشان بماند هم علامت بد بود. و اگر از رشك صعب يا از غم صعب بيمار شود يا دمه دارد اين همه را كه گفتم هيچ‌كس را دارو مكن تا اين علامت با ايشان بود كه معالجت سود ندارد. پس اگر بر سر بيمار شود و ازين علامت‌ها كه گفتم هيچ نبينى جاى اوميد بود.

آنگاه دست بر مجسه‌ی\footnote{جای انگشت نهادن طبیب از دست بیمار} بيمار نِهْ، اگر بِجَهَد و زير انگشت برود بدان كه خون غالب است و اگر زير انگشت باريك و تيز جهد صفرا غالب باشد و اگر زير انگشت دير و سطبر و سست جهد رطوبت غالب باشد، پس اگر مخالف جهد بر ان جانب كه ميلش بيشتر بينى حكمش بر آن جانب كن.

فصل، پس چون مجسه معلوم كردى، اندر قاروره\footnote{قرابه، پیاله} نگاه كن، اگر آبى سپيد بينى و نه روشن مرد از غمى بيمار بود و اگر اين سپيد و روشن بود علت از باد دژم نبام بود و رطوبت و ناگوارد و اگر چون آب روشن بود از كراهيتى بود. و اگر پر خون آب و گونه‌ی ترنج بود و اندر وى ذره ذره باشد بيمارى از شكم بود. و اگر آب چون روغن بينى و اگر در بن قاروره خلط بيند علتى قريب عهد بود. و اگر به رنگ زعفران باشد بدان كه او را تب و صفراست و خون نيز با صفرا يار باشد. اگر بر سر آب زردى باشد و بن آب سيه‌بام باشد علت از گش سبز بود دارو مكن. و اگر بر سر آب سياهى بينى هم چنين، و اگر قاروره به زردى زند يا به سبزى رود بِهْ باشد. و اگر بيمار هذيان گويد و آب سرخى سياه بام باشد گش سپاه با خون آميخته بود و لهب وى بر سر رفته بود، از وى محترز باش. و اگر سياه بينى و بر سر وى چون خونى ايستاده، بيش بر سر آن بيمار مرو.

و اگر سياه بود و اندرو مانند سبوس چيزى پيدا بود، يا بر سر وى چون خونى ايستاده بود آن را بدرود كن. و اگر آب زرد بود و اندر وى چيزى نمايد چون آفتاب لامع\footnote{روشن، درخشان} يا زردى سرخ‌فام بود، غلبه‌ی خون بود، فصد فرماى كه زود بِهْ شود و اگر زرد بود و اندر وى خط‌هاى سرخ، به خدايش تسليم كن. و اگر زرد بود و اندر وى خط‌هاى سپيد، بيمارى دير بكشد و اگر سبز رنگ باشد، علت از سپرز\footnote{طحال} بود و اگر سبز و سياه بينى بيشش باز نه و اگر سبز و سپيد باشد و اندر وى چيزى چون كرم سر كه وى را باد و بواسير بود و جماع نتواند كردن.

و چون آب ديدى و مجسه معلوم كردى، آنگه جنس علت جوى كه اجناس علت‌ها نه از يك گونه بود و چون جنس دانستى نوع را بازدان تا معالجت بتوانى كردن. و تا به غذا كفايت شود به دارو و ضماد\footnote{پماد} و طلى\footnote{طلا، زر} مكوش و تا به نقوع و ضماد كفايت بود، به حب و مطبوخ و معجون مكوش. و نگر كه به دوا كردن دليرى نكنى تا با تسكين و به طعيب كار برآيد در استفراغ تجاوز مكن و دست مبر و چون كار از حد بخواهد شد پس به دواى محض مشغول شو و به تسكين كردن مشغول مشو. و هرگز بيمار را متهم مكن و تعهد ناقه از آن بيشتر از آن كن آن رنجور را و مگوى اين بهتر باشد. و بر بيمار شكم بنده پرهيز سخت حكم مكن كه قبول نكند، لكن دفع مضرّت آن چيز كه وى خورده باشد همى‌كن. بهترين چيزى طبيب را دارو شناختن و علت شناختن دان. و اندرين باب سخن بسيار بگفتم از آنچه من اين علم را دوست دارم كه علمى مفيدست، پس اگر بسيار ازين بگفتم كه سخن دوستان مردمان دوست دارند. اگر چنان افتد كه اتفاق اين علمت نيوفتد علم نجوم علمى شريف است جهد كن در آموختن اين كار كه علمى سخت بزرگست از آن سبب را كه معجزه‌ی پيغامبرى مرسل بوده است و آن علم كه پيغامبرى مرسل را معجز بوده باشد آن علم علمى نبوى بود.

\newpage
