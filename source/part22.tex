\addcontentsline{toc}{section}{باب بيست و دوم - 
در امانت نگاه داشتن}
\section*{باب بيست و دوم - 
در امانت نگاه داشتن}

اگر كسى به نزد تو امانتى بنهد، تا بتوانى به هيچ حال مپذير از آنچه امانت پذيرفتن بلا پذيرفتن است؛ زيرا كه عاقبت آن از سه وجه بيرون نباشد: يا امانت به سلامت به وِى باز رسانى، چنانكه خداى تعالى فرموده است: «إِنَّ اللّهَ [تعالى] يَأْمُرُكُمْ أَنْ تُؤَدُّوا الْأَماناتِ إِلی أَهْلِها» و پيغامبر ما صلى اللّه عليه و سلّم گفت: «ردّوا الامانات الى اهلها» كه طريقِ مردى و جوانمردى آنست كه امانت مردمان را نپذيرى، يا چون بپذيرى نگاه دارى تا به سلامت به خداوند باز ر‍سانى.

چنانكه شنودم كه مردى به سحرگاه از خانه بيرون رفت تا به گرمابرود، به راه اندر دوستى از آنِ خويش را ديد. گفت: موافقت كنى تا به گرمابه شويم‌؟ گفت: تا در گرمابه با تو هم‌راهى كنم، لكن اندر گرمابه نتوانم آمدن، كه شغلى دارم. و تا نزديك گرمابه بيامد، بسر دوراهى رسيد بى‌آنكه اين مرد را اخبر داد، بازگشت و به راه ديگر برفت. اتفاق را طرّارى از پس اين مرد مى‌رفت به طرّارى خويش؛ اين مرد باز نگريد، طرار را ديد و هنوز تاريك بود پنداشت كه آن دوست وي ست. صد دينار در آستين داشت بر دستارچه بسته از آستين بيرون گرفت و بدين طرار داد و گفت: اى برادر اين امانت است به تو چون من از گرمابه بيرون آيم به من بازدهى. طرار زر از وى بستد و آنجا مُقام كرد، تا وى از گرمابه بيرون آمد، روز روشن شده بود. جامه بپوشيد و راست همى‌رفت. طرار وى را باز خواند و گفت: اى جوانمرد زر خويش بازستان و پس برو كه من امروز از شغل خويش فروماندم، ازين نگاه داشتن امانت تو. مرد گفت: اين زَر چيست و تو چه مردى‌؟ گفت: من مردى طرّارم، [تو] اين زر به من دادى. گفت: اگر تو طرارى چرا زر من نبردى‌؟ طرار گفت: اگر به صناعت خويش بردمى، اگر هزار دينار بودى از تو يك جو نه‌انديشيدمى و نه بازدادمى و لكن تو به زنهار به من دادى، زينهار[دار] نبايد كه زينهارخوار باشد كه امانت بردن جوانمردى نيست.

پس اگر بر دست تو مستهلك شود بى‌مراد تو، يا خود چيزى نيك بود، ديو ترا از راه ببرد و طمع [در آن كنى و منكر شوى. اگر چنانكه به خداوند حق باز‌رسانى بسى رنج‌ها] به تو رسد، در نگاه‌داشتن آن چيز، چون رنج‌هاى بسيار بكشى و آن چيزش به خداوند باز دهى رنجى خيره به تو بماند و آن مرد به هيچ روى از تو منّت ندارد. گويد: چيز من بود آنجا نهادم و باز بياوردم و راست گويد. پس رنج كشيدن بى‌منت بر تو بماند و مزدى تو آن كرد كه جامه بيالايد. و اگر مستهلك شود هيچ‌كس باور نكند و تو بى خيانتى نزديك مردمان خاين گردى و اندر خصو[مت] اوفتى و باشد كه خود غرامت آن ببايد كشيد. و اگر منكر شوى با تو نماند يا به خوشى يا بستم از تو باز ستانند، خاين گردى و حشمت تو ميان اشكالان\footnote{همانندان} تو بشود، بيش كس بر تو اعتماد نكند و اگر به تو بماند مظالمى در گردن تو بماند، بدين جهان در برخوردار نباشى و بدان جهان عقوبت خداى عزّوجل حاصل كرده باشى.

فصل امّا اگر به كسى وديعتى نهى پنهان منه، كه نه كسى چيز تو از وى بخواهد  ستد. بى دو گواىِ عدل چيز خويش به نزد هيچ كس وديعت منه و بدآنچه دهى حجّتى از وى بستان تا از داورى رسته باشى. پس اگر داورى افتد، در داورى دلير مباش، كه دليرى به داورى اندر نشان ستمكارى بود. و تا بتوانى هرگز سوگند به دروغ و راست مخور و خويشتن را به سوگند خوردن معروف مكن تا اگر وقتى سوگند بايدت خوردن، چنانكه افتد، مردمان تو را بدان سوگند، راست‌گوى دارند. و هر چند توانگر باشى چون تن‌آسان و نيك‌نام و راست‌گوى نباشى خويشتن از جمله‌ی درويشان دان كه بدنامان و دروغ‌زنان عاقبت ايشان جز درويشى نبود. و امانت را كاربند كه امانت را كيمياى زر گفته‌اند و هميشه توانگر زى، يعنى كه امين باش و راست گوى كه مال همه عالم، امينان و راست‌گويان راست. و بكوش كه فريبنده نباشى و حذر كن كه فريفته نشوى خاصه در ستد و دادى كه درِ شهوت بسته باشد.

\newpage


















