\addcontentsline{toc}{section}{باب سى و نهم
در آداب و آيين دبيرى و شرط كاتب}
\section*{باب سى و نهم
در آداب و آيين دبيرى و شرط كاتب}

اگر دبير باشى و خط نيكو دارى [بايد كه] بر سخن قادر باشى و تجاوز كردن در خط بعادت كنى و بسيار نبشتن نيز عادت كنى تا ماهرتر باشى بر نبشتن.

حكايت از آنچه شنيدم كه صاحب اسماعيل عباد روز شنبدى در ديوان چيزى مى‌نبشت. روى سوى كاتبان كرد و گفت: هر روز شنبه در كاتبى خويش نقصانى همى‌بينم از آنچه روز آدينه به ديوان نيامده باشم و چيزى نَنِبِشته باشم آن يك روز تقصير در من تأثير كند. پس پيوسته به چيزى نبشتن مشغول باش به خطى گشاده و مبين و سر بر بالا و سخن درهم بافته و در نامه بايد كه بسيار غرض و معانى در اندك مايه سخن به كار برى چنانكه شاعر گويد:

\begin{quote}
نكته‌اى بود از دهان دهر بيرون آمده\quad \quad
نامه‌اى بود از معانى در حديث مختصر
\end{quote}

و نامۀ خود را به استعارت و آيات قرآن و اخبار رسول عليه السلام آراسته دار و اگر نامه پارسى بود پارسى مطلق منبيس كه ناخوش بود، خاصه پارسى درى كه نه معروف بود آن خود نبايد نبشت به هيچ حال كه خود ناگفته بهتر از گفته بود. و تكلف‌هاى نامۀ تازى خود معلوم است كه چون بايد كرد و اندر نامه‌ی تازى سجع هنرست و خوش آيد لكن اندر نامه‌ی پارسى سجع ناخوش آيد اگر نگويى بهتر باشد. اما هر سخنى كه گويى عالى و مستعار گوى و مختصر بايد گفت و كاتب بايد كه اسرار كاتبى نيك داند و سخن‌هاى مرموز را نيك اندر يابد.

حكايت چنانكه شنيدم اى پسر كه جدّ تو، سلطان محمود، رحمه اللّه بخليفه‌ی بغداد، القادر باللّه، نامه فرستاد و گفت كه: بايد كه ماورا‌ء النهر به من بخشى و مرا منشور دهى تا بروم و به شمشير ولايت بستانم و آن القادر باللّه گفت: اندر همه اسلام مرا ازيشان مطيع‌تر كس نيست معاذ اللّه كه من اين كنم و اگر تو بى‌فرمان من اين كنى همه عالم را بر تو بشورانم. سلطان محمود از آن سخت طيره شد، رسول خليفه را گفت: چه گويى من از بو مسلم كمترم‌؟ مرا خود اين شغل با توست، اينك آمدم با دو هزار پيل و دار الخلافه بپاى پيلان ويران كنم و خاك وى بر پشت پيلان بغزنين آرم و تهديد عظيم بنمود از بارنامه‌ی پيلان خويش. رسول برفت و بعد از آن به چند گاه باز آمد. سلطان باز بنشست و حاجبان و غلامان صف زدند و پيلان مست بر در سراى بداشتند و لشكر تعبيه كردند و رسول را بار دادند. رسول بيامد و نامه‌اى قريب يك دسته كاغذ منصورى پيوسته و درهم پيچيده و مهر كرده پيش سلطان محمود بنهاد، گفت: امير المؤمنين همى گويد كه نامه‌ی تو خواندم و تحميل تو شنودم و جواب نامه‌ی تو و جواب تحميل تو اينست كه اندرين نامه نبشته است. خواجه بو نصر مشكان كه عميد ديوان رسايل بود دست دراز كرد و نامه برداشت و بگشاد تا بخواند اول نامه نبشته بود:

«بسم اللّه الرحمن الرحيم»، آنگه سطرى چنين: الم، الف و لام و ميم، و آخر نامه: «الحمد للّه و الصلوة على رسوله محمد و آله اجمعين.» و هيچ ديگر نبشته نبود. محمود با همه كاتبان خويش اندر انديشه افتاد كه اين چه سخن مرموزست‌؟ و هر آيتى كه در قرآن الف و لام و ميم بود همه بخواندند و تفسير بكردند هيچ جواب محمود نبود. آخر الامر خواجه ابو بكر قهستانى منشور بر عام عرضه كنم تا بفرمان و منشور رعيت بر من مطيع  باشند. جوان بود و هنوز درجه‌ی نشستن نداشت، اندر ميان نديمان كه بر پاى بودند ايستاده بود، وى گفت:

اى خداوند، امير المؤمنين نه الف و لام و ميم نبشته است، خداوند وى را تهديد بپيلان كرده بود و گفته كه خاك دار الخلافه بپشت پيلان به غزنين آرم، جواب خداوند نبشته است: «أَ لَمْ تَرَ كَيْفَ فَعَلَ رَبُّكَ بِأَصْحٰابِ الْفِيلِ ؟» جواب پيلان خداوند همى‌دهد. شنيدم كه سلطان محمود را تغيير افتاد تا چند گاه به هوش باز نيامد و بسيار بگريست و زارى كرد چنانكه ديانت او بود عذرها خواست از خليفه و آن سخنى درازست. بو بكر قهستانى را خلعت فرمود و ساخت زر، و فرمود تا ميان نديمان بنشيند و بدين يك سخن درجه‌ی بزرگ يافت.

حكايت و نيز همچنين شنيدم كه بروزگار سامانيان بو على سيمجور كه بنيشابور بود گفتى كه من اسفه‌سلاّر و امير خوراسانم. و لكن به درگاه نرفتى و آخر عهد سامانيان بود و چندان قوت نداشتند كه بو على را بدست آوردندى به عنف، پس با او به اضطرار به خطبه و سكه و هديه راضى شدندى. و عبد الجبار خوجانى خطيب خوجان بود، مردى بود فقيه و اديبى نيك بود و كاتبى تمام با راى سديد، به همه كارى كافى. بو على سيمجور او را از خوجان بياورد و كاتب الحضرتى خويش بدو داد و تمكينى تمامش بداد اندر شغل، و هيچ شغل بى‌مشورت او نكردى از انكه مردى سخت با كفايت بود. و احمد بن رافع اليعقوبى كاتب حضرت امير خوراسان بود، مردى سخت فاضل و محتشم و شغل همه ماور‌‌اء النهر در زير قلم او بود و احمد بن رافع را با عبد الجبار خوجانى دوستى بود سخت بى‌ممالحتى و ملاقاتى كه در ميان ايشان بوده بود امّا به مناسبت فضل، دوستى داشتندى با يكديگر به مكاتبت.

روزى امير خوراسان را وزير او گفت كه: اگر عبد الجبار خوجانى كاتب بو على سيمجور نبودى بو على را به دست شايستى آوردن كه اين همه عصيان بو على از كفايت عبدالجبارست، نامه‌اى بايد نبشتن به بو على و گفتن كه: اگر تو به طاعتى و چاكر منى چنان بايد كه چون اين نامه به تو رسد در وقت سر عبد الجبار خوجانى ببرى و اندر توبره‌اى نهى و بدست اين قاصد به درگاه من فرستى تا من دانم كه تو به طاعتى كه هر چه تو مى‌كنى معلوم ماست كه به مشاورت و تدبير او همى‌كنى و اگر نكنى اينك من كه امير خوراسانم به تن خويش مى‌آيم جنگ را ساخته باش. چون اين تدبير بكردند گفتند به همه حال اين خط به خط احمد بن رافع بايد كه بود، و احمد دوست عبدالجبارست، ناچار كس فرستد و اين حال باز نمايد و عبد الجبار بگريزد. امير خوراسان احمد رافع را بخواند و گفت: نامه‌اى به بو على سيمجور بنويس درين باب و چون نامه نبشتى نخواهم كه تو سه شبان روز ازين سراى بيرون آيى و هيچ كهترى از آن تو نبايد كه پيش تو آيد كه عبد الجبار دوست تست اگر بدست نيايد دانم كه تو نموده باشى. احمد هيچ چيز نتوانست كردن، و چنين گويند كه احمد نامه همى‌نبشت و همى‌گريست و با خود مى‌گفت كه:

كاشكى من دبيرى ندانستمى تا دوستى بدين فاضلى بخط من كشته نشدى و اين كار را هيچ تدبير نمى‌دانم. آخر اين آيت كه خداى تعالى در محكم تنزيل خويش همى‌گويد يادش همى‌آمد: «أَنْ يُقَتَّلُوا أَوْ يُصَلَّبُوا أَوْ تُقَطَّعَ أَيْدِيهِمْ وَ أَرْجُلُهُمْ مِنْ خِلافٍ أَوْ يُنْفَوْا مِنَ الْأَرْضِ »، با خويشتن انديشيد كه هر چند او اين رمز نداند و هيچ بر سر اين سخن نيفتد من آنچه شرط دوستى بود به جاى آرم. چون نامه بنبشت و عنوان بر كرد، برين كناره‌ی نامه الفى بكرد به قلمى باريك و به جانب ديگر نونى بكرد يعنى: أن يقتّلوا، و نامه بامير خوراسان عرضه كرد، كس خود در عنوان ننگريست و بدان نگاه نكرد. نامه برخواندند و به مهر كردند و به جمّازه‌بانى دادند و جمّازه‌بان را ازين حال آگاه نكردند و گفتند: اين نامه به بو على سيمجور ده و آنچه وى بتو دهد بستان و باز آور. و احمد رافع را سه روز نگاه داشتند و بعد از سه روز باز خانۀ خويش آمد دل تنگ. چون جمّازه‌بان بنشابور رسيد و پيش بو على سيمجور رفت و نامه بداد، چنانكه رسم باشد، بو على نامه ببوسيد و از حال سلامت امير خوراسان بپرسيد و خطيب عبد الجبار نشسته بود، نامه بوى داد و گفت: مهر بردار و فرمان عرضه كن. عبد الجبار نامه بستد و در عنوان نگه كرد، پيش از انكه مهر برداشت بر كناره الفى ديد و بر كنارۀ ديگر نونى، در وقت اين آيت يادش آمد: «أَنْ يُقَتَّلُوا »، بدانست كه نامه در باب كشتن وى است، نامه از دست بنهاد همچنان بمهر، و دست بر بينى نهاد يعنى كه از بينى من خون همى‌آيد بشويم و باز آيم. همچنان از پيش بو على برفت و دست بر بينى نهاده، و راست كه از در بيرون رفت بجايى متوارى شد. زمانى منتظر او بودند، بو على گفت: بخوانيد. خواجه را طلب كردند و نيافتند. گفتند: بر اسب خود ننشست، پياده از سراى بيرون آمد و بخانۀ خويش نشد و كس ندانست كه كجا رفت. بو على سيمجور گفت: ديگر دبيرى را بخوانيد تا بخواند. و نامه بگشادند و برخواندند پيش جمّازه‌بان، چون حال معلوم شد همه خلق بعجب ماندند كه با وى كه گفت كه اندرين نامه چيست نبشته‌؟ بو على سيمجور اگر چند بدان شادمانه بود پيش جمازه‌بان لختى ضجرت نمود و به شهر منادى كردند و عبد الجبار خود اندر نهان كس فرستاد كه من فلان جاى نشسته‌ام، بو على بدان شاد شد و خداى را تعالى شكر كرد و فرمود كه: همانجا همى‌باش. چون روزى چند برآمد جمّازه‌بان را صلتى نيكو بداد و جواب نامه بنوشت كه حال بر چه جمله بود و سوگندان ياد كرد كه ما ازين هيچ خبر نداشتيم. امير خوراسان از ان حال عاجز شد و متحير بماند و خطى و مهرى و زنهار نامه‌اى فرستاد كه ما وى را عفو كرديم بدان شرط كه بگويد كه اندرين نامه چه دانست كه چيست‌؟ احمد رافع گفت: مرا زنهار ده تا من بگويم. امير خوراسان وى را زنهار داد، وى آن حال شرح داد.

امير خوراسان عبد الجبار را عفو كرد و آن نامه‌ی خويش بازخواست تا آن رمز ببيند. نامه باز آوردند، رمز همچنان بود كه احمد گفت. خلق شگفت بماندند از فضل او وز آن آن مرد ديگر.

و شرط كاتبى آنست كه مادام مجاور حضرت باشى و سابق كار ياد دارى و تيز - فهم و نافراموش كار باشى و متفحّص باشى و از همه كارها تذكره همى دارى از انچه تو را فرمايند و از انچه تو را نفرمايند. بر حال همه اهل ديوان واقف باشى و از معاملات همه اعما‌ل‌ها آگاه باش و تجسس كن و از همه گونه تعريف اعمالها همى‌كن اگر چه در وقت به كارت نيايد، باشد كه وقتى ديگرت به كار آيد و لكن آن سرّ با كس مگوى مگر وقتى كه ناگزير بود و به ظاهر كردن تفحص شغل وزير منماى و لكن در باطن از همه چيز آگاه باش. و بر حساب قادر باش و يك ساعت از تعرف كدخدايى و نامهاى معاملان نبشتن خالى مباش كه اين همه در كاتبان هنرست.

و بزرگترين هنرى كاتب را زبان نگاه داشتن است و سرّ ولى نعمت پيدا ناكردن و خداوند خويش را از هر شغلى آگاه كردن و فضولى نابودن. اما اگر چنانكه خطاطى و قادر باشى و از هر گونه خطى كه بينى بدانى نبشت، اين دانش سخت نيكست و لكن بر كسى پيدا مكن تا به مزوّر كردن معروف نشوى كه آنگاه اعتماد ولى نعمت تو از تو برخيزد، و اگر كسى ديگر مزوّرى كند چون ندانند كه كردست بر تو بندند. و به هر محقرى مزوّرى مكن تا اگر وقتى به كار آيد چون منافعى بزرگ خواهد بود، آنگه اگر بكنى كس بر تو گمان نبرد، كه بسيار كاتبان فاضل و محتشم را وزيران عالم هلاك كردند بسبب خطهاى مزور ايشان.

حكايت شنيدم كه ربيع بن المطهر القصرى كاتبى محتشم بود اندر ديوان صاحب، وى خطّ مزوّر كردى. اين خبر بگوش صاحب رحمه اللّه رسيد، صاحب فروماند، نه مرد را هلاك توانست كردن از سبب فضل آن مرد كه سخت فاضل بود و يگانه و نه بر وى همى‌توانست پيدا كردن؛ همى انديشيد كه با وى چكند؟ اتفاق را اندرين ميانه صاحب را عارضه‌اى پديد آمد، مردمان بعيادت همى‌رفتند تا اين ربيع بن المطهر - القصوى اندر آمد و پيش صاحب بنشست چنانكه رسمست. صاحب را بپرسيد  كه چه نالا نيست‌؟ وى بگفت كه علت چيست. آنگه پرسيد كه شراب چه مى‌خورى‌؟ گفت: فلان شراب. پس بپرسيد كه طعام چه مى‌خورى‌؟ صاحب گفت:

از آنچه تو مى‌كنى يعنى مزوّر. كاتب بدانست كه صاحب از ان كار آگاه شدست، گفت: اى خداوند بسر تو كه ديگر نكنم. صاحب گفت: اگر ديگر نكنى بدانچه كردى عفوت كنم و عقوبت نكنم.

پس اين مزور كردن كارى بزرگست، از ان بپرهيز و اندر هر پيشه و اندر هر شغلى تمام داد سخن نمى‌توانم داد كه سخن دراز گردد و از مقصود باز مانم و ناگفته نيز يله نمى‌توانم كرد. چون از هر نوعى طرفى گفتم اگر بگوش دل بشنوى تو را از آنجا استخراج‌ها اوفتد كه از چراغى فراوان چراغ‌ها بتوان افروختن.

و اگر چنانكه خداى تعالى بر تو رحمت كند و از درجه‌ی كاتبى به وزارت رسى شرط وزارت بدان كه چگونه است و وى را. 






















































