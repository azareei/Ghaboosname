\addcontentsline{toc}{section}{باب شانزدهم
اندر گرمابه رفتن}
\section*{باب شانزدهم
اندر گرمابه رفتن}

چون به گرمابه رفتن حاجت اوفتد، بر سيرى مرو، كه زيان دارد. و نيز در گرمابه به جماع مشغول مباش، البته خاصه در گرمابه‌ی گرم، كه محمّد زكرياى رازى گويد: عجب دارم كه كسى سير بگرمابه‌ی گرم اندر جماع كند و اندر وقت فجا\footnote{ناگهان} بِنَميرَدْ.

اما گرمابه سخت جليل است، شايدگفتن\footnote{سزاوار است گفتن} كه تا حكيمان بناها ساختند از گرمابه بهتر هيچ بنا نساختند. و لكن با همه نيكى هر روز به گرمابه رفتن سود ندارد، بل كه زيان دارد كه عصب‌ها و مفاصل نرم گرداند و سختى وى ببرد و طبيعت عادت كند هر روز به گرمابه رفتن، تو چون يك روز نه روى آن روز  تنِ تو چون بيمارى بود و اندام‌ها درشت شود. چنان بايد كه هر دو روز يك روز شود، تا هم تن را سود دارد و هم به رعنائى منسوب نكنند. و چون به گرمابه روى، اوّل بخانه‌ی سرد شو و يك زمان توقف كن، چندانكه طبع از وى حظّى يابد. آنگه در خانه‌ی ميانگى رو و آنجا يك زمان بنشين تا از [ان] خانه نيز بهره يابى. آنگه در خانه‌ی گرم شو و ساعتى همى باش تا حظّ خانه‌ی گرم نيز بيابى. چون گرما در تو اثر كرد در خانۀ خلوت رو [و آنجا سر بشوى]. و بايد كه در گرمابه بسيار مقام نكنى و آب سخت‌ْگرم و سخت‌ْسرد بر خود نريزى، معتدل بايد كه بود. و اگر گرمابه خالى بود، غنيمتى بزرگ‌دان، كه حكما گرمابه‌ی خالى را غنيمتى دارند [از جمله] غنيمت‌هاى بزرگ. و چون از گرمابه بيرون آيى، موى سخت خشك بايد كردن و آنگه بيرون رفتن، كه با موى تر به راه رفتن نه كار محتشمان باشد. و نيز از گرمابه بيامده بامداد با موىِ تَر، پيش خداوندان نشايد رفتن كه بى‌ادبى باشد. و همچنين با موىِ تَرْ به سلام مردمانِ محتشمْ شَرطْ نباشدْ رفتن. و نفع و ضرّ گرمابه اينست كه گفتم. امّا در گرمابه از آب خوردن و فُقاع\footnote{آب‌جو} خوردن، پرهيز كن كه سخت زيان دارد و استسقا\footnote{تشنگی} آورد مگر سخت مَحرور\footnote{گرما} بُوَد، آنگه روا باشد كه اندكى بخورد تشنگى و شكستن خمار را.






















\newpage
