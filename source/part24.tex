\addcontentsline{toc}{section}{باب بیست‌و‌چهارم - 
در خریدن ضیاع و عقار}
\section*{باب بیست‌و‌چهارم
در خریدن ضیاع و عقار}

و اما اى پسر! بدان و آگاه باش كه اگر ضيعت و خانه خواهى خريدن و هر چه خواهى خريد و فروخت، حدّ شرا و بيع نگه دار. هر چه خرى در وقت كسادى خر و آنچه فروشى در وقت روايى فروش. و از سود طلب‌كردن عيب مدار كه گفته‌اند:
\begin{quad}
ببايد چميد ار بخواهى چريد 
\end{quad}
و از مکاس\foonote{چانه زدن} كردن، غافل مباش كه مكاس و تعبير نيمى از تجارتست. اما آنچه خرى باندازه‌ی سود [و] زيان بايد كرد، و اگر خواهى كه مفلس نگردى از سود ناكرده خرج مكن. و اگر خواهى كه بر مايه زيان نكنى از سودى كه عاقبت آن سود زيان باشد بپرهيز. و اگر خواهى كه با خواسته‌ی بسيار درويش نباشى، حسود و آزمند مباش، و در همه كارها صبور باش كه صبورى دَوْم عاقليست. اندر صلاح و كار خويش به هيچ گونه غافل مباش، كه غافلى دوم احمقى است. و در كارها تهور مكن كه تهور دوم جاهليست. و چون كار بر تو پوشيده شود و شغل بر تو بسته شود، زود با سر رشته شو و صبور باش تا روىِ كار پديد آيد كه هيچ كار به شتاب‌زدگى نيكو نشود. و چون بر سر شرى و بيع رسيدى اگر خواهى كه خانه خرى در كويى خر كه مردم مصلح باشند و به كناره‌ی شهر مخر و اندر بن بارو مخر و از بهر ارزانى خانه‌ی ويران مخر. اول به همسرايه\footnote{همسایه} نگه كن، كه كيست كه گفته‌اند: مثل «الجار ثم الدار». بزرجمهر گويد: چهار چيز بلاى بزرگست: همسرايه‌ی بد، و عيال بسيار، و زن ناسازگار، و تنگ‌دستى. و البته به همسرايگى علويان و دانشمندان و خادمان خانه مخر، و جهد كن تا به كويى خرى كه اندران كوى توانگرتر كسى تو باشى. اما همسرايه‌ی مصلح گزين و حق و حرمت همسايه نگه‌دار، خبر «المؤمن من امن جاره بوائقه». و با مردمان كوى و محلت نيكو باش، و بيماران را به پرسيدن رو و خداوند عزّيه\foonote{صاحب عزا} را به تعزيت رو، و مرده را به جنازه رو. و بهر شغل كه همسرايه را باشد، با وى موافقت كن، اگر شادى بود با همسرايه هم موافقت كن و به طاقت خويش هديه فرست يا خوردنى يا داشتنى تا محتشم‌ترين كوى تو باشى. و كودكان همسرايگان را كه بينى بپرس و بكنار گير و بنواز. و پيران كوى را حرمت دار. و در مَزْگِت\foonote{مسجد} كوى جماعت به پاى دار، و ماهِ رمضان به شمع و قنديل فرستادن تقصير مكن، كه مردمان با هر كسى آن راه دارند كه مردمان با ايشان دارند.

بدان كه هر چه مردم يابد از بَرزيده‌ی\foonote{عمل‌کرده} خويش يابد، پس ناكردنى مكن و ناگفتنى مگوى، كه هر كس كه آن كند كه نبايدكردن آن بيند كه نبايد‌ديدن. اما وطنِ خويش تا بتوانى در شهرهاى بزرگ ساز و اندران شهر باش كه ترا سازگار باشد. و خانه چنان خر كه بام تو از بامه‌اى ديگران بلندتر باشد تا مردمان را در تو ديدار نباشد، و لكن رنج ديدارِ خويش از همسايه بازدار. و اگر ضيعت خرى، بى‌معدن و بى‌همسايه مخر، و تاوان زده و عيب‌گن\foonote{عیب‌گین} مخر. هر چه خرى به فراخ سال خر و تا\foonote{مواضب باش} ضيعت مقسوم و بى‌شبهت يا بى نامقسوم و باشبهت مخر. و خواسته‌ی بى‌مخاطره ضيعت شناس، اما چون ضيعت خريدى، پيوسته در بندِ عمارت باش، هر روز عمارتى به نو[ى] مى‌كن، تا هر وقت دخلى به نو[ى] همى يابى. البته از عمارت كردن ضياع و عقار مياساى، كه ضياع به دخل عزيز بود، كه اگر بى‌دخل باشد چنان دان كه همه بيابانها ضياع تست، كه دهخدا را به دِه قيمت بود و ديه\foonote{ده‌، آبادی} را به دخل\foonote{عمارت} و دخل جز به عمارت حاصل نشود.


\newpage
