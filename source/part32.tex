\addcontentsline{toc}{section}{باب سى و دوم
 - در بازرگانى كردن}
\section*{باب سى و دوم
 - در بازرگانى كردن}

هر چند بازرگانى صناعتى نيست كه آن را پيشه‌ی مطلق توان گفتن ولكن چون حقيقت بنگرى رسوم او بر رسوم پيشه‌وران‌ است و زيركان گويند: اصل بازرگانى بر جهل نهاده‌اند و فرع آن بر عقل، چنانكه گفته‌اند: «لو لا الجهال لهلك الرجال» يعنى اگر نه بى‌خردانندى جهان تباه شدى و مقصود ازين سخن آنست كه هر كه او به طمع فزونى يك درم از شرق به غرب شود و از غرب به شرق شود به كوه و به دريا و تن [و] خواسته را بر مخاطره نهد، از دزد و صُعلُوك\footnote{دزد}  و حيوان مردم‌خوار و ناايمنى راه باك ندارد، از بهر مردمان غرب نعمت شرق رساند و به مردمان شرق نعمت غرب رساند، ناچاره كه آبادانى جهان بدو باشد و اين جز بازرگان نباشد و چنين مخاطره كسى كند كه چشم خرد دوخته باشد. و بازرگانى دو گونه است و هر دو مخاطره است: يكى معامله و يكى مسافره، و معامله مقيمان را بُوَد كه متاع كاسد\footnote{کساد و بی‌قدر}  را بر طمع فروتر بخرند و اين مخاطره بر مال بود و دلير مردى بايد كه او را دل دهد كه چيز كاسد خرد بر اميد فزونى؛ و مسافر را گفتم كه كدام باشد، بر هر دو روى بازرگانى دليريست و ناباكى بر مال و بر تن. و با دليرى بايد كه راستى و امانت دارد و طريق ديانت سپرد و از بهر سودِ خويش زيان ديگران نخواهد و به طمع سود سورى\footnote{ظاهری}  خلق نجويد. و معامله با آن گروه كند كه زير دست او باشند و اگر با بزرگتر از خود كند با كسى كند كه ديانت و مروت دارد و از مردم فريبنده بپرهيزد. و با مردمى كه در متاع بصارت ندارد معامله نكند تا از دركوب\footnote{کوبنده در، طلبکار}  ايمن باشد. و با مردم تنگ‌بضاعت و سفيه معاملت نكند؛ پس اگر كند، طمع از سود ببرد تا دوستى تباه نشود كه بسيار دوستى بسبب اندك‌مايه سوزيان تباه شده باشد. و به طمع بيشى بنسيه معاملت نكند كه بسيار بيشى بود كه كمى بار آورد. و خرد‌نگرش نباشد كه خرد‌نگرش بسيار زيان بود. و اصل تباهى بازرگانى در باتصرفيست، تا از تصرف بتوان خوردن از مايه نبايد خوردن كه بزرگتر زيانى بازرگان را از سرمايه خوردنست.

و بهتر متاعى آنكه به من و رطل\footnote{پیاله}  خرند و به درمسنگ\footnote{از: درم ، مخفف درهم ،در وزن سنگ} فروشند و بترين متاعى برخلاف اينست. و از غله خريدن بر اميد سود بپرهيزد كه غله فروش مادام بد نام باشد و بدنيت. و تمام‌تر ديانتى آنست كه [بر] خريده دروغ نگويد، كافر و مسلمان را بر خريده دروغ گفتن ناپسنديده بود چنانكه من گويم: 
\begin{quote}
اى در دل من فگنده عشق تو فروغ\quad \quad بر گردن من نهاده تيمار تو يوغ\\
عشق تو به جان و دل خريد ستم من\quad \quad دانى به خريده بر نگويند دروغ
\end{quote}
و بايد كه بيع ناكرده هيچ چيز از دست بنه\footnote{مایملک دکان} دهد و در معامله شرم ندارد كه زيركان گفته‌اند كه: شرم بسيار از روزى بكاهد، و محابا كردن از بيشى، عادت نكند كه متصرفان اين صناعت گفته‌اند كه: اصل بازرگانى تصرفست و مروت، [تصرف] مال نگاه دارد و مروت جاه.

چنانكه در حكايتى شنيدم كه: روزى بازرگانى بود، بر دكان بياع\footnote{فروشنده}  به هزار دينار معامله كرده بود. چون معامله به پايان رسيد، ميان بياع و تاجر در حسابى خلاف بود به قيراطى\footnote{واحد وزن، چهار جو و چهار حبه! معادل ۰/۲۵۰ گرم}  زر؛ بياع گفت: تو را بر من دينارى زر باقى است تاجر گفت: دينارى و قيراطى باقيست. برين حديث از بامداد تا نماز پيشين شمار كردند و تاجر صداع\footnote{صُداع: دردسر}  مى‌داد و بانگ همى‌داشت و هيچ‌گونه از قول خويش باز نمى‌گشت تا بياع ضجر شد، دينارى و قيراطى بدو داد. مرد بستد و برفت و هر كسى كه آن ديدند، آن تاجر را ملامت مى‌كردند. چون تاجر برفت، شاگرد بياع از پس تاجر بدويد و گفت: اى خواجه شاگردانه‌ی من بده. تاجر آن زر جمله بدان كودك داد، كودك بستد و بازگشت. بياع گفت: اى حرام‌زاده مردى از بامداد تا نيم‌روز از بهر قيراطى زر بانگ مى‌داشت، ميان قومى درو شرم نمى‌داشت، چه طمع داشتى كه او تو را چيزى دهد؟ كودك زر باستاد نمود. مرد عاجز گشت و با خود گفت: اى سبحان‌اللّه! اين كودك خوب‌روى نيست و ديگر گونه ظنى نتوان بردن و اين مرد بدين بخيلى اين سخا چرا كرد؟ بر اثر بازرگان برفت و گفت اى شيخ، چيزى عجب ديدم از تو، يك روز مرا با قومى در صداع قيراطى نبخشانيدى و اكنون چون زر بستدى جمله به شاگرد من دادى، آن صداع چه بود و اين سخا چيست‌؟ مرد گفت: اى خواجه، عجب مدار كه من مردى بازرگانم و در شرط بازرگانى چنانست كه در وقت بيع و تصرف اگر به يك درم كسى مغبون گردد چنان بود كه به نيم عمر مغبون گردد و اگر در وقت مروت از كسى بى‌مروتى آيد چنان بود كه بر ناپاكى اصل خويش گواهى داده بود، پس من نه مغبونى عمر خواستم و نه ناپاكى اصل.

اما بازرگانى كه كم سرمايه بود، بايد كه از هَنبازى\footnote{همان همبازی، رفاقت، شراکت}  پرهيز كند و اگر هنبازى كند، با كسى غنى و با مروت كند و شرمگين، تا وقتِ حيف ازو حيف برگيرد. و بر وى حيف نكند و به نو سرمايگى متاعى نخرد، كه وى را خرج بسيار افتد و چيزى نخرد كه تغيّر در وى آيد و چيزى مرده و كشته نخرد. و بر سرمايه بخت‌آزمايى نكند، مگر كه داند كه اگر زيانى بود بيش از نيم‌سرمايه نبود. و اگر كسى نامه دهد كه فلان جاى برسان، نخست نامه بخواند آنگه برساند كه بسيار بلاها در نامه‌ی سربسته بود و نتوان دانست كه حال چون بود و شر از كجا ادا كند، اما در نامه‌ی نيازمندان زنهار نخورد. و به هر شهرى كه در شود خبر اراجيف\footnote{سخنان یاوه و دروغ} به خير دهد و چون از راهى اندر آيد خبر تعزيت كس ندهد و [به] خبر تهنيت تقصير نكند. و بى‌هم راه براهى نرود و هم‌راه ثِقه\footnote{مورد اعتماد، معتمد} جويد و در كاروان ميان انبوه فرو آيد و قماشه\footnote{کالا، اثاث}  به جاى انبوه نهد و ميان سلاح‌داران ننشيند و نرود كه صعلوك نخست قصد سلاح‌داران كند. و اگر پياده باشد با سوار هم‌راهى نكند و از مردم بيگانه راه نپرسد، مگر از كسى كه وى را به صلاح داند، كه بسيار مردم ناپاك بود كه راه غلط نمايد و خود از پس بيايد و كالا بستاند. و هر كسى را كه به راه آيد، به تازه‌رويى سلام كند و خويشتن را به مضطرى و درماندگى ننمايد و با رصدبانان خيانت نكند ولكن به چيز و فريفتن ايشان به سخن خوش تقصير نكند. و بى‌زاد به راه نرود و تابستان بى‌جامه‌ی زمستان و زمستان بى‌جامه‌ی تابستان نرود، اگر چه راه سخت آبادان بود. و خفير و مكارى را خشنود دارد و چون جايى فرود آيد آشنا دلير نباشد و بياع با ديانت و امين گزيند و صحبت با سه قوم كند: با مردم جوانمرد پيشه و عيّار و با مردم توانگر و با مروت و با مردم راه‌دان و بوم‌شناس. و جهد كند تا خود را به گرما و سرما و تشنگى و گرسنگى خود كند و اسراف نكند در آسايش، تا اگر وقتى به ضرورتى اوفتد رنجش نرسد. و به هر كارى كه بتوانى كردن، هم تو كن و بر كس ايمن مباش، كه دنيا زود فريبست. اما سرمايه‌ی بازرگانى راستى و ديانت شناختن بود و در خريد و فروخت جلد باش و امين و راست‌گوى و بسيارْخَر و بسيارْفروش باش. و تا بتوانى به نسيه ستد و داد مكن، پس اگر كنى با چند گونه مردم مكن: با مردم كم‌چيز و با مردم نو‌كيسه و با علوى و با كودك و با دانشمند و با وكيلان قاضى و با مفتيان شهر و با خادمان. و هر كه با اين قوم ستد و داد كند از دردسر و زيان و پشيمانى نرهد.

و مردم چيز ناديده را به چيز استوار مدار و بر مردم ناآزموده ايمن مباش و آزموده را هر دم ميازماى و آزموده را به ناآزموده مده و مردم به مردم‌ آزماى، پس به خويشتن كه هر كه به خويشتن نشايد، ممكن بود كه كسى ديگر را هم نشايد و آزموده را به نا‌آزموده مده كه روزگارى بايد دراز تا كسى آزموده و معتمد بدست آيد كه اندر مثل آمده است كه: ديو آزموده بِهْ از مردم ناآزموده. اما هر كه را آزمايى به كردار آزماى نه به گفتار. و     بنجشكى\footnote{گنجشکی} نقد بِهْ دان كه طاووسى به نسيه. و تا در سفر خشك، ده نيمى يابى، به طَمَعِ دَه يازده در دريا منشين، كه سود دريا تا كعب بود و زيانش تا گردن؛ نبايد كه به طمع اندك مايه سود سرمايه‌ی بزرگترين بر باد دهى. بر خشك اگر واقعه‌اى بيفتد كه مال بشود، اگر مال بشود مگر جان بماند و مال را عوض بود جان را عوض نبود. و كار دريا را با كار پادشاه برابر كردند كه به جمع آيد و به جمع رود و لكن از بهر آثار تعجب را يك بار در نشينى روا بود به وقت توانگرى كه رسول صلى اللّه عليه و سلم گفته است: «اركبوا البحر مرّة و انظروا فى آثار عظمة اللّه جل جلاله». و بوقت ستد و داد بى‌مكاس\footnote{کسی که چانه نمیزند که قیمت را پایین بیاورد}  مباش و لكن مكاس در خور اخريان\footnote{کالای خرید و فروخت}  كن و كار خويش جمله به دست كسان باز مهل، چه گفته‌اند كه: به دست كسان مار بايد گرفت.

و سوزيان‌\footnote{نفع. سود. فایده که در مقابل زیان}هاى خويش هميشه نبشته\footnote{نوشته}  دار، تا از بيهوده و غلط ايمن باشى و با معاملان خويش شمار كرده‌دار كه آن چون دادى داده باشد و هيچ چيز بر خويشتن هوشنگى\footnote{بر ذهن خود}  حجت مكن، تا اگر خواهى كه منكر شوى توانى شدن. و پيوسته سود و زيان و كدخدايى خويش و معامله‌ی خود را مطالعه همى‌كن تا از آگاه بودن سوزيان خويش فرو نمانى. و از خيانت كردن با مردم بپرهيز كه هر كه با مردمان خيانت كند پندارد كه خيانت با ديگران كرده است و غلط سوى اوست كه آن خيانت با خويشتن كرده بود.

چنانكه شنيدم كه مردى گوسفندى رمه داشت فراوان، وى را شبانى بود صاين\footnote{حافظ} و پارسا. هر روزى شير گوسفندان چندانكه بودى حاصل كردى و به نزديك خداوند گوسپند بردى. آن مرد هم چندان آب بر شير كردى و به شبان دادى و گفتى: رو بفروش. و شبان آن مرد را نصيحت همى‌كرد و پند همى‌داد كه: چنين مكن و با مسلمانان خيانت مكن و روا مدار كه عاقبت مردم خاين نامحمود بود. و آن مرد سخن شبان نشنود و همچنان همى‌كرد تا به‌     اتفاق شبى اين شبان گوسپندان را در رودكده‌اى بداشته بود و خود بر بلندى رفته و خفته و فصل بهار بود، مگر بر كوه بارانى آمد عظيم و سيلى سخت عظيم بيامد و اندرين رودخانه افتاد و اين گوسفندان را جمله ببرد و هلاك كرد. روز ديگر شبان به شهر آمد و بخانه‌ی صاحب گوسفندان رفت بى‌شير، مرد پرسيد كه: چونست كه شير نياوردى‌؟ شبان گفت: اى خواجه من تو را گفتم كه: آب بر شير مزن و خيانت مكن، فرمان من نبردى اكنون آن آب‌ها جمله گرد شد و بر گوسفندان تو گماشتند و گوسپندان تو جمله ببرد و هلاك كرد. آن مرد پشيمان شد و پشيمانى سود نداشت.

تا بتوانى از خيانت پرهيز كن كه هر كه يك‌بار خاين گشت، بيش كس را برو اعتماد نكند. و راستى پيشه كن كه بزرگترين طرارى\footnote{حیله گری} راستيست و نيك معامله و خوش ستد و داد باش كه تا ده يازده كنى يك‌بار، دو بار ده نيم توان كرد. و كس را وعده مكن و چون كردى خلاف مكن و خريده مگوى و اگر گويى راست گوى تا خداى تعالى بر معاملت تو بركت كند. و در معاملت به حجت دادن و ستدن هشيار باش كه چون حجت بخواهى دادن تا نخست حق به دست نگيرى حجت از دست مده. و هر كجا كه همى‌روى آشنايى همى‌طلب و اگر چه بازرگان باشى به شهرى كه هيچ نرفته باشى با بارنامه و محتشمى رو و به تعريف خويش اگرت حاجت آيد به‌به و اگرنه زيانت نبود نتوان دانست كه حال چون بود. و با مردم ناساخته و جاهل ديدار و بى‌نماز و ناباك‌دار سفر مكن كه گفته‌اند: «الرفيق ثمّ الطّريق.» و هر كسى كه تو را امين دارد گمان او در خويشتن دروغ مكن. و هر چه بخواهى خريدن ناديده و نانموده مخر و آنچه بخواهى فروختن از نرخ نخست آگاه باش و به شرط و پيمان فروش تا آخر آن از گفت‌و‌گوى رسته باشى. و طريقت كدخدايى نگاه‌دار كه بزرگتر بازرگانى كدخدايى خانه است، بايد كه كدخدايى خانه پراگنده نكنى، حوايج خانه‌ی خويش به سالى در به وقت نوقان‌ها جمله به يك‌بار بخرى، هر چه تو را به كار آيد از هر چيزى دو چندان كه تو را به سالى به كار شود بخر. پس از نرخ آگاه باش، چون نرخ گران شود از هر چيزى نيمى بفروش از آنچه جمله خريده باشى تا آن يك سال رايگان خورده باشى و اندرين نه‌بزه بود و نه بد‌نامى و هيچ‌كس تو را اندرين معنى به بخل منسوب نتواند كردن كين از جمله‌ی كدخداييست نه از جمله‌ی بخيلى و اندرين هيچ عيبى نيست. و چون در كدخدايى خويش خلل بينى تدبير آن كن كه دخل خويش بزيادت گردانى تا خلل اندر كدخدايى تو راه نيابد. پس اگر چاره‌ی زيادت كردن دخل ندارى و نتوانى، از خرج بكاه كه همچنان باشد كه در دخل زيادت همى‌كنى.

پس اگر اتفاق بازرگانيت نيافتد و علمى شريف خواهى كه بدانى، از علم دين گذشته هيچ علمى سودمندتر از علم طب نيست كه رسول عليه السلام گفته است كه: «العلم علمان: علم الاديان و علم الابدان» و مراد از علم ابدان طب است.


\newpage
