\addcontentsline{toc}{section}{باب پانزدهم - اندر تمتع کردن}
\section*{باب پانزدهم - اندر تمتع کردن}

بدانچه كسى را دوست دارى، در مستى و هشيارى پيوسته به مجامعت مشغول مباش؛ كه آن نطفه كه از تو جدا مى‌شود معلومست كه به هر بارى تخمِ جانى و شخصى است. پس اگر كنى بارى به مستى مكن، كه آن زيان‌كار بُوَد. اما به وقت خمار صواب‌تر بود. و به هر وقت كه يادت آيد بدان مشغول مباش، كه آن بهايم باشند كه وقت هر شغلى ندانند، آدمى را وقتى بود كه پيدا بود، تا فرقى بود  ميان وى و ميان بهايم.

اما از غلامان و زنان ميل خويش به يك جنس مدار؛ تا از هر دو گونه بهره‌ور باشى، وز دو گونه يكى دشمن تو نباشند. و هم چنان كه گفتم كه مجامعت كردن بسيار زيان دارد ناكردن نيز هم زيان دارد. پس هر چه كنى بايد كه به اشتها كنى نه به تكلف، تا زيان كمتر دارد. امّا به اشتها و نه به ‌اشتها بپرهيز در گرماى گرم و در سرماى سرد، كه اندرين دو فصل زيان‌كارتر باشد خاصه پيران را؛ از همه وقتى، وقت بهار سازد و ازين بود كه در فصل بهار هوا معتدل گردد و چشم‌ها زيادت گردد و جهان روى به خوشى و راحت نهد. پس عالم كه پيرست جوان شود از تأثير وى، تن ما كه عالم صغيرست همچنان شود. طبايع اندر تنِ ما معتدل شود و خون اندر رگ‌هاى ما زيادت شود و منى در پشت زيادت گردد؛ بى‌قصدْمردمِ حاجتمند تمتع و مباشرت گردد.

پس چون اشتها صادق گردد، آنگه زيان كمتر دارد. و رگ همچنين بُوَد، پس تا بتوانى در سرماى سرد و گرماى گرم رگ مزن، و اگر زيادتى بينى اندر خون، تسكينِ خون كن به شراب‌ها و طعام‌هاى موافق. و تابستان ميل به غلامان و زمستان ميل به زنان كن. و مخالف فصل چيزى مخور. و اندرين سخن مختصر كردم كه بيش ازين كرا\footnote{سود، فایده} نكند «و نستغفر».

\newpage
