\addcontentsline{toc}{section}{باب چهاردهم- 
در عشق ورزيدن}

\section*{باب چهاردهم - 
در عشق ورزيدن}



بدان اى پسر! كه تا كسى لطيف طبع نبود عاشق نشود از انچه عشق از لطافت طبع خيزد [و هر چه از لطافت خيزد] بى‌شك لطيف بود. خبر: «من اشبه اباه فما ظلم». چون او لطيف بود ناچاره در طبع لطيف آويزد. نه‌بينى كه جوانان بيشتر عاشق شوند از پيران، از آن كه طبع جوانان لطيف‌تر بود، از پيران. و نيز هيچ غليظ [طبع] و گران [جان] عاشق نشود از آنكه اين علتى است، كه خفيف روحان را بيشتر افتد.

اما تو جهد كن تا عاشق نشوى؛ اگر گرانى و اگر لطيف، از عاشقى بپرهيز كه عاشقى با بلاست؛ خاصه به هنگامِ مفلسى كه هر مفلسى كه عاشقى ورزد معاينه در خون خويش سعى كرده باشد، خاصه كه پير باشد، كه پير را جز به سيم غرض حاصل نشود چنانكه من گويم:

\begin{quote}
\centering
بى‌سيم بدم بر من ازين آمد درد \quad \quad وز بى‌سيمى بماندم از روى تو فرد \\
دارم مثلى بحال خويش اندر خورد \quad \quad بى‌سيم ز بازار تهى آمد مرد
\end{quote}

پس اگر باتفاق تو را وقتى به روزگار با كسى وقت خوش گردد، تو معين دل خود مباش و پيوسته طبع را عشق باختن مياموز و دايم متابع شهوت مباش كه اين نه كار خردمندان باشد. از آنچه مردم [در عشق] يا در وصال باشد يا در فراق، بدان كه يك ساله راحت‌ِ وصال به يك ساعته رنجِ فراق نیارزد، كه سر تا سر عاشقى رنجست و درد دل و محنت، كه هر چند دردى خوش است، اما اگر در فراق باشى در عذاب باشى، و اگر در وصال باشى و معشوقه از دل تو خبر دارد خود از ناز خيره و خوى بد او خوشى وصال ندانى. پس اگر وصالى بُوَد كه بعد از ان فراقى خواهد بود، آن وصال خود از فراق بَتَر بود. و اگر به مثل آن معشوقه‌ی تو فريشته‌ی مقربست، به هيچ وقت از ملامت خلقان رسته نباشى، پيوسته در مساوى تو باشند و در نكوهش معشوق تو، از آنكه عادت خَلْقْ چنين رفته است. پس خويشتن را نگاه دار و از عاشقى پرهيز كن كه بى‌خودان از عاشقى پرهيز نتوانند كردن. از آنچه ممكن نگردد كه به يك ديدار كسى بر كسى عاشق شود. نخست چشم بيند، آنگه دل پسندد. چون دل را پسند اوفتاد، طبع بدو مايل شود. چون طبع مايل گشت، آنگاه دل متقاضى ديدار دُوُمْ باشد. اگر تو شهوتِ خويش در امرِ دلْ كنى و متابع شهوتِ دل گردانى، باز تدبير آن كنى كه يك بار ديگر او را ببينى، چون ديدار دوباره شود، ميل طبع بدو نيز دوباره شود و هواىِ دل غالب‌تر گردد؛ پس قصدِ ديدار سِوُمْ كنى. چون سومْ‌بار ديدى و در حديث آمدى، سخنى گفتى و جوابى شنيدى، خر رفت و رَسَن برد.

پس از آن اگر خواهى كه خويشتن را نگاه دارى، نتوانى كه كار از دست تو در  گذشته بود. هر چه روز بود، عشق تو بر زيادت بود، به ضرورت تو را متابع دل مى‌بايد بود. اما اگر به ديدار اول خويشتن نگاه دارى، چون دل تقاضا كند، خرد را بر دل مُوَكِل كنى تا بيش نام وى نبرد، و خويشتن به چيزى ديگر مشغول همى‌دارى، و جاى ديگر استفراغ\footnote{طلب راحتی کردن}  شهوت همى‌كنى، و چشم از ديدار وى بر بندى كه همه رنج يك هفته بود، بيش ياد نيايد، زود خويشتن را از بلا بتوانى رهانيدن. ولكن اين‌چنين كردن، نه كار هر كسى باشد، مردى بايد با عقلى تمام كه اين علّت را مداوا تواند كردن. از آنچه عشق علّتى است، چنانكه محمد بن زكريا گويد در تقاسيم‌العلل كه سببِ علتْ عشقْ و داروىِ عشقْ چون روزه‌داشتنِ پيوسته بُوَد، و بارِ گران كشيدن، و سفرِ دراز كردن، و دايم خويشتن را در رنج داشتن، و تمتع كردن بسيار، و آنچه بدين ماند.

اما اگر كسى را دوست دارى كه تو را از ديدار و خدمتِ او راحتى بود روا دارم، چنانكه شيخ ابوسعيد ابو‌الخير رحمه اللّه گفته است كه: آدمى را از چهار چيز ناگزير بود: اول نانى، دوم خُلْقانى\footnote{لباس}، و سوم ويرانى\footnote{مسکن، کلبه درویشی}، چهارم جانانى، هر كس بر حدّ و اندازه‌ی او از رویِ حلال. اما دوستى ديگرست و عاشقى ديگر؛ در عاشقى كس را وقتِ خوش نَبُوَد، هر چند آن بود كه آن مرد عاشق گويد در بيتى:
 
\begin{quote}
\centering
اين آتش عشق تو خوش است اى دلكش \quad \quad هرگز ديدى آتش سوزنده‌ی خوش
\end{quote}

و بدان كه در دوستى، مردم هميشه با وقتى خوش بود و در عاشقى دايم اندر محنت بود. اگر به جوانى عشق‌ ورزى، آخر عذرى بود هركس كه بنگرد و بداند، معذور دارد، گويد كه جوانست؛ جهد كن، تا به پيرى عاشق نشوى كه پير را هيچ عذرى نباشد.

چنانكه از جمله‌ی مردمان عام باشى كار آسان‌تر بود پس اگر پادشاه باشى و پير باشى زينهار، تا ازين معنى انديشه نكنى و به ظاهر دلْ دَرْ کَسْ نبندى، كه پادشاه را به پيران‌ْسر عشق باختن دشوار كارى بود.

حكايت: چنانكه به روزگار جدّ من شمس‌المعالى خبر آوردند كه: بازرگانى به بخارا بنده‌اى دارد بهايى\footnote{گران‌بها}. احمد سغدى اين حكايت پيش امير بگفت و گفت: ما را كسى بايد فرستادن تا آن غلام را بخرد. امير گفت: تو دانى يا سغدى. نخاس\footnote{برده فروش} را بفرستاد و آن غلام را به هزار [و] دويست دينار بخريد و به گرگان پيش امير آوردند. امير بديد و پسنديد اين غلام را؛ دستاردارى داد كه چون دست بشستى، دستار روى بدو دادى تا دستِ تَرْ خشك كردى؛ تا چَنْد‌ْگاهى برآمد. روزى امير دست پاك همى كرد و بدين غلام همى‌نگريست، بعد از ان كه دست خشك كرده بود، در آن ميزر\footnote{عمامه، پیش‌بند} دست مى‌ماليد و درين غلام مى‌نگريد، مگر بچشم وى خوش همى‌آمد؛ دستار باز داد. چون زمانى ازين حال بگذشت، ابوالعباس غانمى را گفت كه: اين غلام را آزاد كردم و فلان دِه وى را بخشيدم، مَنشور بنبيس\footnote{بنویس} و از شهر دختر كدخدايى براى وى بخواه تا به خانه‌ی خويش بنشيند و تا آنگه كه ريش بر نياورد نخواهم، كه از خانه بيرون آيد. ابوالعباس غانمى وزير بود، گفت: فرمان خداوند راست، امّا اگر راى خداوند اقتضا كند، بنده را بگويد كه مقصود اندرين چيست‌؟ امير گفت: امروزْ حالْ چِنينُ و چِنينْ رفت و سخت زشت بود پادشاهى هفتاد ساله و عاشق، مرا بعدِ هفتاد سال به نگاه‌داشتِ بندگانِ خداى تعالى مشغول بايد بودن و به صلاح رعيت و لشكر و مملكت خويش، من به عشق مشغول باشم، نه نزديك خداى معذور باشم، نه نزديك خَلْقان.

بلى جوان هر چه كند معذور باشد [امّا يك‌باره ظاهر عاشق نبايد بودن، هر چند جوان باشد] تا [در] طريق سياست و حشمت خلل راه نيابد.

حكايت: چنانكه به غزنى در شنودم كه ده غلام بود در خزانه‌ی سلطان مسعود، جامه‌داران خاصِّ او بودند. و از جمله‌ی ايشان يكى بود نوشتگينِ نوبى گفتندى. سلطان مسعود وى را دوست داشت. چند سال برآمد ازين حديث كه هيچ كس نتوانست دانست كه سلطان مسعود كه را دوست دارد؟ و از جمله‌ی اين ده غلام كس ندانست كه معشوق و منظور سلطان مسعود از آن جمله كدامست‌؟ از آنكه هر عطائى كه بدادى، همه چنان دادى كه نوشتگين را، تا هر كسى پنداشتى كه معشوق اوست و مقصود خود نوشتگين بود و كس ندانست. تا ازين حال پنج سال برآمد، روزى اندر مستى فرمود كه: هر چه پدرِ مَنْ اياز\footnote{غلام سلطان محمود} را فرموده بود، همان به قطاع\footnote{زمین} و معاش، جمله نوشتگينِ‌ نوبى را منشور نبيسند. آنگاه بدانستند كه مقصود او نوشتگين نوبى بوده‌است.

اكنون اى پسر! هر چند كه اين قصه بگفتم، اگر تو را اتفاقِ عشق اوفتد، دانم كه بر قولِ من كارْ نكنى، كه خود به پيران‌ْسر بيتى همى گويم اندر حالِ عشقْ:

\begin{quote}
\centering
هر آدميى كه حىّ و ناطق باشد \quad \quad بايد كه چو عُذرا و چو وامِقْ باشد\\
هر كو نه چنين بُوَد، منافق باشد \quad \quad مؤمن نَبُوَد كه او نه عاشق باشد
\end{quote}

هر چند كه من چنين گفته‌ام، تو برين دو بيتى كار مكن، جهد كن تا عاشق نشوى.


اگر كسى را دوست دارى، بارى كسى را دوست دار كه به دوستى ارزد. معشوقِ خود بطلميوس و افلاطون نباشد و لكن بايد كه اندك مايه خردى دارد. و نيز دانم كه يوسفِ يعقوب نباشد، اما چنان بايد كه حلاوتى و ملاحتى باشد وى را؛ تا زبانِ مردم بسته باشد و عُذرْ مقبول دارند، كه مردم را از عيب كردن و عيب جستن يك ديگر چاره نباشد؛ چنانكه يكى را گفتند كه: عيبت هست‌؟ گفت: نه. گفتند: عيب جويت هست‌؟ گفت: هست. گفتند: پس چنان‌دان كه معيوب‌ترين كسى تويى. اما اگر به ميهمانى روى، معشوق را با خويشتن مبر؛ و اگر برى پيش بيگانگان به وِى مشغول مباش و دل درْ وِى بسته مدار، كه خود وى را كسى به نتواند خوردن. و مپندار كه وى بچشم همه كسى چنان درآيد، كه به چشم تو درآمده باشد؛ چنانكه شاعر گويد:

\begin{quote}
\centering
اى واى، منا گر تو به چشم همه كسها\quad \quad زان گونه نمايى كه به چشم من درويش\\
\end{quote}
چنانكه به چشم تو نيكوتر از همه مى‌نمايد مگر به چشم ديگران زشت‌تر نمايد. و نيز هر زمانى وِى را ميوه مده، و هر ساعتى وِى را مخوان، و در گوش وى سخن مگوى؛ يعنى كه من سود و زيانى همى‌گويم كه مردمان دانند كه تو با وِى چيز نگفتى و السلام.

\newpage


























