\addcontentsline{toc}{section}{باب سيزدهم- 
اندر مزاح كردن و نرد و شطرنج باختن}

\section*{باب سيزدهم- 
اندر مزاح كردن و نرد و شطرنج باختن\footnote{بازی کردن}}

بدان اى پسر! كه به تازى گفته‌اند: «المزاح مقدمة الشّر»؛ تا بتوانى از مزاحِ سرد كردن پرهيز كن. و اگر مزاح كنى، بارى، در مستى مكن كه شر بيشتر خيزد، كه مزاح پيش رو شرست؛ و از مزاح ناخوش و فحش شرم‌دار، اندر مستى و هشيارى، خاصّه در نرد و شطرنج باختن، كه در ميان اين هر دو شغل مرد [ضجر\footnote{آرام}]تر باشد، مزاح كمتر بر تواند داشتن. و نرد و شطرنج باختنِ بسيارْ عادت مكن، اگر بازى به اوقات باز و به گرو به‌ مباز الاّ به مرغى يا به مهمانى يا بچيزى از محقرات؛ به درم مباز كه بى‌درم باختن ادبست و به درم باختن مقامرى\footnote{قمار بازی}. و اگر چه نيك‌دانى باختن با كسى كه به مقامرى معروف [بود مباز كه تو نيز بمقامرى معروف] شوى. و اگر با كسى محتشم‌تر از خويشتن بازى در نرد و شطرنج، ادب هر دو\footnote{نرد و شطرنج} آنست كه تو دست به مهره نكنى، نخست او آنچه خواهد برگيرد. اگر نرد باشد، نخست كعبتين\footnote{تاس} بدو ده تا كتار كند و در شطرنج در دستِ اول بازى بدو بده. اما با مستان و تركان و معربدان و گران‌جانان هرگز به گرو مباز تا عربده نخيزد. و بر نقش كعبتين با حريف جنگ مكن و سوگند مخور كه تو فلان زخم زدى، كه اگر چه راست گويى همه كس گويد كه دروغ همى‌گويد.

و اصل همه شرّ و عربده، مزاح كردنست؛ پرهيز كن از مزاح كردن هر چند مزاح كردن نه عيب است و نه بزه، كه رسول صلى اللّه عليه و سلم نيز مزاح كرده است. 

و اندر خبرست كه پيرزنى بود در خانۀ امّ المؤمنين عايشه رضى اللّه عنها، روزى از رسول عليه السلام پرسيد كه: اى رسول، روىِ من روىِ بهشتيانست يا روىِ دوزخيان، و من بهشتى خواهم بودن يا دوزخى‌؟ و گفته‌اند: «كان رسول اللّه صلى اللّه عليه و سلم يمزح و لا يقول الاّ حقا.» پس پيغامبر صلى اللّه عليه و سلم گفت بر روىِ مزاح كه: در آن جهان هيچ پيرزن اندر بهشت نباشد. آن پيرزن دل‌تنگ شد و بگريست. آنگه رسول عليه السلام تبسم كرد و گفت: مگرى كه سخن من خلاف نباشد، راست گفتم كه هيچ پير در بهشت نباشد از آنكه روز قيامت همه خلق از گور جوان خيزند. پير زن را دل خوش گشت. 

اما مزاح شايد كرد و لكن فحش نبايد گفت، پس اگر گويى و كنى با كمتر از خويش مكن و مگوى تا حشمت خويش در سر جواب او نكنى. اگر ناچاره بود آنچه گويى با هم‌سران خويش گوى تا اگر جوابى  دهند عيبى نبود. و اما هزلى كه گويى جِدْ آميخته گوى و از فحش بپرهيز. هر چند مزاح بى‌هزل نه‌بود اما تا حدّى بايد، كه خوار كننده‌ی همه قدرها مزاح است، هر چه بگويى ناچاره بشنوى. از مردمان همان چشم‌دار كه از تو به مردمان رسد. اما با هيچ كس جنگ مكن كه جنگ كردن نه كار محتشمانست بل كار زنانست يا كار كودكان؛ پس اگر اتفاق افتد كه با كسى جنگ كنى هر چه بدانى و بتوانى گُفْتْ مگوى، جنگ چندان كن كه آشتى را جاى بود. و يك‌باره بى‌آزرم و لجوج مباش و از عادت‌هاى مردمان فرومايه بَتَرين عادتى لجوجى شناس و بهترين عادتى متواضعى است كه متواضعى نعمت ايزديست كه كس برو حسد نبرد. و بهر سخنى مگوى در خطاب كه: اى مرد، كه اى مرد گفتنِ بى‌حجت، مرد را از مردمى بيفگند.

اما سيكى خوردن و مزاح كردن و عشق باختن اين همه كار جوانانست، چون حدّ و اندازه نگاه دارى بر نيكوترين وجهى بتوان كردن، و هم بتوان پرهيز كردن چون خرد را كارفرمايى. و اندر سيكى خوردن و مزاح كردن لختى گفته آمد، در باب عشق ورزيدن نيز آنچه دانيم بهرى بگوييم؛ ندانم كه تو به جاى توانى آوردن يا نه‌؟ كه با دل داورى كردن كارى دشوارست.


\newpage
























