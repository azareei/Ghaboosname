\addcontentsline{toc}{section}{باب بیست‌و‌ششم - اندر آیین زن خواستن}
\section*{باب بیست‌و‌ششم - اندر آیین زن خواستن}

و چون زن خواستى اى پسر حرمت خود را نيكو دار. اگر چند چيز عزيزست، از زن و فرزند خود دريغ مدار. اما از زن به صلاح و [فرزند] فرمان‌ بردار و مهربان و اين كار توست كه به دست تست، چنانكه من در بيتى گويم: 

\begin{quote}
فرزند چو پرورى و زن چون دارى
\end{quote}

اما چون زن كنى، طلب [مالِ] زن مكن و طلب غايت نيكويى زن مكن، كه به نيكويى معشوقه گيرند. زن پاك‌روى و پاك‌دين بايد و كدبانو و شوى‌دوست\footnote{شوهر دوست}  و پارسا و شرم‌ناك و كوتاه‌دست و كوتاه‌زبان و چيز‌نگاه دارنده بايد كه باشد تا نيك بود كه گفته‌اند كه: زن نيك عافيت زندگانى بود. اگر‌چه زن مهربان و خوب‌روى و پسنديده‌ی تو باشد، تو يك‌باره خويشتن را در دست او منه و زير فرمان او مباش كه اسكندر را گفتند كه: چرا دختر دارا را به زنى نكنى كه بس خوبست‌؟ گفت: سخت زشت باشد كه چون ما بر مردان جهان غالب شديم زنى بر ما غالب شود. اما زن محتشم‌تر از خويشتن مخواه، و تا دوشيزه يابى، شوى كرده مخواه، تا در دل او جز مهر تو مهر كسى ديگر نباشد و پندارد كه همه مردان يك گونه باشند، طمع مردى ديگرش نباشد. و از دستِ زنِ باد‌دست\footnote{کنایه از تهی‌دست}  و زفان‌دار و ناكدبانو بگريز كه گفته‌اند كه: كدخداى رود بايد [و كدبانو بند امّا نه] چنانكه چيز تو را در دست گيرد و نگذارد كه تو بر چيز خويش مالك باشى، كه آنگه تو زن او باشى نه او زن تو. و زن از خاندان به صلاح بايد خواست و نبايد كه دختركى بود، كه زن از بهر كدبانويى بايد خواست نه از بهر طبع، كه از بهر شهوت از بازار كنيزكى توان خريد كه چندين رنج و خرج نباشد. بايد كه زنى رسيده و تمام و عاقله باشد، كدبانويى و كدخدايى مادر و پدر ديده باشد، تا چنين زنى يابى درخواستن او هيچ تقصير مكن و جهد كن تا وى را بخواهى. و ديگر بكوش تا به هيچ وجهى او را غيرت ننمايى و اگر رشك خواهى نمود خود، نخواهى بهتر بود كه زن را رشك نمودن به ستم ناپارسايى آموختن بود. و بدان كه زنان به غيرت، بسيار مردان را هلاك كنند و نيز تن خويش را فراز كمتر كسى دهند از رشك و حميت\foonote{ ننگ و عار داشتن، در اصطلاح ، حمیت عبارتست از آنکه در محافظت ملت یا حرمت از چیزهایی که محافظت از آن واجب بود تهاون ننماید} و باك ندارند. اما چون زن را رشك ننمايى و با وى دو كيسه نباشى بدانچه خداى تعالی تو را داده بود، وى را نيكو دارى، از مادر و پدر و فرزند تو بر تو مشفق‌تر بود، خويشتن را از وى دوست‌تر كس مدار. و اگر غيرتش نمايى از هزار دشمن دشمن‌تر شود بر تو و از دشمن بيگانه حذر توان كرد و از وى نتوان كرد.

و چون زن دوشيزه خواستى اگر چه بوى مولع باشى، هر شب با وى بازى مكن كه وى از تو بدان نيازارد، پندارد كه خود همه خلق چنانند تا اگر وقتى تو را عذرى يا سفرى باشد اين زن را بى تو صبر بود، و اگر هر شب با وى خفتن عادت كنى وى را همان آرزو كند، دشوار صبر تواند كردن. و زنان را به ديدار و نزديكى هيچ مرد استوار مدار، اگر چه مرد پير و زشت، و هيچ خادم را در خانه‌ی زنان راه مده و اگر چه سياه و ساده باشد مگر سياهى زشت و پير و ممسوخ\footnote{مسخ شده ، تغییر شکل و صورت داده} بود. و شرط غيرت نگاه‌دار و مرد بى‌غيرت را به مرد مدار
كه هر كه را غيرت نباشد، وى را دين نباشد. و چون زن خويشتن را برين جمله داشتى اگر خداى تعالى تو را فرزندى دهد، انديشه كن به پروردن فرزند خويش.
\newpage
