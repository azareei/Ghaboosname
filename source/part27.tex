\addcontentsline{toc}{section}{باب بیست و هفتم - در حق فرزند و حق شناختن}
\section*{باب بیست و هفتم - در حق فرزند و حق شناختن}

اگر پسريت آيد، اى پسر، اول چيزى بايد كه نام خوش برو نهى كه از جمله حق‌هاى پدران بر فرزندان يكى آنست كه او را نام خوش نهد. دوم آنكه: 
به دايگان عاقل و مهربان سپارد، و بوقت سنّت كردن، سنّت كنى و به حسب طاقتِ خويش شادى كنى. و قرآن‌اش بياموزى تا حافظ قرآن شود. و چون بزرگ‌تر شود، اگر رعيت باشى، وى را پيشه‌اى بياموزى و اگر اهل سلاح باشى، به معلم سلاح دهى تا سوارى و سلاح شوريدن بياموزد و بداند كه بهر سلاحى كار چون بايد كردن. و چون از سلاح آموختن فارغ گردى، بايد كه فرزند را شناو كردن بياموزى، چنانكه من چون ده ساله شدم ما را حاجبى\footnote{دربان، قاپوچی} بود با منظرِ، گفتندى وى را، رايضى\footnote{مربی اسب} و فروسيّت\footnote{سوارکاری} نيكو دانستى. پدرم رحمه‌اللّه مرا به وِى سپرد تا مرا سوارى و زوبين\foonote{نیزه} و تير انداختن و نيزه باختن و كمند افگندن جمله هر چه در باب فروسيّت و رجوليت بود بياموختم. پس حاجب با منظر و ريحان خادم پيش امير شدند و گفتند: اى خداوند، خداوندزاده هر چه ما دانستيم بياموخت، خداوند فرمان دهد تا فردا به نخجيرگاه آنچه آموخته است بر خداوند عرضه كند. امير گفت: نيك آيد. روز ديگر برفتم و هر چه دانستم بر پدر عرضه كردم. امير ايشان را خلعت فرمود. پس گفت: اين فرزند مرا آنچه بياموخته‌اى نيك بدانسته است و لكن بهترين هنرى نياموخته است. گفتند: آن چه هنرست‌؟ امير گفت: اين همه هر چه داند از معنى هنر و فضل همه آنست كه اگر به وقت حاجت اگر وى نتواند كرد ممكن بود كه كسى از بهر وى بكند، آن هنر كه وى را بايد كرد از بهر خويش و هيچ كس از بهر وى نتواند كرد وى را نياموخته‌ايد. ايشان پرسيدند كه آن كدام هنرست‌؟ امير گفت: شناو كردن كه از بهر وى جز وى كسى نتواند كردن. و دو ملاح جلد را از آبسكون بياوردند و مرا بديشان سپرد تا مرا شناو بياموختند به كراهيت نه به طبع و لكن نيك بياموختم. تا اتفاق اوفتاد كه آن سال كه به حج همى رفتم از راه شام بر در موصل، ما را قطع افتاد و قافله بزدند و عرب بسيار بودند و ما با ايشان بسنده نبوديم. در جملة‌الامر من برهنه به موصل آمدم، هيچ چاره نداشتم، اندر كشتى نشستم به دجله و به بغداد رفتم و آنجا كار نيكو شد و ايزد تعالى توفيق حج داد. غرضم آنست كه اندر دجله پيش از آنكه به عبكره رسند جايى مخوفست و گردابى صعب، چنانكه اوستادى جلد بايد كه ملاحى داند تا آنجا بگذرد، كه اگر صرف آن نداند كه چون بايد گذشت كشتى هلاك شود. ما چند تن اندر كشتى بوديم بدان جاى رسيديم. ملاح اوستاد نبود، ندانست كه چون بايد گذشتن، كشتى به غلط اندر ميان آن جايگه برد و غرقه گشت. قريب بيست و پنج مرد بوديم، من و مردى ديگر بصرى و غلامى از آن من زيرك كيكاوسى نام به شناو بيرون آمديم؛ ديگران جمله هلاك شدند. بعد از آن مهر پدر اندر دل من زيادت گشت، در صدقه دادن از بهر پدرم و ترحم فرستادن زيادت كردم و بدانستم كه آن پير اين چنين روزى را پيش همى‌ديد كه مرا شناوگرى آموخت و من ندانستم.


پس بايد كه هر چه آموختنى باشد، از فضل و هنر، فرزند را همه بياموزى تا حق پدرى و شفقت پدرى به جاى آورده باشى كه از حوادث عالم ايمن نتوان بود و نتوان دانست كه بر سر مردمان چه گذرد. هر هنرى و فضلى روزى به كار آيد، پس در فضل و هنر آموختن تقصير نبايد كردن. و در هر علمى كه مرو را آموزى اگر معلمان از بهر تعليم مر او را بزنند شفقت مبر، بگذار تا بزنند كه كودك علم و ادب و هنر به چوب آموزد نه به طبع خويش. اما اگر بى‌ادبى كند و تو از وى در خشم شوى، به دست خويش وى را مزن، به معلمانش بترسان و ادب كردن ايشان را فرماى كردن تا كينه‌ی تو اندر دل وى نماند. اما با وى هميشه صبور باش، تا ترا خوار نگيرد و دايم از تو ترسان بود. و درم و زر و آرزويى كه وى را بايد از وى باز مدار تا از بهر درم مرگ تو نخواهد از بهر ميراث. و نان فرزند ادب آموختن دان و فرهنگ دانستن، اگر چه بد روز فرزندى بود تو بدان منگر، شرط پدرى به جاى آر و اندر ادب آموختن وى تقصير مكن هر چند كه اگر هيچ مایه خرد ندارد اگر تو ادب‌آموزى و اگر نياموزى خود روزگارش بياموزد، چنانكه گفته‌اند: حكمت «من لم يؤدّبه والداه ادّبه الليل و النهار\footnote{هر که پدر و مادرش او را ادب نکنند، روز و شب ادبش خواهند کرد}» و همين معنى به عبارتى ديگر جدّ من شمس‌المعالى، رحمةاللّه‌عليه، گويد: [من لم يؤدّبه الابوان يؤدّبه الملوان\footnote{هر که او را پدرش ادب نکرد، کشتیبان ادب کرد}]. اما تو شرط پدرى نگاه دار كه وى خود چنان زيد كه فرستاده باشد، مردم چون از عدم به وجود آيد خلق و سرشت وى با او باشد اما از بى‌قوتى و عجز و ضعيفى پيدا نتواند كردن، هرچند بزرگ‌تر همى‌شود و جسم و روح وى قوى‌تر همى‌گردد فعل وى پيدا همى‌گردد، نيك‌و‌بد، تا چون وى به كمال رسد، عادت وى نيز به كمال رسد، تمامى روزْبهى و روز‌ْبَتَرى پيدا شود. و لكن تو فرهنگ و هنر را ميراث خود گردان و به وى بگذار تا حق وى گزارده باشى كه فرزندان مردمان خاصه را بِهْ از هنر و ادب و فرهنگ نيست و فرزندان عامه را ميراث بِهْ از پيشه نيست، هر چند پيشه نه كار كودكان محتشمانست، هنر ديگرست و پيشه ديگر. اما از روى حقيقت، نزديك من، پيشه بزرگترين هنرست و اگر فرزندان مردمان خاصه صد پيشه دانند چون به كسب نكنند همه هنرست و هنر يك روز به برآيد.

چنانكه گشتاسپ چون از مستقر خويش بيفتاد، و آن قصه درازست، اما مقصود اينست كه وى به روم افتاد در قسطنطنيه رفت، با وى هيچ چيز نبود از دنياوى و عيب مى‌داشت نان خواستن. مگر اتفاق چنان افتاده بود كه به كودكى در سراى پدر خويش آهنگران ديده بود كه كاردها و تيغ‌ها و ركاب‌ها و دهان‌ها كردندى مجاور، و مگر در طالع وى اين صناعت اوفتاده بود. هر روز گرد ايشان همى‌گشتى و همى‌ديدى و اين صناعت بياموخته بود. و اين روز كه به روم درماند هيچ حيله ندانست، به دوكان آهنگران رفت و گفت: من اين صناعت دانم. وى را به مزدور گرفتند و چندانكه آنجا بود از آن صناعت همى‌زيست و به كس نيازش نبود؛ تا آن وقت كه به وطن خويش باز رسيد، چنانكه شنيده‌اى، بعد از ان بفرمود كه هيچ محتشم فرزند خويش را صناعت آموختن عيب مداريد كه بسيار وقت باشد كه ابوّت و شجاعت سود ندارد، هر دانشى كه بدانى روزى به كار آيد، و بعد از آن در عجم آن رسم در افتاد كه هيچ محتشم نبود كه صناعتى ندانستى هر چند بدان حاجتش نبودى و آن به عادت كردند.

پس هر چه بتوانى آموختن بياموز كه منافع آن به تو بازگردد. اما چون پسر بالغ گشت بنگر اندر وى اگر سر صلاح و كدخدايى دارد و دانى كه بزن و كدخدايى مشغول خواهد شدن پس تدبير زن خواستن كن و زنش بده تا آن حق نيز گزارده باشى. اما اگر پسر را زن همى‌دهى و اگر دختر را به شوى دهى با خويشاوندان خويش وصلت مكن و زن از بيگانگان خواه كه با قرابات خويش اگر وصلت كنى و اگر نكنى ايشان خود خون و گوشت توند، پس زن از قبيله‌ی ديگر خواه تا قبيله‌ی خويش را بدو قبيله كرده باشى و بيگانه را خويش گردانيده تا قوت تو يكى دو باشد و از دو جانب تو را معونت كنان باشند. پس اگر دانى كه سر كدخدايى و روزبهى ندارد پس دختر مسلمانى را با وى در بلا مفگن كه هر دو از يك ديگر برنج باشند، بگذار تا چون بزرگ شود خود چنانكه خواهد كند يا به زندگانى تو يا بعد از مرگ تو كه به همه حال چنان تواند بود كه فرستاده باشند. 

اگر دختريت باشد وى را به دايگان مستور سپار و نيكو بپرور، و چون بزرگ شود به معلم ده تا نماز و روزه و آنچه در شريعت است بياموزد و لكن دبيرى مياموزش. و چون بزرگ شود جهد آن كن كه هر چند زودتر به شويش دهى كه دختر نابوده بِهْ، و چون بِهْ بُوَد يا به شوى بِهْ يا به گور، كه صاحب شريعت ما صلى‌اللّه‌عليه‌و‌سلم گويد: «دفن البنات من المكرمات». اما تا در خانه‌ی توست مادام بر وى به رحمت باش كه دختران اسير مادر و پدر باشند، كه پسران اگر پدر ندارند ايشان به طلب شغلى توانند‌ رفت و خويشتن توانند داشت، دختر بيچاره بود آنچه دارى نخست در وجه برگ وى كن و شغل وى بساز و وى را در گردن كسى كن تا از غم وى برهى. اما اگر دخترت دوشيزه باشد داماد دوشيزه كن تا چنانكه زن دل در شوى بندد شوى نيز دل در وى بندد.

چنانكه شنيدم كه چون شهربانو دختر يزدگرد شهريار را اسير بردند از عجم به عرب، امير‌المؤمنين عمر‌ِ خطّاب، رضى‌اللّه‌عنه، فرمود كه وى را بفروشيد. چون وى را بيع خواستند كردن امير‌المؤمنين على، رضى‌اللّه‌عنه، فراز رسيد؛ گفت: قال رسول‌اللّه، صلى‌اللّه‌عليه‌و‌سلم «ليس البيع على ابناء الملوك»، چون وى اين خبر بداد بيع از شهربانو برخاست. او را بخانه‌ی سلمان فارسى بنشاندند تا به شوى دهند. چون شوى برو عرضه كردند، شهربانو گفت: تا مرد را نبينم زن او نباشم، مرا بر منظره‌اى بنشانيد و سادات عرب را بر من بگذرانيد تا آنكه مرا اختيار افتد شوى من باشد. در خانه‌ی سلمان وى را بر منظره‌اى بنشاندند و سلمان به بَرِ او بنشست و آن قوم را تعريف همى‌كرد كه اين فلانست و آن فلانست. وى هر كسى را نقصى همى كرد تا امير‌المؤمنين عمر، رضى‌اللّه‌عنه، برگذشت، شهربانو پرسيد كه: اين كيست‌؟ سلمان گفت: امير‌المؤمنين عمرِ خطاب، رضى‌اللّه‌عنه. شهربانو گفت: مردى محتشم است و بزرگوار اما پيرست. امير المؤمنين على، عليه‌السلام، بر گذشت، پرسيد كه: اين كيست‌؟ سلمان گفت: پسر عمّ پيغامبر ماست، على‌بن‌‌ابى‌طالب، عليه السلام. گفت: مردى سخت بزرگوارست و سزاى منست اما مرا بدان جهان از فاطمه‌ی زهرا، رضى‌اللّه عنها، شرم آيد، ازين جهت نخواهم. پس امير المؤمنين حسن‌بن‌على، رضى اللّه عنهما، برگذشت، پرسيد و گفت: اين در خور منست و لكن بسيار نكاح است نخواهم؛ تا امير‌المؤمنين حسين، رضى‌اللّه‌عنه، برگذشت. ازو بپرسيد، گفت: شوى من اين بايد كه باشد كه دختر دوشيزه را شوى دوشيزه بايد و من هرگز شوى نكرده‌ام و او زن نكرده است.

اما داماد نيكو‌روى گزين و دختر به مرد زشت‌روى مده كه دختر دل بر شوهر زشت روى ننهد تو را و شوى را بدنامى بُوَد. پس بايد كه داماد پاك‌روى و پاك‌دين و به‌اصلاح و با بسيار كدخدايى باشد چنانكه تو نان و نفقات دختر خويش دانى كه از كجا و از چه و چون خواهد بودن. اما بايد كه داماد از تو فروتر بود هم به نعمت و هم به حشمت تا وى به تو فخر كند نه تو به وِى تا دخترت به راحت و به استر و بزرگى زيد. و چون چنين آمد كه گفتم از وى چيزى بيشتر مطلب و دختر فروش مباش كه او خود اگر مردم باشد مروت خويش به جاى آرد، تو آنچه دارى بذل كن و دختر خود را در گردن وى كن و برهان خود را از محنتى عظيم. و هر دوستى كه تو را بود وى را همين پند ده تا برين جمله او نيز برود.
\newpage
