\addcontentsline{toc}{section}{باب یازدهم - در آیین شراب خوردن}
\section*{باب یازدهم - در آیین شراب خوردن}


امّا به حديث شراب خوردن؛ نگويم كه شراب خور و نيز نتوانم گفتن كه مخور كه جوانان به قولِ كسى، از جوانى بازنگردند؛ مرا نيز بسيار گفتند [و نشنيدم] تا از پسِ پنجاه سال ايزدتعالى رحمت كرد و توفيق توبه ارزانى داشت. اما اگر نخورى سودِ هر دو جهان با تو بُوَد و هم خشنودى ايزدتعالى بيابى و هم از ملامت خلقان و از نهاد و سيرت بى‌عقلان و فعل‌هاى محال رسته باشى و نيز در كدخدايى بسيار توفير باشد. وزين چند روى اگر نخورى دوست‌تر دارم و لكن جوانى و دانَم كه رفيقانِ بد نگذارند كه نخورى و بدين گفته‌اند كه: «الوحدة خير من جليس السوء.» پس اگر خورى، دل بر توبه دار و از ايزدتعالى توفيق توبه همى‌خواه و بر كردارِ خويش پشيمان همى‌باش،  مگر توفيق توبه دهد و توبۀ نصوح ارزانى دارد به فضل خويش. پس به‌هرحال اگر نبيد\footnote{شراب خرما} خورى بايد كه بدانى كه چون بايد خورد از آنچه اگر ندانى خوردن زهرست و اگر بدانى خوردن پادزهرست و على‌الحقيقه خود همه مأكولات مطعمة و مشربة كه خورى اگر اسراف كنى زهر گردد و ازين سبب گفته‌اند:

\begin{quote}
كه پازهر زهرست كافزون شود \quad \quad 
چو ز اندازه‌ی خويش بيرون شود
\end{quote}

پس بايد كه چون نان خورده باشى در وقت نبيد نخورى تا سه بار تشنه شوى يا آب يا فقاع به كار برى. پس اگر تشنه نگردى مقدار سه ساعت پس از نان خوردن توقف كنى از آنكه معده اگر چه درست و قوى باشد و اگر چه بى‌اسراف طعام خورى به هفت ساعت هضم كند، به سه ساعت بپزاند و به سه ساعت ديگر قوت طعام بستاند و به جگر رساند تا جگر قسمت كند بر اعضاى مردم، از آنكه قسام اوست، و ساعت ديگر آن ثفل را كه بماند به روده رساند، هشتم ساعت بايد كه خالى شده باشد. هر معده كه [نه] برين قوت باشد آن كدوى باشد نه معده؛ پس ازين كه گفتيم سه ساعت از طعام گذشته نبيد خور تا طعام در معده بپخته باشد، تا چهار طبع تو از طعام نصيب برداشته بوند، آنگه نبيد خور تا هم از شراب بهره‌ور باشى و هم از طعام. اما آغاز سيكى خوردن نمازِ ديگر كن تا چون مست شوى شب درآمده باشد و مردمان مستى تو نبينند. و در مستى نقلان مكن كه نقلان نامحمود بود و گفته‌اند مثل: «النقلة مثلة.» و به دشت و به باغ به سيكى خوردن كمتر رو، پس اگر روى مستى را سيكى مخور؛ باز خانه آى و مستى بخانه كن كه آنچه زير آسمانه توان كرد زير آسمان نتوان كرد، كه سايۀ سقف پوشنده‌تر از سايۀ درخت بود. از آنكه مردم در چهار ديوار خويش چون پادشاهى بُوَد در مملكت خويش و اندر دشتِ مردم چون مردى غريب باشد اندر غربت، و اگر چه محتشم بُوَد و منعِم بُوَد غريب بُوَد، پيدا باشد كه دست غريبان تا كجا رسد. و هميشه از نبيد چنان پرهيز كن كه هنوز دو سه نبيد را جاى باشد و پرهيز كن از لقمه‌ی سيرى و قدح مستى كه سيرى و مستى نه در همه‌ی طعام و شراب بود (33 پ) كه سيرى در لقمه‌ی بازپسين بود چنانكه مستى در قدح بازپسين. پس لقمه‌اى نان و قدحى سيكى كمتر خور تا از فزونى هر دو ايمن باشى. و جهد كن تا هميشه مست نباشى كه ثمره‌ی سيكى خوارگان دو چيزست: يا بيمارى يا ديوانگى كه سيكى خواره دايم يا مست بود يا مخمور، چون مست بود از جمله‌ی ديوانگان بود و چون مخمور بود از جملۀ بيماران بود، كه خمار نوعيست از بيمارى. پس چرا مولع بايد بودن به كارى كه ثمرۀ وى يا بيمارى بود يا ديوانگى‌؟ و من دانم كه بدين سخن تو دست از نبيد بازندارى و اين سخن گفتن نشنوى.

بارى تا بتوانى صبوحى عادت مكن و اگر به اتفاق صبوحى كنى به اوقات كن كه خردمندان صبوح را ناستوده داشته‌اند. و نخست شومىِ صبوح آنست كه نماز بامداد از تو فوت شود و ديگر هنوز بُخار دوشين از دماغ تو بيرون نشده باشد، بُخار امروزين با وى يار شود، ثمره‌ی وى جز ماليخوليا نباشد كه فساد دو مفسد بيش از فساد يك مفسد باشد. و ديگر به وقتى كه خلقان خفته باشند تو بيدار باشى و چون خلقان بيدار شوند ناچاره تو را ببايد خفت و چون همه روز بخسبى همه شب هر آينه بيدار باشى [و] روزِ ديگر همه اعضاى تو خسته و رنجه باشد از رنجِ نبيد و رنجِ بى‌خوابى. و كم صبوحى بُوَد كه در وى عربده نرود يا محالى كرده نيايد كه از آن پشيمانى خيزد يا خرجى به ناواجب كرده نيايد. اما اگر به اوقات گاهى صبوحى [كنى] به عذرى واضح روا بُوَد اما به عادت نبايد كردن كه آن عادت نامحمودست.

اگر چه به نبيد مولع باشى عادت كن كه به شب آدينه نبيد نخورى؛ هرچند شب آدينه و شب شنبه هر دو شب شراب حرام است اما شب آدينه را حرمتى است از بهر جمع فردايين. و نيز به يك شب آدينه كه نبيد نخورى يك هفته نبيد خوردن خويش بر دل خلقان خوش گردانى و زبان عامّه بر تو بسته شود و بدان جهان ثوابى بود و بدين جهان نيكونامى به حاصل آيد و اندر كدخدايى توفيرى بود و جسم و عقل و روح تو نيز بياسايد كه همه هفته عروق تو و دماغ تو از بخار پر شده باشد، اندر آن يك شب بياسايد و خالى شوند. و اندر آسودن اين يك‌شب هم صحّت و آرامش تن بود، و هم در مال توفير باشد، و هم بدان جهان ثواب باشد، و هم زفان عامه به خير بر تو گشاده گردد. پس عادتى كه چنين پنج فايده ازو حاصل شود آن عادت به كار بايد داشت كه ستوده بود.

\newpage
