\addcontentsline{toc}{section}{باب بيستم-
اندر كارزار كردن}
\section*{باب بيستم-
اندر كارزار كردن}

اما چون در كارزار باشى آنجا سستى و درنگ شرط نباشد چنانكه تا خصم تو بر تو شام خورد تو بر او چاشت خورده باشى. و چون در ميدان در كارزار افتى هيچ تقصير مكن و بر جان خويش مبخشاى كه آن را [كه] به گور بايد خفت به خانه نتواند خفتن، چنان بيتى كه من گويم به زفان\footnote{زبان} طبرى:

\begin{quote}
\centering
مى دشمن بشير تو دارى دمونهن \quad \quad هر اسمى ورى مبنو كهون وردرو نه \\
حنين كته دوناك بيس هر ردو نه \quad \quad بكورخته اين كسى نخسبى بخو نه

\end{quote}
و هم اين بيت را بپارسى به لفظ درى بگويم تا هر كسى را معلوم باشد

\begin{quote}
\centering
گر شير شود عدو، چه پيدا چه نهفت\quad \quad با شير به شمشير سخن خواهم گفت \\
كان را كه به گور خفت بايد بى‌جفت \quad \quad با جفت به خان خويش نتواند خفت
\end{quote}
و در معركه تا گامى پيش توانى نهادن هرگز گامى باز پس منه. و چون در ميان معركه و خصمان گرفتار آمدى از جنگ ميآساى كه از چنگ خصمان به جنگ توانى رستن تا در تو حركات روزبهى همى‌ بينند ايشان نيز از تو همى شكوهند\footnote{بترسند}. و اندران جاى مرگ بر دل خويش خوش كن و البته مترس (42 پ) و دلير باش كه شمشيرِ كوتاه به دست دليران دراز گردد. و به كوشش كردن تقصير مكن كه اگر هيچ گونه اندر تو ترسى و سست كارى پديد آيد اگر هزار جان دارى يكى نبرى، كمترْكسى\footnote{آدم کهتر و پایین‌تر} بر تو چيره گردد. آنگه يا كشته شوى يا نامت به بدنامى برآيد و چون به نامردى ميان مردمان معروف شوى [از نان برآيى و] در ميان همالان \footnote{هُمال یا هَمال: هم‌تا، همانند، شریک، دو چیز که در کنار هم به مناسبت قرار گیرند} خويش هميشه شرمسار باشى. و چون نان نباشد و نام نباشد، كم ارزى در ميان همالان حاصل شود و مرگ از آنْ زندگانى بِهْ\footnote{بهتر}؛ كه به نامِ نيكو مردن به، كه به ننگْ زندگانى‌كردن و زيستن.

اما به خون ناحق دلير مباش و خون هيچ مردم مسلمان به حلال مدار، الاّ خون صعلوكان\footnote{دزد، درویش، عیّار، فقیر} و دزدان و نبّاشان\footnote{کسی که قبرها را نبش می‌کند، کفن دزد} و خون كسى كه از روى شريعت قتل او واجب شود كه بلاى دو جهانى در خون ناحق بسته بود و پيوسته. اول آنكه در قيامت مكافات آن يابى. و اندرين جهان زشت‌نام گردى و هيچ كهتر بر تو ايمن نباشد و اميد خدمتگاران از تو بريده شود و خلق از تو نفور گردند و به دل دشمن تو شوند. و نه همه مكافات خون ناحق بدان جهان باشد كه من در كتاب‌ها خوانده‌ام و نيز تجربت كرده‌ام كه مكافات بدى هم بدين جهان به مردم رسد. پس اگر آن كس را طالعى نيك اوفتاده باشد ناچار به فرزندان او برسد. پس اللّه اللّه بر خويشتن و بر فرزندان خويشتن ببخشاى و خون ناحق مريز اما به خون [حق] يا خونى كه صلاح تو اندران باشد تقصير مكن كه آن تقصير فساد كار تو گردد چنانكه از جدّم شمس المعالى حكايت كنند:

حكايت بدان كه وى مردى سخت قتّال بود و گناه هيچ كس عفو نتوانستى كردن. مردى بد بود و از بدى او لشكر بر او كينه‌ور گشته بود. و با عمّ من فلك‌المعالى يكى شدند و بيآمدند [و] پدر خويش را شمس المعالى را بگرفت به ضرورت. از آنچه لشكر گفتند: اگر تو با ما يكى نباشى ما اين ملك به بيگانه‌اى دهيم. چون دانست كه ملك از خاندان او بيرون خواهد شدنه به ضرورت ثبات ملك را اين كار بكرد. و مقصود من آنست كه چون وى را بگرفتند، بند كردند و در مهدى نشاندند و بر وى موكلان كردند و به قلعه‌ی چناشك\footnote{دهی است از دهستان کوهسارات بخش مینودشت شهرستان گرگان و در گویش مازنی از روستاهای کوهسار فندرست استارآباد} فرستادند. در جمله‌ی موكلان وى مردى بود نام وى عبد اللّه جمازه‌بان، اندر راه همى رفتند. شمس المعالى اين مرد را گفت: يا عبد اللّه هيچ دانى كه اين كار كه كرد (43 ر) و اين تدبير چون بود كه بدين بزرگى شغلى برفت و من نتوانستم دانست‌؟ عبد اللّه گفت: اين كار فلان و فلان سپاه سلاّر كردند، و پنج كس را نام ببرد، لشكر را بفريبانيدند و در ميان اين شغل من بودم كه عبد اللّه‌ام و مردم را من سوگند دادم و اين كار را من بدين جاى رسانيدم. و لكن تو اين كار از من و ازين پنج كس مبين، از خويشتن بين كه تو را اين شغل از بسيار مردم كشتن افتاد. امير شمس المعالى گفت: تو غلطى مرا خود اين شغل از مردم ناكشتن اوفتاد كه اگر من تو را با اين پنج سپه‌سلاّر بكشتمى مرا اين كار نيفتادى، شش خون ديگر همى بايست كرد و به سلامت همى بودن.

و اين بدان گفتم تا بدآنچه ببايد كردن تقصير نكنى و آنچه نگزيرد سهل نگيرى. و نيز خادم كردن عادت مكن كه خادم كردن \footnote{بریدن بیضه} برابر خون كردن است؛ از بهر شهوت خويش نسل مسلمانى از جهان منقطع كنى، از اين بزرگ‌تر بيدادى نباشد. اگرت خادم بايد خود خادم كرده يابى كه مزه‌ی \footnote{بهره، نصیب} آن تو برگيرى و بزهِ آن به گردن ديگران بُوَد و تن خويش از گناه پاك داشته باشى.

اما در حديث كارزار كردن چنانكه گفتم چنان باش و خويشتن بخشاى مباش كه تا تن خويش را به خورد سگان نكنى نام خويش به نام شيران نتوان كرد. [و حقيقت بدان كه هر كه بزايد روزى بميرد كه جانور سه نوع است: حىّ ناطق، حى ناطق ميّت، [حى ميّت] يعنى فريشتگان و آدميان و وحوش و طيور. و در كتابى از آن پارسيان به خط پهلوى خواندم كه زردشت را پرسيدند هم برين گونه جواب داد، گفت: زياى گويا، زياى گويا ميرا، زياى ميرا، پس معلوم شد كه همه زنده‌اى بميرد، پس در كارزار اين اعتقاد بايد كردن و كوشا بودن تا نام و نان حاصل آيد. و در حديث مردن خود امير المؤمنين على بن ابى طالب «كرّم اللّه وجهه» گويد به لفظى موجز: «متّ اليوم الذى ولدت»، من آن روز مردم كه بزادم. و هر وقت از حديثى به حديثى مى‌روم و بسيار مي‌گويم اى پسر و لكن گفته‌اند كه بسياردان بسيارگوى باشد؛ اكنون آمدم باز بر سر سخن: [بدان] كه نام و نان از جهان به دست توان آوردن و چون به دست آوردى جهد آن كن كه جمع دارى و نگاه همى دارى و خرجى بر موجب دخل همى كنى و اسراف نكنى در كارها كه اسراف مبارك نبود. «و اللّه الهادى الى سواء السبيل.»


\newpage

