\addcontentsline{toc}{section}{باب سى و پنجم -
در آيين و رسم شاعرى}
\section*{باب سى و پنجم - 
در آيين و رسم شاعرى}

و اگر شاعر باشى، جهد كن تا سخن تو سهلِ ممتنع باشد. بپرهيز از سخن غامض\footnote{پیچیده} و چيزى كه تو دانى و ديگران به شرح را آن حاجت آيد، مگوى، كه شعر از بهرِ مردمان گويند، نه از بهرِ خويش. و به وزن و قافيه‌ی تهى قناعت مكن، بى‌صناعتى و ترتيبى شعر مگوى، كه شعرِ راست\footnote{سر راست} ناخوش بُوَد، علحى بايد كه بود اندر شعر و اندر زخمه\footnote{آهنگ، ریتم} و اندر صوت مردم تا خوش آيد، با صناعتى به رسم شعرا چون: مجانس و مطابق و متضاد و متشاكل و متشابه و مستعار و مكرّر و مردّف و مزدوج و موازنه و مضمر و مسلسل و مسجّع و ملوّن و مستوى و موّشح و موصّل و مقطّع و مخلّع و مسمّط و مستحيل و ذو قافيتين و نجر و مقلوب و مانند اين. امّا اگر خواهى كه سخن تو عالى نمايد بيشتر مستعار گوى و استعارت بر ممكنات گوى و اندر مدح استعارت به كار دارد. اگر غزل و ترانه گويى سهل و لطيف و تر گوى و به قوافى معروف گوى، تازي‌هاى سرد و غريب مگوى، حسب حال‌هاى عاشقانه و سخن‌هاى لطيف و امثال‌هاى خوش به كار دارد چنانكه خاص و عام را خوش آيد، تا شعر تو معروف گردد. وزن‌هاى گران عروضى مگوى كه گرد عروض و وزن‌هاى گران كسى گردد كه طبع ناخوش دارد و عاجز باشد از لفظ خوش و معنى ظريف اما اگر بخواهند بگويى روا بود. و لكن علم عروض نيك بدان و علم شاعرى و القاب و نقد شعر بياموز تا اگر ميان شاعران مناظره اوفتد يا با تو كسى مكاشفتى كند يا امتحانى كند عاجز نباشى و اين هفده بحر كه از دايرهاى عروض پارسيان برخيزد، نام دايرها و نام اين هفده بحر چون: هزج و رجز و رمل و هزج مكفوف و هزج اخر و رجز مطوى و رمل مخبون و منسرح و خفيف و مضارع و مضارع اخرب و مقتضب و مجتث و متقارب و سريع و قريب و قريب اخرب و آن پنجاه و سه عروض و هشتاد و دو ضرب كه درين هفده بحر بيايد جمله معلوم خويش كن. و آن سخن كه گويى اندر شعر، در مدح و غزل و هجا و مرثيت و زهد، داد آن سخن به تمامى بده و هرگز سخن ناتمام مگوى و سخنى كه در نثر نگويند تو اندر نظم مگوى كه نثر چون رعيت است و نظم چون پادشاه و آن چيز كه رعيت را نشايد، پادشاه را هم نشايد.

و غزل و ترانه تر و آب‌دار گوى و مدح قوى و دل‌گير، و بلندْهمت باش، سزاى هر كس بشناس و مدح چون گويى قدر ممدوح بدان، كسى را كه هرگز كاردى بر ميان نبسته باشد، مگوى كه تو به شمشير شير افگنى و به نيزه كوه بى‌ستون بردارى و به تير موى بشكافى، و آنكه هرگز بر چيزى ننشسته باشد اسب او را به دلدل و براق و رخش و شبديز\footnote{اسب سیاه رنگ خسروپرویز} ماننده مكن، بدان كه هر كسى را چه بايد گفتن. اما بر شاعر واجب است از طبع ممدوح آگاه بودن و بدانستن كه وى را چه خوش آيد، آنگه وى را چنان ستودن، كه وى خواهد كه تا آن نگويى كه خواهد تو را آن ندهد كه تو خواهى. و حقيرْهمت مباش، در هر قصيده خود را بنده و خادم مخوان، الاّ در مدحى كه ممدوح آن ارزد و هجا گفتن عادت مكن، كه هميشه سبوى از آب درست نيايد. اما بر زهد و توحيد اگر قادر باشى تقصير مكن كه به هر دو جهانت نيكويى رسد. و اندر شعر دروغ از حد مبر، هر چند دروغ در شعر هنرست، و مرثيت دوستان و محتشمان نيز واجب دار، اما غزل و مرثيت از يكى طريق گوى و هجا و مدح بر يك طريق، اگر هجا خواهى گفتن و ندانى همچنان كه كسى را در مدح ستايى، ضدّ آن مدح بگوى و هر چه ضدّ مدح بود هجا باشد و غزل و مرثيت همچنين. و هر چه گويى از جعبه‌ی خويش گوى، گرد سخنان مردمان مگرد كه آنگه طبع تو گشاده نشود و ميدان شعر تو فراخ نگردد و هم بدان درجه بمانى كه اول بوده باشى. بلى چون بر شاعرى قادر شدى و طبع تو گشاده و ماهر گشت اگر از جايى معنى غريب شنوى و تو را آن خوش آيد، خواهى كه برگيرى و ديگر جاى استعمال كنى مكابره مكن، به عينه هم آن لفظ را به كار مبر، اگر در مدحى معنى بود خود در هجوى به كار برو اگر در هجوى بود در مدحى به كار بر و اگر در غزل شنوى در مرثيتى به كار بر و اگر در مرثيتى شنوى در غزل به كار بر تا كسى نداند كه آن از كجاست. اگر ممدوح طلبى و اگر كار بازار كنى مدبر روى و پليد جامه مباش، دايم تازه‌روى باش و خنده‌ناك باش و حكايات و نوادر مسكته و مضحكه بسيار ياد گير كه در پيش مردم و پيش ممدوح ازين جنس شعرا را نگزيرد.

\newpage



















