  \addcontentsline{toc}{section}{باب هجدهم
اندر نخجير كردن}
\section*{باب هجدهم
اندر نخجير كردن}

بدان كه بر اسب نشستن و به نخجير رفتن و چوگان زدن، كارِ محتشمانست خاصه به جوانى. اما هر كارى بحدّ و اندازه بايد و با ترتيب. و همه روز پيوسته به نخجير مرو كه نه به ترتيب بود. هفته هفت. روز بود: دو روز به نخجير رو و روزى دو سه به شراب خوردن مشغول باش و روزى دو به كدخدايیِ خويش مشغول گرد.

اما چون برنشينى بر اسبِ كوچك منشين؛ كه مرد اگر چه منظرانى بود بر اسب كوچك حقير نمايد و اگر مردى حقير بُوَد بر اسب بزرگ بهتر نمايد. و بر اسب رهوار جز در سفر منشين كه چون اسبْ رهوار بُوَد مرد خويشتن را بر اسب افگنده دارد. اندر شهر و اندر ميان موكب، بر اسب تيز و جهنده نشين تا از سبب تندى وى از خويشتن غافل نباشى، مادام راست نشينى تا زشت‌ْركاب ننمايى. و به نخجيرگاه خيره‌ اسب متاز، كه بيهوده اسب تاختن كارِ كودكان و غلامان‌ِ سراى باشد. و از پسِ سباع\footnote{حیوان وحشی} اسب متاز، كه اندر نخجيرِ سباعْ هيچ فلاحى نَبُوَد و جز مخاطره كردن هيچ چيز حاصل نشود. و از اصل ما دو پادشاه بزرگ اندر نخجير سباع هلاك شدند: يكى جدِّ پدر من امير وشمگيربن‌زيار و يكى پسر عمِّ من امير شرف‌المعالى ؛پس بگذار تا كهتران تو بتازند و تو متاز مگر پيش پادشاهى بزرگ باشى، آنگه نام جستن و خويشتن نمودن را روا باشد. پس اگر نخجير دوست دارى، به نخجير يوز و باز و چرغ\footnote{پرنده‌ شکاری} و شاهين و سگ مشغول باش، تا هم نخجير كرده باشى و هم بيمِ مُخاطره نَبُوَد و آنچه بگيرى به كارى باز آيد كه نه گوشت سِباع خوردن را شايد و نه پوست او پوشيدن را.

پس اگر نخجير باز كنى،  پادشاهان از دو گونه كنند: ملوك خوراسان به دست خود باز نپرانند و ملوك عراق را رسميست كه بدست خود پرانند و هر دو گونه رواست. تو اگر پادشاه نباشى چنانكه اشتهاىِ تو باشد همى‌كن، پس اگر پادشاه باشى و خواهى كه به دست خود پرانى رواست. امّا هيچ بازى را بيش يك‌بار مپران، كه پادشاه را نشايد كه باز دو بار پراند، يك‌بار بپران و نظاره همى‌كن، اگر صيد گيرد و اگر نه، بازى ديگر بستان تا بازدار خود به طلب آن برود [كه مقصود پادشاه از نخجير بايد كه تماشا بود] نه طلبِ طعمه. اگر پادشاه به سگ نخجير گيرد پادشاه را مِجَر\footnote{قلاده} سگ نبايد گرفتن، بايد كه بندگان در پيش وى مى‌گشايند وى نظاره مى‌كند. امّا پَسِ نخجير اسب متاز. اگر نخجيرِ يوزكنى البته يوز بر کَفَلِ\footnote{ران} اسبِ خويش منشان، كه هم زشت بود تو را كارِ يوزْداران كردن و هم در شرطِ خرد نيست، سباعى را در پسِ قفاىِ خويش گرفتن، خاصه ملوك را، اينست شرط نخجير كردن.
























\newpage
