\addcontentsline{toc}{section}{ باب بیست و سوم - 
در برده خریدن}
\section*{باب بیست و سوم - 
در برده خریدن}

اگر بنده‌اى خرى، هشيار باش، كه آدمى خريدن علمى است دشوار و بسيار برده‌ی نيكو بُوَد كه چون به علم در وى نگرى به خلاف آن باشد. و بيشتر قوم گمان برند كه برده خريدن از جمله‌ی ديگر بازرگانی‌هاست و ندانند كه برده خريدن و علم آن از جمله‌ی فيلسوفيست، كه هر كس كه متاعى خَرَد و آن را نشناسد مغبون بُوَد و صعب‌تر آن شناختن آدميست، كه عيب و هنر آن بسيارست و يك عيب بود كه صد هنر بپوشاند، چنانكه يك هنر بُوَد كه صد عيب بپوشاند. و آدمى را نتوان شناختن، الاّ به علم فراست و تجربت، و تمامت علم فراست\footnote{چهره‌شناسی} علم نَبَويست، كه به كمال او هر كسى نرسد، الاّ پيغامبرى مرسل، كه به فراست بتواند دانستن نيك و بد و باطن مردم. امّا چندانكه شرطست از شراى\footnote{خریدن} مماليك\footnote{بنده،‌غلام} آنكه [مرا و غير مرا روا بود كه معلوم باشد روا بود كه] بگويم. بدان كه شرايط مماليك سه شرطست: يكى شناختِ عيب و هنرِ ظاهر و باطن ايشان از فراست و ديگر از علت‌هاى نهان و آشكارا آگاه شدن به علامت و سديگر دانستن جنس‌ها و عيب و هنر هر جنسى. امّا اول شرطى كه فراستى است آنست كه: چون بنده خرى نيك تأمل كن از آنكه بندگان را مشترى از دو گونه بود: كسى بود كه بر وى نگرد و به تن و اطراف ننگرد و كسى باشد كه بر وى ننگرد، به اطراف نگرد، نفيس و نعيم خواهد با شحم و لحم. اما هر كسى كه در بنده‌ی تو نگرد اول در روى نگرد آنگه قوايم وى نگرد، پس اولى‌تر كه خوب‌روى طلبى كه تو نيز روى او پيوسته همى بينى و تن او به اوقات بينى. پس اول در چشم و ابروى وى نگاه كن، آنگاه در بينى وى نگر، پس در لب و دندان، پس در موى وى نگر، كه خداى عزوجل همه آدميان را نيكويى در چشم و ابرو آفريد و ملاحت در بينى و حلاوت در لب و دندان و طراوت در پوست روى، و موى سر را مزين اين همه گردانيد، از بهر آنكه موى را از بهر زينت آفريد. پس چنان بايد كه اندرين چيزها نگاه كنى چون دو چشم [و] ابرو نيكو بود و در بينى ملاحت و در لب و دندان حلاوت و در پوست طراوت، بخر و به اطراف وى مشغول مباش. پس اگر اين همه نباشد بايد كه مليح باشد و به مذهب من مليح بى‌نيكويى به كه نيكويى بى‌ملاحت. و گفته‌ام كه بنده از بهر هر كارى بايد كه بدانى كه بر چه فراست بايد خريدن به علامت. اكنون اول علامتى كه بنده از بهر خلوت و معاشرت خرى، چنان بايد كه معتدل بود بدرازى و كوتاهى و فربهى و نزارى و سپيدى و صرخى و سطبرى و باريكى و درازى و كوتاهى گردن، به جعدى و ناجعدى موى در فام كفس گرد و نرم‌گوشت، تن او نرم و تنك‌پوست و هموار استخوان و مى گون موى و سياه مژه و شهلا چشم و سياه و گشاده ابرو و كشيده بينى و باريك ميان و مربع سرين بايد كه باشد و گرد زنخدان و صرخ لب و سپيد پوست بايد و هموار دندان و همه اعضاى او در خورد اين كه گفتم. هر غلامى كه چنين بود زيبا و خوش‌خو و وفادار بود و لطيف طبع و معاشر بود. و علامت غلام دانا و روزبه آنست كه راست قامت بود و معتدل موى و معتدل گوشت، سپيدى لعل فام و پهن كف و گشاده ميان انگشتان، پهن‌پيشانى، شهلاچشم، گشاده روى، بى‌خنده خندناك روى. (46 پ) و چنين غلام از بهر علم [آموختن] و كدخدايى فرمودن و خازنى و بهر شغل ثقه بود. و علامت غلامى كه ملاهى را شايد نرم‌گوشت بود و كم گوشت بود، خاصه بر پشت و باريك انگشتان، نه لاغر و نه فربه. و بپرهيز از انكه بر رخسارهاى او گوشت بسيار بود كه هيچ چيز نتواند آموختن، اما بايد كه نرم كف بود و گشاده ميان انگشتان، روشن‌چهره و تنك پوست، و مويش نه سخت دراز و نه سخت كوتاه و نه سخت سياه و نه سخت صرخ، شهلاچشم، زير پاى او همواره بود. اين چنين غلام هر پيشه كه دقيق بود زود آموزد خاصه خنياگرى\footnote{آواز خواندن}.

علامت غلامى كه سلاح را شايد، سطبرى موى بود و تمام بالا و راست قامت و قوى - تركيب و سخت‌گوشت و سطبر استخوان، و پوست اندام او درشت و سخت مفاصل و سطبر انگشت و پهن كف، فراخ سينه و كتف، و كشيده عروق و رگ و پى بر تن او پيدا و انگيخته، سطبر گردن، گرد سر و اگر اصلع بود به بود، و پهن‌شكم و برچده - سرين و عصبهاى ساق وى چون مى‌رود بر بالا كشد و درهم كشيده روى، ببايد كه سياه چشم بود، و هر غلامى كه چنين بود روزبه و مبارز و شجاع بود. علامت غلامى كه خادمى سراى زنان را شايد، سياه پوست و ترش روى و درشت پوست و خشك - اندام، تنك موى، باريك ساق، باريك بانگ، سطبر لب، پخچ بينى، كوتاه انگشت، منحدب قامت، باريك گردن، چنين [غلام] خادمى سراى زنان را شايد؛ اما نشايد كه سپيد پوست بود و سرخ گونه بود. و پرهيز كن از اشقر، خاصه فرو افتاده موى باشد و نشايد كه در چشمش رعونت بود و ترى بود كه چنين غلام با زن دوست بود يا قواده بود. علامت غلامى كه بى‌شرم و عوان \footnote{ظالم و سخت‌گیر}بود و ستوربانى را شايد، بايد كه گشاده و فراخ برو چشم بود، پلكهاى چشم وى سطبر بود و كوتاه، خاصه كه حدب‌گونه بود و سپيدى چشم او منقط\footnote{نقطه} بود بسرخى و اشقر بود و چشمش كبود و دراز لب و دندان و فراخ دهن بود، چنين غلام سخت بى‌شرم و ناباك بود و بى‌ادب و شرير و بلاجوى بود. علامت غلامى كه فراشى و طباخى را شايد، بايد كه پاك‌روى و پاك‌تن بود و گرد‌روى و باريك‌چشم و شهلا‌چشم كه بكبودى گرايد، و تمام قامت و خاموش و موى سر او فرو افتاده، چنين غلام اين كار را شايسته بود. امّا شرط كه گفتم كه از جنس چيز بايد دانستن و عيب و هنر هر يك بدانستن ياد كنم.

بدان كه غلام ترك [نه] يك جنس است و هر جنسى را طبعى و گوهرى ديگرست از جمله‌ی ايشان از همه بدخوتر قفچاقان\footnote{نژادی از ترکی} و غُز\footnote{نژادی ترک} بود. و از همه خوش‌خوتر و به عشرت فرمان بردارتر خُتَنى و خَلخى و نخشبى و تَبَتى بود، و از همه دليرتر و شجاع‌تر ترك قاى بود، و از همه بلاكش‌تر و رنجورتر و سازنده‌تر بُجناك بود و تاتارى و يغمائى، وز همه سست‌تر و كاهل‌تر چگلى. و به جمع معلوم كند كه از ترك نيكويى به تفسير و زشت بى‌تفسير نخيزد. و هندو به ضدّ اينست چنانكه چون در تركى نگاه كنى، سرى بزرگ بود و روى‌پهن و چشمهاى‌ِ تنگ و بينى پخچ و لب و دندان نه‌نيكو، چون يك‌يك را بنگرى به ذات خويش نه نيكو بود و لكن چون همه را به جمع بنگرى صورتى بود سخت نيكو، و صورت هندوان به خلاف اينست: چون يك‌يك را بنگرى، هر يكى به ذات خويش سخت نيكو نمايد و لكن چون به جمع درو نگرى چون صورت تركان ننمايد. اما ترك را ذاتى رطوبتى و صفاى صقالى\footnote{صیقل خورده} هست، كه هندوان را نيست، امّا به طراوت دست از همه جنسى ببرده‌اند. لاجرم از ترك هر چه خوب بود به غايت خوب بود [و آنچه زشت بود به غايت زشت بود] و بيشتر عيب ايشان آنست كه كند‌خاطر و نادان و مكابر و شغب‌ناك\footnote{فتنه انگیز} و ناراضى و بى‌انصاف و بد مست و بى‌بهانه آشوب‌[جوى] و بى‌زبان باشند و به شب سخت بددل باشند و آن شجاعت كه به روز توانند نمود به شب نتوانند نمود. اما هنر ايشان آنست كه شجاع باشند و بى‌ريا، ظاهر‌دشمن و متعصب به هر كارى كه به وِى سپارى نيك آمين، نرم‌اندام به عشرت و ز بهر تجمل به ازيشان جنسى نيست.

غلام سقلابى\footnote{روسیه} و روسى و آلانى قريب‌اند، به طبع تركان و لكن از تركان بُردْبارتر و كَدُود\footnote{رنج‌کش}تراند، اما آلانى به شب دليرتر از ترك بود و خداوند دوست‌تر بود. غلام گرجى به فعل به رومى نزديك بود لكن دريشان چند عيب است: يكى دزدى و بى‌فرمانى و نهان‌گزى و بى‌شكيبايى و كندكارى و سست‌طبعى و خداوند‌دشمنى و گريز پايى. اما هنرش آن بود كه نرم‌تن بود و مطبوع و گرم‌فهم و آهسته‌كار و درست‌زبان و دلير و راه‌بر و ياد‌گير. رومى عيبش آنست كه بد‌زبان و بد دل بود و سست طبع و كسلان و زود خشم و حريص و دنيا دوست بود و هنرش آنست كه خويشتن‌دار و مهربان و خوش‌خوى و كدخداى سر و روزبهى جوى و زبان‌نگه‌دار بود. غلام ارمنى، اما عيب ارمنى آن بود كه بد فعل و گنده‌تن و دزد و شوخگن و گريزنده و بى‌فرمان و بيهوده‌راى و خاين و دروغ‌زن و كفر‌دوست و بد‌دل و بى‌قوت و خداوند‌دشمن و سر‌تا‌پاى وِى به عيب نزديك‌تر كه به هنر، و لكن راست‌زبان و تيز‌فهم و كارآموز باشد. و عيبِ غلامِ هندو آن بُوَد كه بد‌زبان بود و در خانه، كنيزكان از وى ايمن نباشند. اما اجناسِ هندو، نه چون اجناسِ ديگر‌ْقوم باشند، از آنچه همه خلق با يكديگر آميخته‌اند، مگر هندوان، از روزگار آدم باز عادت ايشان چنانست كه هيچ پيشه‌ور جز با يك ديگر پيوند نكنند چنانكه: بقالان دختر ببقالان دهند و خواهند، و قصابان به قصابان، و خبازان به خبازان، و سپاهى به سپاهى، و برهمن به برهمن. پس درجه‌ی هر جنسى ازيشان طبعى ديگر دارند و من شرح هر يك نتوانم كرد كه كتاب از حال خويش بگردد. اما بهترينِ ايشان هم مهربان باشند و هم به خرد و هم شجاع، و بايد كه يا برهمن بود يا راوت\footnote{بهادر، مهتر هندی} يا كرار. برهمن عاقل بود و راوت شجاع بود و كرار كدخداى سر بود، برهمن دانشمند بود راوت سپاهى كار بود. اما نوبى و حبشى، بى‌عيب‌تر بود و حبشى از نوبى بِه بُوَد، كه در ستايشِ حبشى بسيار خبرست از پيغامبر صلى اللّه عليه و سلم.

اينست معرفت اجناس [و] هنر و عيب هر يك. اكنون شرطِ سوم آنست كه آگاه باشى از علتهاى ظاهر و باطن به علامت، و آن‌چنان بود كه در وقت خريدن غافل نباشى و به يك نظر راضى نباشى، كه به اول نظر بسيار خوب بود كه زشت نمايد و بسيار زشت بود كه خوب نمايد. و ديگر كه چهره‌ی آدمى پيوسته بر يك حال نماند، گه به خوبى و گه به زشتى همى گرايد. و نيك نگه‌كن در اندام او تا بر تو چيزى پوشيده نگردد. و بسيار علّت نهانى بود و علتى كه قصد آمدن كند و هنوز نيامده باشد تا چند روز خواهد آمدن، آن‌را علامتها بود، چنانكه اگر در گونه، لختى زرد‌ فامى بود نه در فام و رنگ لبش گشته بود و پژمرده چشم‌ها بود دليل بواسير كند.

و اگر پلك چشم‌ها دايم آماس دارد، دليل استسقا كند. و سرخىِ چشم و ممتلى\footnote{پر،‌آکنده} بودن رگهاى پيشانى، دليل صرع دموى\footnote{منسوب به دم، خون} كند. و دير جنبانيدن مژگان و لب خاييدن بسيار دليل ماليخوليا كند. و كژى استخوان بينى و ناهموارى بينى دليل ناسور\footnote{جراحت} كند و بواسير بينى. و موى سخت‌‌سياه و سطبر و خشن چنانكه جاى‌جاى سياه‌تر بود، دليل بود كه موى رنگ كرده باشد. و بر تن جاى‌[جاى] كه نه جاى داغ كردن بود، داغ بينى و يا وشم\foonote{خالکوبی} كرده، نگاه كن تا زير او برص\footnote{} نباشد و گشتن رنگِ لب و زردى چشم دليل يرقان بود. و غلامان را سِتان\footnote{بر پشت خوابیده} بخوابان و هر دو پهلوى او بمال و بنگر تا هيچ دردى و آماسى دارد، پس اگر دارد درد جگر و سپرز\footnote{طحال} باشد. چون ازين علتهاى نهانى تجسس كردى، از آشكارا نيز بجوى: از بوى دهن و بوى بينى و ناسور و گرانى گوش و سستى گفتار و تيزى و هموارىِ سخن و رفتن بر طريق و درستىِ مفاصل و سختىِ بن دندان‌ها، تا بر تو مخرقه\footnote{} نكنند.

آنگه كه اين كه گفتم بديده باشى و معلوم كرده باشى، آن بنده كه خرى از مردمان به‌صلاح خَر، تا در خانه‌ی تو هم به صلاح باشد. و تا عجمى يابى، پارسى گو مخر، كه عجم را به‌ خوىِ خود بر توانى آوردن اما پارسى گوى را نتوانى آوردن. و به وقتى كه شهوت بر تو غالب بُوَد، بنده را به عرض پيش خويش ميار، كه آن غلبه‌ی شهوت اندران وقت زشت را به چشمِ تو خوب گرداند. نخست، تسكين شهوت بكن، آنگه به خريدن آن مشغول شو. و آن بنده كه به جاى ديگر عزيز بوده باشد مخر، كه اگر چه او را عزيز دارى از تو منت آن ندارد كه خود را جاى ديگر همچنان ديده باشد، و اگر خوار و ذليل دارى يا بگريزد يا فروختن خواهد يا به دل دشمن شود. بنده را از جايى خر، كه اندران خانه بد داشته باشند تا به اندك‌مايه نيك‌داشت تو از تو سپاس‌دار بود و ترا دوست دارد. و هر چند گاهى از بندگان اندك‌مايه ببخشاى و مگذار كه پيوسته محتاج دِرَم باشند، كه آنگه به ضرورت به طلبِ دِرَمى‌ رَوَند. و بنده‌ی قيمتى خر، كه گوهرِ هر كسى باندازه‌ی قيمتش بود. و آن بنده كه خواجه‌ی بسيار داشته باشد مخر كه بنده‌یِ بسيارْخواجه و زنِ بسيارْشوى ستوده نيست، و آنچه خرى روزافزون خر. و چون بنده فروخت خواهد، ستيزه مكن و بفروش از آنكه زن كه طلاق خواهد و بنده كه فروخت خواهد از آن زن و از آن بنده [هيچ كس] شادمانه نباشد. و اگر بنده به عمد كاهلى كند و به قصد در خدمت تقصير كند نه به سهوى و خطائى، به ستم وى را روزبهى مياموز كه وى به هيچ حال جلد و روزبه نشود؛ زود بفروش كه خفته را ببانگى بيدار توان كرد و تن زده را به صد بانگ بوق و دهل بيدار نتوان كردن.

و عيال نابكار آينده گرد مكن، كه كم عيالى دَوْم توانگريست و خدمتگار چندان دار كه نگريزد، و آن را كه دارى به سزا نيكو دار، كه يك تن ساخته دارى بِهْ كه دو تن ناساخته. و مگذار كه در سراى تو بندگان برادر خواندگى گيرند و كنيزكان خواهر خواندگى كه تولد آن بزرگ بُوَد. و بر بندگان و آزادگان خويش بار به طاقت او بر نِه، تا از بى‌طاقتى بى‌فرمانى نكنند، و خود را به انصاف آراسته دار تا آراسته‌ی آراستگان باشى. و بنده بايد كه پدر و مادر و برادر و خواهر خويش خداوند خويش را داند. و بنده‌ی نخّاس\foonote{برده‌فروش} فرسوده مخر، بايد كه بنده از نخّاس چنان ترسد كه خَر از بِيْطار\footnote{دام‌پزشک}. بنده‌اى كه بهر وقت و بهر كارى فروخت خواهد وز خريدن و از فروختن خويش باك ندارد، دل بر وى منه، كه از وى فلاح نيايد، زود به ديگرى بدل كن، چنان طلب كه برين گونه بود كه گفتم.


\newpage
