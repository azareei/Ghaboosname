\addcontentsline{toc}{section}{باب نهم - در پیری و جوانی}
\section*{ باب نهم - در پیری و جوانی}

اى پسر! هر چند توانى پير‌ْعقل باش. نگويم كه جوانى مكن، لكن جوانى خويشتن‌دار باش. و از جوانانِ پژمرده مباش، كه جوانِ شاطر نيكو بُوَد؛ چنانكه ارسطاطاليس\footnote{ارسطو} همى‌گويد: حكمت «الشباب نوع من الجنون»؛ و نيز از جوانانِ جاهل مباش، كه از شاطرى بلا نخيزد و از جاهلى بلا خيزد. بهره‌ی خويش به حسب طاقت خويش از روزگار خويش بردار، كه چون پير شوى خود نتوانى؛ چنانكه آن پير گفت: چندين سال خيره غم خوردم كه چون پير شوم خوب‌رويان مرا نخواهند، اكنون كه پير شدم خودْ ايشان را نمى‌خواهم؛ و اگر توانى نيز خود نزيبد.

و هر چند جوان باشى خداى را عزّوجل فراموش مكن و از مرگ ايمن مباش، كه مرگ نه به پيرى بُوَد و نه به جوانى؛ چنانكه استاد حكيم عسجدى گويد:

\begin{quote}
\centering
مرگ به پيرى و جوانيستى \quad \quad 
پير بمردى و جوان زيستی
\end{quote}

و بدان كه هر كه زاد بميرد، چنانكه شنودم: حكايت كه به شهرِ مَرْوْ درزى‌اى\footnote{خیاطی} بود [و] بر در دروازه‌ی گورستان دكان داشت؛ و كوزه‌اى در ميخى آويخته بود و هوس آنش داشتى كه هر جنازه‌اى كه از آن شهر بيرون بردندى، وى سنگى اندر آن كوزه افگندى و هر ماهى حساب آن سنگ‌ها بكردى كه چند كس را بردند، و باز كوزه تهى كردى و سنگ همى درافگندى تا ماهى ديگر. تا روزگار برآمد از قضا درزى بمرد. مردى به طلب درزى آمد و خبرِ مرگِ درزى نداشت. دَرِ دوكانش بسته ديد؛ همسايه را پرسيد كه: اين درزى كجاست كه حاضر نيست‌؟ همسايه گفت: درزى نيز در كوزه افتاد.

اما اى پسر! هشيار باش و به جوانى غرّه مشو؛ اندر طاعت و معصيت به هر حالى كه باشى از خداىِ عزّوجل ياد همى كن و آمرزش همى خواه و از مرگ همى ترس، تا چون درزى ناگاه در كوزه نيفتى با بارِ گناهانِ بسيار؛ و همه نشست و خاست با جوانان مدار، با پيران نيز مجالست كن؛ و رفيقان و نديمان پير و جوان آميخته دار، تا جوانان اگر در مستیِ جوانى محالى كنند و گويند پيرانْ مانعِ آن محال شوند، از آنكه پيران چيزها دانند كه جوانان ندانند؛ اگر چه عادت جوانان چنانست كه بر پيران تماخره\footnote{هزل و تمسخر} كنند، از آنكه پيران را محتاجِ جوانى بينند و بدان سبب جوانان را نرسد كه بر پيران پيشى جويند و بى‌حرمتى كنند. ازيرا كه اگر پيران در آرزوىِ جوانى باشند، جوانان نيز بى‌شك در آرزوىِ پيرى باشند و پير اين آرزو يافته است و ثمره‌ی آن برداشته؛ جوان را بَتَر كه اين آرزو باشد كه دريابد و باشد كه درنيابد. و چون نيكو بنگرى، پير و جوان، هر دو حسودِ يك‌ديگر باشند؛ اگر چه جوان، خويشتن را داناترينِ همه كس داند؛ پس از طبعِ چنين جوانان مباش، پيران را حرمت دار و سخن با پيران به گزاف مگوى كه جواب پيران مسكت باشد. حكايت چنان شنودم كه پيرى صد ساله، گوژپشت، سخت دوتا گشته و بر عكازه‌اى تكيه كرده همى‌رفت. جوانى به تماخره وى را گفت: اى شيخ اين كمانك به چند خريده‌اى‌؟ تا من نيز يكى بخرم. پير گفت: اگر صبر كنى و عمر يابى خود رايگان يكى به تو بخشند، هر چند بپرهيزى. 

اما با پيرانِ نه‌بر‌جاى منشين، كه صحبت جوانان بر‌جاى بهتر، كه صحبت پيران نه‌بر‌جاى؛ تا جوانى، جوان باش؛ چون پير شدى، پيرى كن؛ چنانكه بيتى گفته‌ام:

\begin{quote}
\centering
گفتم كه در سراى زنجيرى كن \quad \quad 
با من بنشين و بر دلم ميرى كن\\
گفتا كه سپيدهات را قيرى كن \quad \quad
سردى چه كنى، پير شدى، پيرى كن
\end{quote}
كه در وقتِ پيرى، جوانى نزيبد؛ چنانكه جوانان را پيرى كردن نزيبد. پيرى كه جوانى كند، در هزيمت بوق زدن باشد\footnote{کنایه از کار غیر متعارف کردن؛ بوق سازی است که در بزم  و رزم استفاده می شود. بوق «سورنا» نایی است که در سور و جشن نواخته می شده و «کرنا» نایی بوده که در «کار» یعنی کارزار به کار می رفته است. بوق و کرنا را معمولا در زمان آماده باش سپاه و حمله می نواختند تا هم در دل دشمن ترس بیندازند و هم بر شجاعت سپاه خودی بیفزایند. طبعاً در شرایط عقب نشینی و شکست، زمینه ای برای نواختن بوق وجود ندارد و از همین جاست که مفهوم کنایی و طنزآلود «بوق در هزیمت زدن» به معنی کار غیرمتعارف کردن میباشد}؛ چنانكه من در زاهدى گويم:

\begin{quote}
\centering
چون بوق زدن باشد در وقت هزيمت \quad \quad 
مردى كه جوانى كند اندر گه پيرى
\end{quote}

و پيرِ رعنا مباش، كه گفته‌اند كه پيرِ رعنا بَتَر و بپرهيز از پيرانِ ناباك.
انصافِ پيرى بيش از آن بده كه انصافِ جوانى؛ كه جوانان را اوميد پيرى بُوَد و پيران را جُزْ به مرگ اوميد نباشد؛ و جز به مرگ اوميد داشتن وى محال باشد، از آنكه چون غلّه سپيد گشت، اگر نَدِرَوَند، خود بريزد، و همچنين ميوه كه پخته گشت، اگر نچينند، خود از درخت بيوفتد؛ چنانكه من گفته‌ام:


\begin{quote}
\centering
گر بر سر ماه بر نهى پايۀ تخت \quad \quad 
گر همچو سليمان شوى از دولت و بخت \\ 
چون عمر تو پخته گشت بر بندى رخت \quad \quad 
كان ميوه كه پخته شد بيفتد ز درخت
\end{quote}

پس نه بر باد گفته‌اند: 

\begin{quote}
\centering
اذا تمّ امر دنا نقصه\quad \quad  توقّع زوالا اذا قيل تمّ\footnote{ای چه میگه! انگا بیگی شعر (یا ضرب‌المثل) معروفی هست!}
\end{quote}

و چنان دان كه ترا نگذارند كه همى‌باشى؛ چون حواس‌هاى تو از كار بيفتد، دَرِ بينايى و دَرِ گويايى و دَرِ شنوايى و دَرِ بويايى و دَرِ لمس و ذوق همه بر تو بسته گردد؛ نه تو از زندگانیِ خويش شاد باشى، و نه مردم از زندگانى تو [و] بر مردمان وبالى گردى؛ پس مرگ از چنان زندگانى بِهْ. اما چون پير شدى، از محالِ جوانى دور باش، كه هر كه به مرگ نزديك‌تر بُوَد، بايد كه از محال جوانى دورتر بُوَد. مثالِ عمرِ مردمان چون آفتابست و آفتابِ جوانان در اُفقِ مشرق بود و آفتابِ پيران در اُفقِ مغرب؛ و آفتاب كه در اُفقِ مغرب بُوَد، فرو‌رفته دان، چنانكه من گفته‌ام:

\begin{quote}
\centering
كيكاوسى در كف پيرى شده عاجز  \quad \quad 
تدبير شدن كن تو كه شست و سه درآمد \\
روزت بنماز دگر آمد به همه حال \quad \quad 
 شب زود درآيد كه نماز دگر آمد
\end{quote}

و ازين هم نبايد، كه پير به عقل و فعل جوانان باشد؛ و بر پيران هميشه به رحمت باش، كه پير بيماريست كه كس به عيادت وى نرود، و پير[ى] علتى است كه هيچ طبيب داروى آن نداند الاّ مرگ؛ از آنچه پير از رنجِ پيرى نياسايد تا نميرد؛ و همه علتى كه به مردم رسد اگر نميرد اندر آن علت هر روز اوميد بهترى بود، مگر علت پيرى كه هر روز بَتَر بُوَد و اميد بهترى نبود. از آنكه در كتابى خواندم كه: مردم تا سى و چهار ساله، هر روز بر زيادت باشد بقوّت و تركيب. و پس از سى و چهار ساله تا به چهل سال همچنان بپايد، زيادت و نقصان نكند چنانكه آفتاب ميان آسمان رسد، بطى‌السير\footnote{کُنْدرو} بُوَد تا فروگشتن. و از چهل سالگى تا پنجاه سال، هر سالى در خويشتن نقصانى بيند كه پار نديده باشد. و از پنجاه سال تا به شصت سال، هر ماه در خويشتن نقصانى بيند كه در ماه ديگر نديده باشد. و از شصت سال تا هفتاد سال، هر هفته در خويشتن نقصانى بيند كه هفتۀ ديگر نديده باشد. وز هفتاد سال تا هشتاد سال هر روز در خود نقصانى بيند كه دى نديده باشد و اگر از هشتاد بر گذرد هر ساعتى دردى و رنجى بيند كه در ساعت ديگر نديده باشد. و حدّ عمر چهل سالست چون چهل سال تمام شد بر نردبان پايه ديگر راه نيست، همچنان كه بر رفتى فرود آيى، بى‌شك باز آن‌جاى بايَدَت بر رفتن كه فرود آمدى. پس بخشودنى كسى باشى كه در هر ساعت دردى و رنجى بدو رسد. پس يا ولدى و قرة‌العينى اين شكايت پيرى با تو دراز كردم، از آنكه مرا از وى سخت گله است و اين نه عجب كه پيرى دشمن است وز دشمن گله بُوَد چنان بيت كه من گويم:

\begin{quote}
\centering
اگر كنم گله از وى عجب مدار از من \quad \quad 
 كه وى بلاى منست و گله بود ز بلا
\end{quote}

و تو دوست‌تر كسى مرا و گله‌ی دشمنان با دوستان كنند. ارجو من اللّه، كه تو نيز اين گله با فرزند‌زادگانِ خويش كنى و اندرين معنى مرا دو بيت است:

\begin{quote}
\centering
آوخ گلۀ پيرى پيش كه كنم من‌؟ \quad \quad 
كين درد مرا دارو جز توبه دگر نيست \\
اى پير بيا تا گله هم با تو بگويم \quad \quad 
زيرا كه جوانان را زين حال خبر نيست
\end{quote}

از آنچه رنجِ پيرى كس از پيران بهتر نداند. حكايت چنانكه از جمله حاجبان\footnote{پرده دار. آنکه مردمان را باز دارد از درآمدن. چوبدار} پدرم حاجبى بود، وى را حاجب كامل گفتندى، پير بود و از هشتاد درگذشته. خواست كه اسبى خَرَد؛ رايضى\footnote{اسب‌دار}  اسبى بياورد فربه و نيكو‌رنگ و دُرُست‌قوايم. اسب را به بها پسنديد و به بها فرو داشت، چون دندانش بديد اسب پير بود، نخريد. من او را گفتم: فلان آن اسب را بخريد، تو چرا نخريدى‌؟ گفت: او مردى جوانست و از رنج پيرى خبر ندارد؛ اگر به رنگ و منظر اسب غرّه شود، معذورست؛ من از رنجِ پيرى و ضعف و آفتِ او خبر دارم، اسبِ پير خَرَم، معذور نباشم.

اما جهد كن تا به پيرى به يك‌جا مقام كنى، كه به پيرى سفر كردن، از خرد نيست، خاصه كه مردْ بى‌نوا باشد، كه پيرى دشمنست و بى‌نوايى دشمنست. پس با دو دشمن سفر كردن، نه از دانايى بود. اما اگر وقتى سفرى اوفتد به اضطرارى از خانه‌ی خويش بيفتى، اگر ايزد تعالى در غريبى بر تو رحمت كند و تُرا در سفر نيكويى پديد آرد، بيشتر از آنكه در حضر بوده باشد، هرگز آرزویِ خانه مكن و زادوبود مَطَلَب؛ هم آنجا كه نظام كارِ خويش بينى مقام كن، زادوبود آن‌جاى را شناس كه تُرا نيكويى بُوَد، هر چند كه گفته‌اند: «الوطن الأمّ الثانية». اما تو بدان مشغول مباش، رونقِ كارِ خويش بين، كه نيز گفته‌اند: نيك بختان را نيكى خويش آرزو كند و بدبختان را زادوبود. اما خود را چون رونقى ديدى و شغلى سودمند به دست آوردى، جهد آن كن كه آن شغل خويش را ثبات دهى و مستحكم گردانى. تا آن شغل نيابى، طلب بيشى مكن كه در طلب كردن بيشى به كمترى اوفتى چه گفته‌اند: «چيزى كه نيكو نهاده باشد، نيكوتر منه؛ تا به طمع محال از ان بَتَر نيابى».

اما اندر روزگار عمر گذرانيدن، بى‌ترتيب مباش؛ اگر خواهى كه به چشمِ دوست و دشمن با بَها باشى، بايد كه نهاد و درجه‌ی تو از مردمِ عامه پديد باشد؛ بر گزاف زندگانى مكن و ترتيب خويش نگاه‌دار.



\newpage




