\addcontentsline{toc}{section}{باب ششم-
در فزونى گهر از فزونى خرد و هنر}
\section*{باب ششم-
در فزونى گهر از فزونى خرد و هنر}

بدان اى پسر كه مردم بى‌هنر دايم بى‌سود بود چون مغيلان كه تن دارد و سايه ندارد، نه خود را سود كند و نه غير را، و مردم اصيل و نسيب اگر چه بى‌هنر باشد از روى اصل و نسب از حرمت داشتن مردم بى‌بهره نباشد؛ بَتَر آن باشد كه نه گهر دارد و نه هنر. اما جهد بايد كرد، تا اگر چه اصلى و گهرى باشى، تنْ‌گُهَر باشى؛ كه گوهرِ تَنُ از گوهرِ اصلْ بهترست؛ چنانكه گفته‌اند حكمت: «الشرف بالعقل و الادب لا بالأصل و النسب». يعنى بزرگى خرد و دانش راست نه گهر و تخمه را.

و بدان نامْ كه مادر و پدر نهد، همداستان مباش كه آن نام نشانى باشد؛ نام آن باشد كه تو به هُنرْ بر خويشتن نهى، تا از نام جعفر و زيد و عمرو و عثمان و على به استاد و فاضل و حكيم اوفتى كه اگر مردم را با گوهر اصل گوهر هنر نباشد صحبتِ هيچ كس را نشايد. و هر كرا در وى اين دو گوهر يابى چنگ در وى زن و از دست مگذار كه وى همه را بكار آرد. و بدان كه از همه هنرها بهترين [هنرى] سخن گفتن است كه آفريدگار ما جلّ‌جلاله از همه آفريدهاى خويش آدمى را بهتر آفريد و آدمى فزونى يافت بر ديگر جانوران، به ده دَرْجْ، كه [در] تن اوست: پنج درون و پنج بيرون؛ پنج نهانى چون: انديشه و ياد گرفتن و نگاه داشتن و تخيّل كردن و تمييز و گفتار؛ و پنج ظاهر چون: سمع و بصر و شمّ و لمس و ذوق\footnote{حس چشایی}. و ازين جمله آنچه ديگر جانوران را هست نه برين جمله است كه آدمى راست. پس بدين سبب آدمى پادشاه و كام‌گار شد بر ديگر جانوران. و چون اين بدانستى زبان را بخوبى و هنر آموخته كن و جز چرب زبانى عادت مكن كه زبان تو دايم همه آن گويد كه تو او را بر آن رانى و عادت كنى، چه گفته‌اند: هر كه را زبان خوش‌تر هوا خواهش بيشتر. و با همه هنر جهد كن، تا سخن بر جاى گويى؛ كه سخن نه بر جاى اگر چه خوب گويى زشت نمايد؛ و از سخن كارفزاى خاموشى گزين كه سخن بى‌سود همه زيان بود؛ و سخن كه از وِى بُوىِ هنر نيايد ناگفته بهتر؛ كه حكيمان سخن را مانندِ به نبيذ\footnote{شراب خرما، شراب انگور} كردند كه هم ازو خمار خيزد، و هم بدو درمان خمار بود. اما سخن ناپرسيده مگوى و از گفتار خيره پرهيز كن و چون باز پرسند جز راست مگوى. و تا نخواهندْ كس را نصيحت مگوى و پند مده؛ خاصه كسى را كه پند نشنود كه او خود اوفتد؛ و بر سر مَلَا هيچ كس را پند مده كه گفته‌اند حكمت: «النصح عند الملأ تقريع».

و اگر كسى به كژى برآمده باشد، گِرد راست كردن او مگرد، كه نتوانى چه هر درختى كه كژ برآمده باشد و شاخ زده به كژى و بالا گرفته، جز ببريدن و تراشيدن راست نگردد. و چنانكه بسخن خوب بُخْل نكنى اگر طاقتت بود، به عطاى مال هم بُخل مكن؛ كه مردم فريفتۀ مال زودتر شود، زانكه فريفتۀ سخن. و از جاى تهمت زده پرهيز كن و از يار بدانديش و بدآموز بگريز. و بخويشتن در غَلَط مشو، خود را جايى نه كه اگرت بجويند همانجا يابند تا شرمسار نگردى، و خود را از انجا طلب، كه نهاده باشى، تا بازيابى. و به غم مردمان شادى مكن، تا مردمان نيز بغم تو شادى نكنند. دادْ ده تا داد يابى؛ خوب گوى تا خوب شنوى. و اندر شورستان تخم مكار كه بر ندهد و رنج بيهوده بود يعنى كه با مردمان ناسپاس مردمى كردن چون تخم بود، كه بشورستان افگنى. امّا نيكى از سزاوارِ نيكى دريغ مدار، و نيكى آموز باش كه گفته‌اند: «الدال على الخير كفاعله».

و بِدان كه نيكى كن و نيكوى گوى دو برادرند كه پيوندشان زمانه نَگْسَلد و بر نيك كرده پشيمان مباش كه جزاى نيك و بد هم اندرين جهان به تو رسد، پيش از انكه به جاى ديگر روى؛ و چون تو با كسى خوبى كنى بنگر كه در وقت خوبى كردن هم چندان راحت كه بدان كس رسد، در دل تو خوشى و راحت پديد آيد؛ و اگر با كسى بدى كنى به چندان رنج كه بدو رسيده باشد بنگر تا بر دل تو چه ضَجْرَت\footnote{اندوه، ملال} و گرانى برسد، از تو خود بر كسى بَد نيايد چون حقيقت بى‌ضجرت تو رنج از تو به كس نرسد. و بى‌خوشىِ تو راحت از تو به كس نرسد. درست شد\footnote{چنین باشد} كه مكافات نيك و بد هم بدين جهان مى‌يابى پيش از آنكه بدان جهان رسى؛ و اين سخن را كه گفتم كس مُنکِر نتواند بود كه هر كه در همه عمر خويش با كسى نيكى يا بدى كرده است چون به حقيقت بينديشد داند كه من بدين سخن بر حقم و مرا بدين سخن مصدَّق دارد. پس تا بتوانى نيكى از كس دريغ مدار كه نيكى يك روز بَرْدهد.

% حكايت
% \HekaiatBegin
و چنين شنودم كه بدان روزگار كه متوكل خليفه بود به بغداد، وى را بنده‌اى بود فَتْح نام؛ سخت نجيب و روزبه بود، و همه هنرها و ادبها آموخته بود، و متوكل وى را به فرزندى پذيرفته بود و از فرزند عزيزتر داشتى. اين فتح را خواست كه شناوْ كردن بياموزد. ملاحان را آوردند و او را اندر دجله شناو مى‌آموختند. و اين فتح هنوز كودك بود و بر شناو كردن دلير نگشته بود و اما چنانكه عادت كودكانست از خود نمودى كه آموختم. يك روز تنها بى‌ اوستادان به شناو رفت و اندر آب جست و آب تيز همى آمد فتح را بگردانيد. فتح چون دانست كه با آب بسنده نيستَ، خود را با آب گذاشت و همى‌شد تا از ديدار مردمان ناپديد گشت. چون لختى راه رفته‌بود به آب، بر كنار رود سوراخهاى آب خورده بود تا به سوراخى برسيد آب‌خورده به روزگار، جهد كرد و دست بزد و خود را اندران سوراخ افگند و آنجا بنشست و گفت: تا خداى تبارك و تعالى چه خواهد؟ بدين وقت بارى جان بجهانيدم؛ و هفت روز آنجا بماند و اول روز كه خبر دادند متوكل را، كه فتح در آب جست و غرقه شد، از تخت فرود آمد و در خاك نشست و ملاحان را بخواند و گفت: هر كه فتح را مرده بيابد و بيارد هزار دينارش بدهم؛ و سوگند ياد كرد كه: تا آن وقت كه وى را بدان حال كه يابند نيارند و نبينمش، طعام نخورم. ملاحان در دجله اوفتادند، و غوطه همى‌خوردند، و هر جاى طلب همى‌كردند، تا سر هفت روز به اتفاق ملاحى بدين سوراخ رسيد، فتح را ديد، شاد گشت و گفت: همين‌جا بنشين تا سمارى \footnote{کَلَک} آرم؛ و پيش متوكل آمد و گفت: يا امير المؤمنين! اگر فتح را زنده بيارم مرا چه عطا بخشى‌؟ گفت: پنج هزار دينار بدهم. ملاح گفت: يافتمش زنده؛ سمارى ببردند و وى را بياوردند. متوكل آنچه ملاحان را پذيرفته بود، در وقت بفرمود دادن؛ و وزير را بفرمود كه: در خزينه رو و از هر چه در خزينۀ من چيزيست يك نيمه بدرويشان ده. آنگه گفت: نان و طعام آوريد كه وى گرسنۀ هفت روزه است. فتح گفت: يا امير المؤمنين من سيرم. متوكل گفت: مگر از آب دجله سيرى‌؟ فتح گفت: نه، من اين هفت روز گرسنه نبودم كه هر روز بيست تا نان بر طبقى نهاده، بر روى آب فرود آمدى، و من جهد - كردمى، و دو سه نان بگرفتمى؛ و زندگانى من از ان نان بود و بر هر نانى نبشته بود: محمد بن الحسين الاسكاف. متوكل فرمود كه: در شهر منادى كنيد كه آن مرد كه نان در دجله مى‌افگند كيست‌؟ بيايند و بگويند كه امير المؤمنين با او نيكويى خواهد كردن. روز ديگر مردى بيامد و گفت: منم آن‌ كس. متوكل گفت: بچه نشان‌؟ مرد گفت: بدان نشان كه نام من بر روى هر نانى نبشته بود: محمد بن الحسين الاسكاف. گفتند او را: اين نشان درست آمد اما چند گاهست تا تو اين نان در آب مى‌افگنى‌؟ مرد گفت: يك سالست. گفت: غرض تو ازين چه بوده است‌؟ گفت: شنوده بودم كه نيكى كن و به رود انداز كه روزى بر دهد. بدست من نيكى ديگر نبود؛ آنچه توانستم كردن همى كردم تا خود چه بردهد؟ متوكل گفت: آنچه شنيدى كردى و بدانچه كردى ثمرت يافتى؛ وى را بر در بغداد پنج ديه داد. مرد بر سر مُلْك رفت و محتشم گشت و هنوز فرزندزادگان آن مرد مانده‌اند در بغداد؛ و به روزگار القايم باللّه كه من حج كردم و ايزد تعالى مرا توفيق داد زيارت خانۀ خود، فرزندزادگان اين مرد را ديدم و اين سخن از پيران بغداد شنودم.
% \HekaiatEnd


پس تا بتوانى كردن از نيكى مياساى و خويشتن را به نيكويى و نيكوكارى به مردم نماى؛ و چون نمودى به خلاف نموده مباش، به زبان ديگر مگوى و بدل ديگر مباش تا گندم‌نماى جو فروش نباشى. و اندر همه كارى داد از خويشتن بده، كه هر كه داد از خويشتن بدهد از داور مستغنى باشد. و گر غم و شاديت بود، غم و  شاديت با آن كس بگوى كه او را تيمار و غم تو بود؛ و اثر غم و شادى پيش مردمان پيدا مكن، از بهر نيك و بد زود شاد و اندهگن مباش كه اين فعل كودكان باشد. بدان كوش كه بهر محالى از حال و نهاد خويش بنگردى كه بزرگان به هر حقى و باطلى از جاى نروند؛ و هر شادى كه بازگشت آن به غم است آن را به شادى مشمر. و بوقت نوميدى اميدوار باش و نوميدى را دَر اوميد بسته‌دان\footnote{همبسته. در پی هم} و اميد را دَر نوميدى؛ و حاصل همه كارهاى جهان بر گذشتن دان. و تا تو باشى حق را منكر مشو. و اگر كسى با تو بستيهيد به خاموشى آن ستوه او را بنشان؛ و جواب احمقان خاموشى دان. اما رنج هيچ كس ضايع مكن و همه كس را به سزا حق‌شناس باش، خاصه قرابت خويش را چندان كت\footnote{که تو را} طاقت باشد و با ايشان نيكويى كن و پيران قبيلۀ خويش را حرمت‌دار كه رسول گفته است عليه السلام: «الشيخ فى قبيلته كالنبىّ فى امته.» و لكن بديشان مولع\footnote{حریص} مباش تا همچنان كه هنر ايشان مى‌بينى عيب نيز بتوانى ديد. و اگر از بيگانه ناايمن شوى به مقدار ناايمنى زود خويشتن را ازو ايمن گردان و بر ناايمنى به گمان امن مباش كه زهر خوردن به گمان نه از دانايى بود. و به هنر و خرد مردمان نگاه همى كن؛ اگر از بى‌هنرى و بى خردى نان و نام بدست توانى آوردن پس بى‌هنر و بى‌خرد باش و اگر نه هنر آموز؛ و از آموختن و شنيدنِ سخنْ ننگ مدار تا از ننگ رسته باشى. و اندر نگر به عيب و هنر مردمان كه نفع و ضرّ ايشان از چيست‌؟ و سود و زيان ايشان تا كجاست‌؟ آنگه منفعت خويش از ميان بجوى، نه‌بينى كه چيزهاست كه مردم را به منفعت نزديك گرداند؟ و دور باش از آن چيزى كه مردم را به زيان نزديك گرداند. و تن خويش را بعث كن به فرهنگ و هنر آموختن، و اين تُرا بدو چيز حاصل شود: يا به كار بستن چيزى كه دانى يا بآموختن [آن چيز] كه ندانى، حكمت: و سقراط گويد: هيچ گنجى بهتر از هنر نيست، و هيچ دشمن بتر از خوى بد نيست، و هيچ عزّى بزرگوارتر از دانش نيست، و هيچ پيرايه بهتر از شرم نيست. پس آموختن را وقتى پيدا مكن، چه در هر وقت و در هر حال كه باشى چنان باش كه يك ساعت از تو در نگذرد تا دانشى نياموزى و اگر در آن وقت دانايى حاضر نباشد از نادانى بياموز كه دانش از نادان نيز ببايد آموخت، از انكه هر هنگام كه به چشم دل در نادان نگرى و بصارت عقل بر وى گمارى آنچه تُرا از وى ناپسنديده آيد دانى كه نبايد كرد؛ چنانكه اسكندر گفت، حكمت: من منفعت نه همه از دوستان يابم، بلكه از دشمنان نيز يابم؛ از انچه اگر [در] من فعلى زشت بُوَد دوستان بر موجب شفقت بپوشانند تا من ندانم و دشمن بر موجب دشمنى بگويد تا مرا معلوم شود؛ اين فعل بد را از خويشتن دور كنم پس آن منفعت از دشمن يافته باشم، نه از دوست؛ تو نيز آن دانش از نادان آموخته باشى نه از دانايان.

و بر مردم واجب است چه بر بزرگان و چه بر فروتران هنر و فرهنگ آموختن؛ كه فزونى بر هم‌سران\footnote{هم رده} خويش به فضل و هنر توان يافت؛ چون در خويشتن هنرى بينى كه در امثال خويش نبينى، هميشه خود را فزون‌تر ازيشان دانى و مردمان نيز تُرا فزون‌تر دانند از هم‌سران تو به قَدْرِ فَضْلْ و هُنرِ تو. و چون مردِ عاقل بيند كه وى را فزونى نهادند بر هم‌سرانِ وى، به فضلى و هنرى جهد كُنَد تا فاضل‌تر و بهره‌مندتر شود و هر آنگاه كه مردم چنين كند، بس دير بر نيايد تا بزرگوارتر هر كسى شود. و دانش جستن برترى جستن باشد بر هم‌سران و مانندان خويش؛ و دست بازداشتن از فضل و هنر نشانِ خرسندى  بُوَد بر فرومايگان؛ و آموختنِ هُنَر و تن را ماليده داشتن از كاهلى سخت سودمندست كه گفته‌اند: كاهلى فساد تن بود و اگر تن ترا فرمان بردارى نكند، نگر تا ستوه نشوى، ازيرا كه تنت از كاهلى و دوستى آسايش تُرا فرمان نبرد؛ از انكه تن ما را تحرّك طبيعى نيست و هر حركتى كه تن كند به فرمان كند نه به مراد، كه هرگز تا نخواهى و نفرمايى تن ترا آرزوى كار كردن نباشد؛ پس تو به ستمْ تنِ خويش را فرمان بردار گردان و به قهر او را به طاعت آور؛ كه هر كه تن خويش را مطيع خويش نتواند گردانيد، وى را از هنر بهره نباشد و چون تن خويش [را فرمان بردار خويش] كردى، به آموختنِ هنر سلامت دو جهانى اندر هنرش بيابى و سرمايۀ همه نيكي‌ها اندر دانش و ادبِ‌نفس و تواضع و پارسايى و راست‌گويى و پاك دينى و پاك شلوارى و بى‌آزارى و بردبارى و شرمگنى‌‌ شناس؛ اما به حدیثِ شرمگنى اگر چه گفته‌اند: «الحياء من الايمان»، بسيار جاى بُوَد، كه حيا بر مرد وَبال\footnote{سختی و عذاب} بود؛ و چنان شرمگن مباش، كه از شرمگنى در مهمّات خويش تقصير كنى و خلل در كار تو آيد؛ كه بسيار جاى بُوَد كه بى‌شرمى بايد كرد تا غرض حاصل شود. شرم از فحش و ناجوانمردى و ناحفاظى و دروغ زنى دار، از گفتار و كردار به اصلاح شرم مدار؛ كه بسيار مردم بود كه از شرمگنى از غرضهاى خويش باز ماند؛ همچنان كه شرمگينى نتيجۀ ايمانست؛ بى‌نوايى نتيجۀ شرمگينى است.

جاى شرم و جاى بى‌شرمى ببايد دانست و آنچه به صواب نزديك‌ترست همى بايد كرد؛ كه گفته‌اند: مقدمۀ نيكى شرمست و مقدمۀ بدى بى‌شرميست. اما نادان را مردم مدان و داناى بى‌هنر را دانا مشمر و پرهيزگار بى‌دانش را زاهد مدان. و با مردمِ نادان صحبت مكن خاصه با نادانى كه پندارد كه داناست. و بر جهل خرسند مشو و صحبت جز با مردم نيك‌نام مكن، كه از صحبتِ نيكانْ، مردْ نيك‌نام شَوَد؛ چنانكه روغنِ كُنْجيدْ از آميزش با گل و بنفشه [است]، كه به گل و بنفشه‌اش باز مى‌خوانند از اثر صحبت ايشان. و كردارِ نيك را ناسپاس مباش و فراموش مكن. و نيازمند خود را به سر باز مزن كه وى را رنجِ نيازمندى بَسْ است. خوش‌خويى و مردمى پيشه كن، وز خوي‌هاى ناستوده دور باش؛ و بى‌سپاس و زيان‌كار مباش كه ثمرۀ زيان‌كارى رنج‌مندى بود و ثمرۀ رنج نيازمندى بود و ثمرۀ نيازمندى فرومايگى؛ و جهد كن تا ستودۀ خَلْقان باشى، و نگر تا ستودۀ جاهلان نباشى كه ستودۀ عام نكوهيدۀ خاص بُوَد؛ چنانكه در حكايتى شنودم. حكايت گويند: روزى افلاطن نشسته بود، از جملۀ خاص آن شهر مردى به سلام او اندر آمد و بنشست و از هر نوع سخن همى گفت. در ميانۀ سخن گفت: اى حكيم امروز فلان مرد را ديدم كه سُخَنِ تو مى‌گفت و تُرا دعا و ثنا همى گفت و مى‌گفت: افلاطون بزرگوارْ مرديست كه هرگز كَسْ چُن او نبوده است و نباشد، خواستم كه شُكرِ او به تو رسانم. افلاطون چون اين سخن بشنيد، سر فروبرد و بگريست و سَختْ دل‌ْتَنگ شد. اين مرد گفت: اى حكيم از من چه رنج آمد تُرا كه چنين تنگ‌دل گشتى؟ افلاطون گفت: از تو مرا رنجى نرسيد و لكن مرا مصيبتى ازين بَتَر چه بود كه جاهلى مرا بستايد و كار من او را پسنديده آيد؟ ندانم كه كدام كارِ جاهلانه كردم كه به طبعِ او نزديك بود، كه او را آن خوش آمد و مرا بدان بستود؟! تا توبه كنم از آن كار و اين غم مرا از آنست كه مگر من هنوز جاهلم؛ كه ستودۀ جاهلان، جاهلان باشند. و هم درين معنى حكايت ديگر ياد آمد. حكايت چنين شنيدم كه محمد بن زكريا رازى رحمه اللّه مى‌آمد با قومى از شاگردانِ خويش. ديوانه‌اى پيشِ ايشان اوفتاد؛ در هيچْ كسْ ننگريست مگر در محمد بن زكريا و نيك درو نگاه كرد و در روى او بخنديد. محمد بن زكريا بازِ\footnote{به سمتِ} خانه آمد و مطبوخ اَفتيمون\footnote{نوعی دوا} بفرمود پختن و بخورد. شاگردان گفتند كه: چرا مطبوخ خوردى‌؟ گفت: از بهر آن خندۀ ديوانه، كه تا وى از جمله سوداى خويش جز وى با من نديد با من نخنديد؛ چه گفته‌اند: «كل طاير يطير مع شكله».


ديگر تندى و تيزى عادت مكن و ز حِلْم خالى مباش ولكن يك‌باره چنان مباش نرم كه از خوشى و نرمى بخورندت و نيز چنان درشت مباش كه هرگز به دست نپساوندت\footnote{پساویدن: دست مالیدن، لمس کردن}. و با همه گروه موافق باش كه به موافقت از دوست و دشمن مرادْ حاصل توان كرد. و هيچ كس را به بدى مياموز كه بد آموختن دُوُم بدى كردنست. و اگر چه بى گناه، كسى تُرا بيازارد تو جهد كن، تا تو او را نيازارى كه خانۀ كم‌آزارى در كوىِ مردميست؛ و اصلِ مردمى گفته‌اند كه كم آزاريست، پس اگر مردمى، كم‌آزار باش. ديگر، كردار با مردمانِ نيكو دار، از آنچه، مردم بايد كه در آينه نگرد، اگر ديدارشْ خوب بُوَد بايد كه كردارشْ چو ديدارش بُوَد، كه از نيكويى زشتى نه‌زيبد. نشايد كه از گندم جو رويد و از جو گندم، و اندرين معنى مرا دو بيت است:

\begin{quote}
\centering
ما را صنما همى بدى پيش آرى\quad \quad 
از ما تو چرا اميد نيكى دارى‌؟


رو جانا رو همى غلط پندارى \quad \quad
گندم نتوان درود چون جوكارى
\end{quote}

پس اگر در آينه نگرد، روى خويش زشت بيند هم بايد كه نيكى كند كه اگر زشتى كند زشتى بر زشتى فزوده باشد و بس ناخوش و زشت بُوَد دو زشتى به يكجا. و از ياران مشفق و آزموده نصيحت پذيرنده باش و با ناصحان خويش، هر وقت\footnote{همیشه}، به خلوت باش،ازيرا كه فايدۀ تو ازيشان به وقت خلوت باشد. و چنين سخن‌ها كه من ياد كردم بخوانى و بدانى و بر فضل خويش چيره گردى آنگاه به فضل و هنر خويش غرّه مباش و مپندار كه تو همه چيزى بدانستى؛ خويشتن را از جملۀ نادانان شُمُر، كه دانا آنگه باشى كه بر دانش خويش واقف گرد،ى چنانكه در حكايت شنودم: حكايت كه به روزگار خسرو اندر وقت وزارت بزرجمهر رسولى آمد از روم. خسرو بنشست ،چنانكه رسم ملوك عجم بود و رسول را بار داد. وى را با رسولْ بارْنامه همى بايست كند، به بزرجمهر يعنى كه مرا چنين وزيريست. پيش رسول با وزير گفت: اى فلان همه چيز در عالم تو دانى‌؟ بزرجمهر گفت: نه اى خدايگان. خسرو از آن طيره شد و ز رسول خجل گشت. پرسيد كه: همه چيز پس كه داند؟ بزرجمهر گفت: همه چيز همگان دانند و همگان هنوز از مادر نزاده‌اند. پس تو خويشتن را از جمع داناترينْ كس مدان كه چون خود را نادان دانستى، دانا گشتى؛ و سخت دانا كسى باشد كه بداند كه نادانست و عاجز، كه سقراط با بزرگى او همى‌گويد كه: اگر من نترسيدمى كه بعد از من بزرگان اهل خرد بر من عيب كنند و گويند: سقراط همه دانش جهان را به يكبار دعوى كرد، مطلق بگفتمى كه: هيچ چيز ندانم و عاجزم. و ليكن نتوانم گفتن كه اين از من دعوى بزرگ باشد. و بو شكور بلخى گويد و خويشتن را به دانش بزرگ در بيتى بستايد و آن بيت اينست:

\begin{quote}
\centering
تا بدانجا رسيد دانش من \quad \quad
كه بدانم همى كه نادانم
\end{quote}


پس به دانش خويش غره مشو، اگر چه دانا باشى. چون شغليت پيش آيد هرچند تُرا كفايت گزاردن آن باشد، پسندْراى خويش مباش؛ كه مرد پسندْراى خويش هميشه پشيمان بُوَد. و از مشورت كردن با پيران عار مدار، و با عاقلان و دوستان مُشْفِق مشورت كن، كه با حكمت و با نبوت و تأييد محمد مصطفى صلى اللّه عليه و سلم از پس از آنكه آموزگار وى و سازندۀ كار وى خداى عزّوجل بُوَد هم بدان رضا نداد و گفت سبحانه و تعالى: «وَ شاوِرْهُمْ فِي الْأَمْرِ» يا محمد، با ايشان، پسنديدگان و ياران خويش، مشورت كن، تدبير بر شما و نُصرَت بر من كه خدايم؛ و بدان كه راى دو كس، نه چون راى يك كس باشد چه يك چشم آن نتواند ديد كه دو چشم بيند، و يك دست آن نتواند برداشت كه دو دست بردارد. نبينى كه چون طبيبى بيمار بود و بيمارى بر وى دشوار بود، اعتماد بر معالجت خود نكند، طبيبى ديگر آرد و باستطلاع\footnote{طلب آگاهی کردن. اطلاع خواستن} راى او مداواى خويش كند و اگر چه سخت دانا طبيبى باشد. و اگر هم جنسى از آن تُرا شغلى اوفتد، ناچار از بهر او بكوش، رنج تن و مال دريغ مدار، اگر چه دشمن و حاسد تو باشد، كه اگر وى بدان اندر بداند فرياد رسيدن تو او را از ان محنت بيش بود و باشد كه آن دشمنى دوستى گردد. و مردمان سخن‌دان و سخن‌گوى كه به سلام تو آيند ايشان را حرمت دار و با ايشان احسان كن تا بر سلام تو حريص‌تر باشند؛ ناكس‌ترين كس آن بود كه بر وى سلام نكنند. و اگر چه با دانشى تمام باشى با مردمان سخن‌گوى فَدَم\footnote{کسی که از سخن گفتن عاجز است} مباش كه مردم دانا فدم نه نيكو باشد كه مردم اگر چه حكيم بُوَد، چون فدم بُوَد، حكمت وى به حكمت نماند و سخن وى رونقى ندارد. پس شرط سخن گفتن بدان كه چونست و چيست.




\newpage









