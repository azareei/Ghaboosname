\addcontentsline{toc}{section}{باب هفدهم
اندر خفتن و آسودن}
\section*{باب هفدهم
اندر خفتن و آسودن}

رسم روميان و حكيمان ايشان چنانست كه چون از گرمابه بيرون آيند تا زمانى در مَسلخِ گرمابه بِنَخُسْبد، بيرون نشود و لكن هيچ قوم را اين رسم نيست.

اما حكيمان خواب را موت‌ُالاصغر خوانند، از آنكه چه خفته و چه مرده كه هيچ دو را از عالم آگاهى نيست. و بسيار خفتن ناستوده است، تن را كاهل كند و طبع را شوريده كند و صورت روى را از حال به حالى برد. پنج چيزست كه در وقت چون به مردم رسد صورت روى تغيير كند: يكى نشاط ناگهان، و يكى غم مفاجا\footnote{ناگهانی}، و يكى خشم، و يكى خواب، و يكى مستى. و ششم او پيرى است كه چون مردم پير شود، از صورت خويش بگردد و آن خود نوعى ديگرست. اما مردم تا خفته بُوَد، نه در حكم زندگان  باشد چنانكه بر مرده قلم نيست، بر خفته هم قلم نيست چنانكه من گفته‌ام در دو بيتى: 


\begin{quote}
\centering
گر تو بجفا پشت مرا دادى خم \quad \quad 
من مهر تو در دلم نگردانم كم \\
از تو نبرم از آنكه اى شهره صنم \quad \quad 
تو خفته‌اى و به خفته بر نيست قلم
\end{quote}

اما همچنان كه خفتن بسيار زيان‌كارست، ناخفتن نيز هم زيان كارست؛ كه اگر آدمى را هفتاد و دو ساعت يعنى كه سه شبان‌ْروز به قصد نگذارند، كه بخسبد، پيوسته به ستم بيدار همى دارند، آن‌كس را بيمِ مرگِ فجا بود.

اما هر كارى را اندازه‌اى است، حكيمان چنين گفته‌اند كه: شبان‌روزى بيست و چهار ساعت [باشد چنان بايد كه ازين بيست و چهار ساعت]، دو بهر\footnote{بخش} بيدار باشى و بهرى خفته. هشت ساعت به طاعت خداى تعالى و به كدخدايى خود مشغول بايد بودن، و هشت ساعت به طيبت و عشرت و تازه داشتن روح خويش، و هشت ساعت ببايد آرميد تا اعضاها كه شانزده ساعت رنجه گشته باشد از حركات تكلفى آسوده باشد، كه جاهلان ازين بيست و چهار ساعت نيمى بخسبند و نيمى بيدار باشند، و كاهلان دو بهر بخسبند و بهرى بيدار باشند و بكار خويش مشغول باشند، و عاقلان بهرى بخسبند و دو بهر بيدار باشند؛ بدين قسمت كه ياد كرديم، هر هشت ساعتى از گونه‌ی ديگر. و بدان كه ايزد تعالى شب را از بهرِ خواب و آسايش بندگان آفريد و روز را از بهرِ معيشت و تحصيل اسباب معيشت آن چنانكه گفت: «وَ جَعَلْنَا اللَّيْلَ لِباساً وَ جَعَلْنَا النَّهارَ مَعاشاً». حقيقت همه زندگيست از جان و [تن و] كه تن مكانست و جان مُمْكَنْ. و سه خاصيت است جان را چون: زندگانى و حركات و سبكى و سه خاصيت تن راست چون: مرگ و سكون و گرانى. و تا تن و جان به يكجا باشند جان به خاصيت خويش تن را نگاه دارد: گاه اندر كارى آرد و گاه تن را به خاصيت خويش از كار باز دارد و اندر غفلت كشد.

هر گه كه تن خاصيت خويش پديد كند، مرگى و گرانى و سكونى فرو خسبد. و مَثَلِ فرو خفتنش چون خانه‌اى بود كه بيفتد، هر چه اندر خانه بود فرو گيرد. [پس تن كه فرو خُسْبَد همه ارواحْ مردم را فرو گيرد] كه نه  سمع بشنود و نه بَصَر بيند و نه ذوق چاشنى داند و نه لمس گرانى و سبكى و نرمى و درشتى داند. و نطق [و كتاب] خفتگان اندر مكانِ خويش باشد، پس ايشان را نيز فرو گيرد تا نه نطق گويد و نه كتاب نويسد. و حفظ و فكرت بيرونِ مكانِ خويش باشند، ايشان را فرو نتواند گرفتن، نبينى كه تن چون فرو خسبد، فكرت خواب همى‌بيند گوناگون و حفظْ ياد همى‌دارد، تا چون بيدار شود بگويد چه ديدى. اگر اين دو نيز اندر مكان خويش بودندى، هر دو را فرو گرفتى چنانكه نه فكرت توانستى ديد و نه حفظ توانستى ياد گرفتى. و اگر نطق و كتاب نه در مكان خويش بودندى، پس تو به خواب اندر نتوانستى رفتن و بخواب اندر گفتى و كردى. آنگاه [خود] خواب نبودى، چون مرد هميشه كُنا\footnote{ُکُن(کَردن) + ا} و گويا بودى، خواب نبودى و راحت و آسايش نبودى. و همه راحت جانوران در خوابست پس ايزد سبحانه و تعالى هيچ چيز بى‌حكمت نيافريد

اما خواب روز به تكلف از خويشتن دور بايد كرد، و اگر نتوانى اندك مايه بايد خفتن؛ كه روزِ خويش  شب گردانيدن نه از حكمت بود. اما رسمِ محتشمان و منعمان، چنانست كه تابستان نيم‌روز به قيلوله روند باشد، كه خسبند يا نه. اما آن طريقِ تنعم است، چنانكه در رسم است يك ساعت بياسايند و اگر نه با كسى كه وقت ايشان با وى خوش بود به خلوت همى‌باشند، تا آفتاب فرو گردد و گرما بشكند و آنگاه بيرون آيند. و در جملة‌ُالامر جَهْد بايد كرد، تا بيشترينِ عمر در بيدارى گذارى و در كمترْ خُفتنْ كه بسيار خفتن ما را خود پيش اندرست.

امّا به روز و به شب، هر گه كه به خواهى خفتن، تنها نبايد خفتن با كسى بايد خفت كه روح تو تازه دارد، زيرا كه خفته و مرده از قياس يكيست و هيچ دو را از عالم خبر نيست، لكن يكى خفته‌ی با جانست و يكى خفته‌ی بى‌حيات. اكنون فرقى كنيم ميان اين دو خفته. فرقْ آن كنيم، كه آن يكى را به ضرورت تنها همى‌بايد خفت، به عذر عاجزى، و اين خفته را كه اضطرار نيست چرا چنان خُسْبَد كه آن عاجز به اضطرار؛ پس مونسِ بسترِ اين جان‌فزاى بايد، كه مونس بسترِ آن چنانكه هست خود هست، تا خفتنِ زندگان از خفتن مردگان پيدا شود. و لكن پگاه خاستن عادت بايد كردن، چنان بايد كه پيش از آفتاب برخيزى كه وقت طلوع باشد، تو فريضه‌ی خداى عزّ‌و‌جل بگزارده باشى. و هر كسى كه با آفتاب برآمدن برخيزد تنگ‌روزى بود، از آن قبل كه نماز از وى درگذشته بود شومى وى او را دريابد. پس پگاه برخيز و فريضه‌ی خداى عزّوجل بگزار و آنگاه آغاز شغلهاى خويش كن. پس بامداد اگر شغليت نباشد و خواهى كه به نخجير و تماشا روى روا باشد، كه بدان مشغول باشى.



\newpage














































