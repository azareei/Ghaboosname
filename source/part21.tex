\addcontentsline{toc}{section}{باب بيست و يكم - 
در جمع كردن مال}
\section*{باب بيست و يكم - 
در جمع كردن مال}

اى پسر! خويشتن را از فراز آوردن مال غافل مدار و لكن از بهر چيز مخاطره مكن. و جهد كن تا هر چه فراز آورى از نيكوترين روى بود تا بر تو گوارنده بود. و چون فراز آوردى، نگاه‌دار و به هر باطل از دست برمده كه نگاه داشتن سخت‌تر از فراز آوردنست. و چون به هنگام دَرْبايِستى خرج كنى، جهد كن تا عوضِ آن زود باز جاى نهى كه چون براى خرج همى بردارى و عوضِ آن باز جاى ننهى، اگر گنج قارون بُوَد، هم روزى اِسْپَرى\footnote{سِپَری} شود. و نيز چندانى دل در چيز مبند، كه وى را ابدى شناسى؛ تا اگر وقتى اسپرى شود سخت دل‌تنگ نباشى. و اگرچه چيز بسيار بود، تو به تقدير و به تدبير به كار بر؛ كه اندكى به تقدير و تدبير بهتر كه بسيارى بى‌تقدير و تدبير. و اگر بسيارى از تو بماند دوست‌تر دارم كه باندكى نيازت اوفتد، چه گفته‌اند كه چيزى كه به دشمنان ماند، به كه از دوستان خواهى. و سخت داشتن به از سخت جستن، و اگر چه كم مايه چيز بود نگاه داشتن واجب دان كه هر كه اندك‌مايه نداند، داشتن بسيار هم نداند داشتن. كار خويش بِهْ دان، كه كار كسان، وز كاهلى ننگ دار كه كاهلى شاگرد بدبختى است. رنج‌بُردار باش ،از انچه چيز از رنج گرد شود نه از كاهلى و چنانكه از رنج فراز آيد از كاهلى برود كه حكيمان گفته‌اند: كوشا باشيد تا آبادان باشيد و خرسند باشيد تا توانگر باشيد و فروتن باشيد تا بسيارْدوست باشيد. پس آنچه از رنج و جهد بدست آيد از كاهلى و غفلت از دست بدادن نه خوب بود و نه از خرد بود، كه هنگام نياز پشيمانى سود ندارد. و لكن چون رنج خود برى، كوش كه بَرْ\footnote{ثمر} هم تو خورى. و اگر چه چيز عزيزست از سزاوار دريغ مدار كه بهمه حال كس چيز به گور نبرد. اما خرج به اندازه‌ی دخل كن، تا نياز در تو راه نيابد كه نياز همه نه در خانه‌ی درويشان باشد، بل كه نياز اندر خانه‌اى بود كه دخل درمى بود و خرج درمى و حبّه‌اى، هرگز آن خانه بى‌نياز نبود. و بى‌نيازى در ان خانه بود كه درمى دخل بود و درمى كم حبّه‌اى خرج، هر كرا خرج از دخل كمتر بود هرگز خلل در خانه‌ی او راه نيابد. و بدآنچه دارى قانع باش، كه قانعى دَوْمْ بى‌نيازيست، كه هر آن روزى كه قسمت تُو‌ست آن خود بى‌گمان به تو رسد. و هر كارى كه آن به سخن نيكو يا به شفاعت مردمان نيكو گردد چيز برآن كار بذل مكن، كه مردم بى‌چيز را هيچ قدر نبود. بدان كه مردمان عامه همه توانگران را دوست دارند بى‌نفعى؛ و همه درويشان را دشمن دارند بى‌ضررى كه بترين حال مردم نيازمنديست. 

بدان كه هر خصلتى كه آن ستايش توانگرانست، هم آن خصلت نكوهش درويشانست. و آرايش مردم در چيز دادن بين و قدر هر كس بر مقدار آرايش آن كس شناس، اما اسراف را شوم دان و هر چه خداى تعالى آن را دشمن دارد آن بر بندگان خداى تعالى شوم بود و خداى تعالى همى‌گويد: «وَ لا تُسْرِفُوا إِنَّهُ لا يُحِبُّ الْمُسْرِفِينَ » چيزى كه خداى تعالى آن را دوست ندارد تو نيز مدار. هر آفتى را سببى هست، سبب درويشى اسراف دان و نه همه اسراف خرج نفقات بود كه در خوردن و در گفتن و در كردن و در همه شغل بُوَد، در جمله‌ی كارها اسراف مذموم است از انچه اسراف تن را بكاهد و نفس را برنجاند و عقل را بِرَماند و زنده را بميراند. نه‌بينى كه زندگانیِ چراغ از روغن است، امّا اگر بى‌حدّ‌و‌اندازه روغن اندر چراغ‌دان افگنى، چنانكه از نوك چراغ‌دان بيرون آيد و بر سر فتيله بيرون گُذَرد، بى‌شك چراغ بميرد؛ همان روغن كه از اعتدال سبب حيوة\footnote{حیات} او بود، از اسراف سببِ ممات او بود. پس معلوم شد كه تنها نه از روغن زنده بُوَد، بلكه از اعتدال روغن زنده بود و چون از اعتدال بگذرد، اسراف پديد آيد؛ هم بدان روغن كه زنده بدان بود بميرد. و خداى عزّوجل اسراف را بدين [سبب] دشمن دارد و حكما نپسنديده‌اند اسراف كردن در هيچ‌كار كه عاقبتِ مسرفى همه زيان است. امّا زندگانىِ خويش نيز تلخ مدار و درِ روزى بر تنِ خويش مبند و خود را به تقدير نيكو دار و آنچه دربايست بود تقصير مكن، بر خويشتن هزينه كن كه چيز اگر چه عزيزترست، آخر از جان عزيزتر نيست. در جملة‌الامر جهد كن تا آنچه فراز آرى به صلاح بكار برى و چيزِ خويش جز به دست بخيلان مسپار، بر مُقامِر و سيكى‌خوار هيچ چيز استوار مدار و همه كس را دزد پندار تا چيز تو از دزد ايمن بود. و در جمع كردن چيز تقصير مكن كه هر كه در كار خويش تقصير كند از سعادت هيچ توفيرى نيابد و از غرض‌ها بى‌بهره ماند، ازيرا كه تن‌آسانى در رنج است و رنج اندر تن‌آسانى؛ چنانكه آسودنِ امروز رنج فردايين است و رنجِ امروزين آسايش فردايين. و هر چه از رنج و بى‌رنج به دست آيد، زود جهد آن كن كه از هر درمى دو دانگ به نفقات خويش و آن عيالان خويش به كار برى، اگر چه دربايست بود و محتاج باشى، بيش ازين به كار مبر.

چون [دو] دانگ ازين روى به كار برى، دو دانگ ذخيره بنه از بهر روز ضرورت را و پشت به روى كن و بهر خللى از وى ياد ميار؛ يا بگذار از بهر وارثان خود را؛ يا روز ضعيفى و پيرى را تا فريادرسِ تو باشد.  و آن دو دانگ ديگر كه باقى بماند به تجمل خويش كن و تجمل آن كن كه نميرد و كهن نشود چون جواهر و سيمينه و زرينه و برنجينه و مسينه و رويينه و آنچه بدين ماند، پس اگر بيشتر چيزى بود به خاك بده كه هر چه به خاك بدهى هم از خاك باز يابى و مايه دايم بر‌جاى بود و سود حلال روان. و چون تجمّل ساختى به هر ضرورتى و دربايستى كه ترا بود تجملى از خانه مفروش و مگوى كه: اى مرد اكنون ضرورتست بفروشم و وقتى ديگر كه به ازين به كار آيد باز خرم، كه اگر بِهَر خللى چيزى از تجمل خانه بفروشى، به اوميد عوض باز‌خريدن، عوض باز خريده نيايد و آن خود از دست تو بشود و خانه تهى گردد. پس روزگارى بر نيايد كه تا تو مُفلس گردى. و نيز به هر ضرورتى كه تو را بود وام مكن و چيز خويش به گرو منه و البته زر به سود مستان و وام خواستن ذليلى و كم‌آزرمى بود.

و تو تا بتوانى كس را وام مده، خاصه دوستان را كه از باز‌خواستن آزار بزرگتر از آن بُوَد كه از نادادن. پس اگر بدادى درم اوام داده را از خواسته‌ی خويش مشمر و اندر دل چنان دان كه اين درم بدين دوست خويش بخشيدم تا وى باز ندهد از وى طلب مكن تا به استقضا دوستى منقطع نشود كه دوست را زود دشمن توان كرد، اما دشمن را دوست گردانيدن مشكل بود، كه آن كار كودكانست و اين كار پيران عاقل بود. و از هر چيزى كه تو را بود مستحق را بهره كن و از چيز مردمان طمع مدار كه تا بهترين همه مردمان باشى و چيز خويش را از آن خويش دان و چيز ديگران از آن ديگران تا به امانت معروف‌ شوى و مردم را بر تو اعتماد افتد و ازين قبل هميشه توانگر باشى «و اللّه اعلم».




















\newpage


