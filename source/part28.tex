\addcontentsline{toc}{section}{باب بيست هشتم -  
در آيين دوست گرفتن}
\section*{باب بيست هشتم -  
در آيين دوست گرفتن}

بدان اى پسر، كه مردمان تا زنده باشند ناگزير باشد از دوستان؛ كه مرد اگر بى‌برادر باشد بِهْ كه بى‌دوست؛ از آنچه حكيمى را پرسيدند كه: دوست بهتر يا برادر؟ گفت: برادر هم دوست بِهْ\footnote{برادر هم بهتر است دوست باشد.}. پس انديشه كن به كار دوستان بتازه داشتن رسم هديه فرستادن و مردمى كردن، ازيرا كه هر كه از دوستان نیانديشد، دوستان نيز ازو نیانديشند. پس مرد همواره بى‌دوست بود و ايدون\footnote{این‌چنین} گويند كه: دوست گویند دست‌باز \footnote{کسی را که هرچه در دست داشته باشد، همه را ببازد و تمام کند. کنایه از جوانمردی و سخاوتمندی} دارنده‌ي خويش بود\footnote{دوست جوانمرد و سخی، حافظ دوستان خویش است.}. و عادت كن كه هر وقت دوستى گرفتن، ازيرا كه با دوستان بسيار عيب‌هاى مردم پوشيده شود و هنرها گستريده گردد. و لكن چون دوستِ نو گيرى، پشت با دوستان كهن مكن، دوست نو همى‌طلب و دوست كهن را برجاى همى دار، تا هميشه بسيارْدوست\footnote{کسی که دوست بسیار دارد} باشى كه گفته‌اند: دوستِ نيك، گنجى بزرگ است؛ ديگر انديشه كن كه از مردمانى كه با تو به راه دوستى روند و نيم‌دوست\footnote{دوست معمولی} باشند با ايشان نيكويى و سازگارى كن و به هر نيك و بد با ايشان متفق باش، تا چون از تو همه مردمى بينند، دوست يكدل شوند؛ كه اسكندر را پرسيدند كه: بدين كم‌مايه روزگار اين چندين ملك به چه خصلت بدست آوردى‌؟ گفت كه: به دست آوردن دشمنان به تلطّف و به جمع كردن دوستان به تعهّد. و آنگه انديشه كن از دوستانِ دوستان، كه دوستانِ دوستان هم از جمله‌ی دوستان باشند. و بترس از دوستى كه دشمن تو را دوست دارد، كه باشد كه دوستى او از دوستى تو بيشتر باشد پس باك ندارد از دشمنى با تو كردن از قبل دشمن تو. و بپرهيز از دوستى كه مر دوست ترا دشمن دارد و دوستى كه از تو بى بهانه و بى‌حجتى به گله شود نيز به دوستى وى طمع مكن.

و اندر جهان، بى‌عيب كس مشناس؛ اما تو هنرمند باش كه هنرمند كم‌عيب بود و دوست بى‌هنر مدار كه از دوست بى‌هنر فلاح نيايد. و دوستان قدح\footnote{دوستان هم‌پیاله} را از جمله‌یِ نديمان شمار، نه از جمله‌ی دوستان؛ كه ايشان دوستانِ دَمْ و قدح باشند، نه دوستان غم و فَرَح. و بنگر ميان نيكان و بدان و با هر دو گروه دوستى كن، با نيكان به دل دوست باش و با بدان به زفان\footnote{زبان} دوستى نماى، تا دوستى هر دو گروه تو را حاصل گردد. و نه همه حاجتى به نيكان افتد، وقتى باشد كه به دوستى بدان حاجت آيد، به ضرورت كه از دوستِ نيك مقصود برنيايد، اگر چه راه بردن تو نزديك بدان، به نزديك نيكان تو را كاستى درآيد\footnote{قدر و ارج تو را کم میکند. کم‌بها میشوی}، چنانكه راه بردن تو به نيكان، نزديك بدان آب‌روى فزايد، و تو طريق نيكان نگه‌دار كه دوستى هر دو قوم تو را حاصل گردد. اما با بى‌خردان هرگز دوستى مكن كه دوستِ بى‌خرد از دشمن بخرد\footnote{خرددار} بتر\footnote{بدتر} بود، كه دوست بى‌خرد، با دوست از بدى آن كند كه صد دشمن باخِرَدْ با دشمن نكند. و دوستى با مردم هنرى و نيك‌عهد و نيك‌محضر دار، تا تو نيز بدان هنرها معروف و ستوده شوى كه آن دوستان تو بدان معروف و ستوده باشند. و تنهايى دوست‌تر دار از هم‌نشين بد، چنانكه من گويم:


\begin{quote}
اى دل رفتى چنانكه در صحرا دد\footnote{حیوان درنده} \quad \quad نه انده من خورى و نه انده خود \\
هم‌جالس\footnote{هم‌نشین} بد بودى تو رفته بهى\footnote{بهتر که رفته باشی} \quad \quad تنهايى بِهْ بسى ز هم جالس بد
\end{quote}


و حق مردمان و دوستان به نزديك خويش ضايع مكن، تا سزاوار ملامت نگردى كه گفته‌اند: دو گروه مردم سزاوار ملامت باشند: يكى ضايع كننده‌ی حق دوستان و ديگر ناشناسنده‌ی كردار نيكو. بدان كه مردم را به دو چيز بتوان دانست كه دوستى را شايد يا نه\footnote{شایسته دوستی هستند یا نه}: يكى آنكه دوست او را تنگ‌دستى رسد چيزِ\footnote{مال و ثروت} خويش ازو دريغ ندارد، به حسبِ طاقتِ خويش و به وقت تنگى\foonote{هنگام فقر و تنگ‌دستی} از وى برنگردد، تا آن وقت كه با دوستى وى ازين جهان بيرون شود، او فرزندان آن دوست را و خويشاوندان و دوستان آن دوست را طلب كند و به جاى ايشان نيكى كند. و هر وقت به زيارت تربت آن دوست رود و حسرتى بخورد هر چند آن نه تربت آن دوست او بود، چنانكه سقراط را شنيدم كه همى‌بردند تا بكشندش كه وى را الحاح\footnote{اصرار، پافشاری} كردند كه: بت‌پرست شو، وى گفت: معاذ اللّه\footnote{به‌ خدا پناه میبرم} كه من صُنْعِ\footnote{آفرینش} صانعِ\footnote{آفریننده} خويش را پرستم، ببردندش تا بكشند. قومى شاگردان با وى همى‌رفتند و زارى همى‌كردند چنانكه رسم باشد. پس وى را پرسيدند كه: اى حكيم، اكنون دل خويش به كشتن نهادى بگوى تا تو را كجا دفن كنيم‌؟ سقراط تبسّم كرد و گفت: اگر چنان باشد كه مرا باز يابيد، هر كجا كه شما را بايد دفن كنيد، يعنى كه آن نه [من] باشم چه قالب\footnote{جسم} من باشد.

و با مردمان دوستى ميانه دار\footnote{در دوستی میانه‌رو باش}، بر دوستان به اميد دل مبند، كه من دوستان بسيار دارم، دوست خاصه‌ی خويش خود باش، و از پيش‌وپس خويشتن خود نگر و بر اعتماد دوستان از خويشتن غافل مباش، چه اگر هزار دوست باشد تو را از تو دوست‌تر تورا كس نبود. و دوست را به فراخى و تنگى آزماى، به فراخى حرمت و به تنگى سود و زيان. و دوستى كه دشمن تو را دشمن ندارد، وى را جز آشناىِ خويش مخوان، چه آن كس آشنا بُوَدْ نه دوست. 

و با دوستان در وقت گله همچنان باش كه در وقت خشنودى و بر جمله دوست آن را دان كه تو را دوست دارد. و دوست را به دوستى چيزى مياموز، كه اگر وقتى دشمن شود، تو را آن زيان دارد و پشيمانى سود نكند. و اگر درويش باشى، دوست توانگر طلب مكن، كه درويش [را] خود كس دوست نباشد، خاصه توانگران. و دوست بدرجه‌ی خويش گزين و اگر توانگر باشى و دوست درويش دارى روا باشد. اما در دوستى مردمان دل استوار دار، تا كارهاى تو استوار بود و لكن دوستى بى‌جُرم دل از تو بردارد، به باز آوردن او مشغول مباش و نيز از دوست طامع دور باش كه دوستى او با تو به طمع باشد نه به حقيقت. و با مردم حقود\footnote{کینه‌توز} هرگز دوستى مدار كه مردم حقود دوستى را نشايد، از آنكه حقد هرگز از دل حقود بنشود\footnote{بیرون نرود} چون هميشه آزرده و كينه‌ور باشد دوستى تو اندر دل وى محكم نباشد و بر وى اعتماد نبود. و چون حال دوست گرفتن بدانستى آگاه شو از حال و كار دشمن، انديشه كن درين معنى.





\newpage














