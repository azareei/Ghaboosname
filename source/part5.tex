\addcontentsline{toc}{section}{باب پنجم - در شناختنِ حقِ مادر و پدر}
\section*{باب پنجم - در شناختنِ حقِ مادر و پدر}

و بدان اى پسر! كه آفريدگار چون خواست كه جهانْ آبادْ مانَدْ، اسبابِ نسل پديد كرد در شهوتِ جانور؛ و پدر و مادر را سببِ كون فرزند كرد؛ پس هميدون از موجبِ خِرَد بَرْ فرزندْ واجب بُوَد پدر و مادرِ خود را حرمتْ داشتن، و اصل او پدر و مادرست، و تا نگويى كه پدر و مادر را بر من چه حق است‌؟ كه ايشان را غرضْ  شهوت بود [مقصود نه من بودم؛ هرچند مقصودْ شهوت بود]. مضاعفِ شهوتْ، شفقتى استاده است كه از بَهْرِ تو خودْ را به كُشْتن سپارند. و كمترْ حرمتِ پدر و مادر آنست كه هردو واسطه‌اند ميانِ تو و آفريدگارِ تو؛ پس چندانكه آفريدگارِ خود را و خودْ را حرمت دارى، واسطه را نيز دَرْ خورِ او ببايد داشت. و آن فرزندْ كه مادام  خِرَدْ رهنمونِ او بود، از حق و مهرِ مادر و پدر خالى نباشد؛ و خداى تعالى همى‌گويد در مَحْكَمِ تنزيل خود: «أَطِيعُوا اللّه وَ أَطِيعُوا الرَّسُولَ وَ أُولِي الْأَمْرِ مِنْكُمْ».اين آيت را تفسير كرده‌اند از چند روىْ، و يك روايت چنين خوانده‌ام كه "اولو الأمر" پدر و مادرند كه به حقيقت؛ أمر به تازى دوست، يا كارست، يا فرمان؛ و اولوالأمر آن بُوَد كه او را هم فرمان بُوَد و هم تَوان؛ و پدر و مادر را تَوان است به پروردن تو و فرمان است به خوبى آموختن.



و زينهار اى پسر! كه رنجِ مادر و پدرْ خوار ندارى، كه آفريدگار به حقِ مادر و پدر بسيار همى‌گيرد؛ و خداى تعالى همى‌گويد: «فَلا تَقُلْ لَهُما أُفٍّ وَ لا تَنْهَرْهُما وَ قُلْ لَهُما قَوْلاً كَرِيماً». و در خَبَرست كه اميرالمؤمنين را عليه السلام پرسيدند كه حقِ پدر و مادر بَرْ فرزند چيست و چند است‌؟ گفت: «اين ادب، خداى تعالى در مرگِ پدر و مادر، پيغامبر را عليه السلام بنمود كه اگر ايشان روزگار پيغامبر دريافتندى، بر پيغامبر واجبْ بودى ايشان را برتر از خويشتن داشتن و حقِ ايشان بشناختن، و در ايشان تواضع كِهْتَرى و فرزندى نمودن»؛ آنگه اين سخن ضعيف آمدى كه گفت:«أنا سَیِّدُ وَلَدُ آدَمْ و لا فَخْر»؛ پس حقِ پدر و مادر اگر از روىِ دينْ ننگرى، از روى مردمى و خِرَد بنگر؛ كه پدر و مادر مَنَّبِتِ\footnote{رویاننده}
 نيكى و اصل پرورش نفس تواند؛ و چون در حقِ ايشان مُقْصِر باشى، چنان نمايد كه تو سزاىِ هيچْ نيكى نباشى؛ كه آن‌كس كه او حق‌شناسِ نيكى اصل نباشد، نيكى فرع را هم حق نداند؛ و با ناسپاسان نيكى كردن از خيرگى بُوَد، و تو نيز خيرگىِ خويشْ مَجوى، و با پدر و مادرِ خويش چنان باش كه از فرزندان خويش طَمَعْ دارى كه با تو باشند؛ زيرا كه آنكه از تو زايد، همان طبع دارد كه تو ازو زادى؛ چه مثل آدمى چون ميوه است و پدر و مادر چون درخت؛ هر چند درخت را تَعَهُدْ بيش كنى، ميوۀ او نيكوتر و بهتر باشد؛ چون مادر و پدر را حرمت و آزرم بيش دارى، دعاو آفرين ايشان اندر تو مستجاب‌تر باشد و به خُشنودىِ خداى تعالى نزديك‌تر باشى. و نگر كه از بَهْرِ ميراثِ مرگ پدر و مادر نخواهى، كه بى‌مرگِ پدر و مادرْ آنچه روزىِ تو باشد به تو برسد، كه روزى مقسوم است؛ به هر كس آن رسد كه در اَزَلْ قسمت كرده شده است، و تو از بَهرِ روزى رنجِ بسيارْ بر خويشتن مَنِه، كه به كوششْ روزى افزون نشود؛ چه گفته‌اند مثل: «بِالْجَدِّ، لا بِالْكَدّ». و اگر خواهى كه از بَهرِ روزى از خداى تعالى خشنود باشى، بامداد به كسى منگر، كه حال او از حال تو بهتر باشد؛ بدانْ کَسْ نگر كه حال او از حال تو بَتَرْ باشد، تا دايم از خداى تعالى خشنودْ باشى. و اگر به مالِ درويش گَردى جهد كن، تا به خرد توانگر باشى، كه توانگریِ خِرَدْ از توانگرىِ مالْ بهتر باشد؛ و جاهل از مالْ زود مُفلِس شود، و مال خِرَدْ را دزد نتواند بردن، و آب و آتش هلاكْ نتواند كردن. پس اگر خِرَدْ دارى، با خِرَدْ هنرْ آموز؛ كه خِرَدِ بى‌هنر چون تنى باشد بى‌جامه، و شخصى بُوَد بى‌صورت؛ چه گفته‌اند مثل: «اَلْأدَبُ صورَةُ العَقْلِ».
\newpage