\addcontentsline{toc}{section}{باب بيست و نهم - 
در انديشه كردن از دشمن}
\section*{باب بيست و نهم - 
در انديشه كردن از دشمن}

اما جهد كن اى پسر تا دشمن نيندوزى، پس اگر دشمنت باشد مترس و دل‌تنگ\footnote{پریشان، مضطرب} مشو، كه هر كه را دشمن نباشد دشمن‌كام\footnote{تیره‌بخت، کسی که اوضاع و احوالش بر وفق مراد دشمن است} باشد. و لكن در نهان و آشكارا از كار او غافل مباش و ز بدكردنِ او مياساى، دايم در تدبير و مكر و بدى او باش و به هيچ‌وقت از حيله‌ی او ايمن مباش و از حال و راى دشمن پرسيده همى دار تا در بلا و آفت و غفلت بسته نباشى. و تا روى كار نباشد با دشمن دشمنى آشكارا مكن و خويشتن را به دشمن بزرگ نماى، اگر چه اوفتاده باشى، چاره را كار بند\footnote{به عمل در آوردن} و با وى خويشتن را از اوفتادگان منماى. و به كردار نيك و به گفتار خوش دل در دشمن مبند و اگر از دشمن شكر يابى آن را بى‌گمان شرنگى\footnote{حنظل، خربزه تلخ.} شمر. و از دشمنِ قوى هميشه ترسان باش كه گفته‌اند كه از دو كس ببايد ترسيد: يكى از دشمن قوى و ديگر از يار غدار\footnote{بی‌وفا}.

و دشمنِ خُرْدْ را هم‌ خوار\footnote{ذلیل، زبون، بدبخت} مدار\footnote{دشمن ضعیف را دست کم نگیر} و با دشمنِ ضعيف همچنان دشمنى كن كه با دشمن قوى كنى و مگوى كه او خُرْدَست.

حكايت چنانكه شنودم كه در خوراسان\footnote{خراسان} عيارى بود سخت محتشم و نيك‌مرد و معروف، مهلّب نام. گويند روزى در كوى همى‌رفت، اندر راه پاى بر خربزه‌پوستى\footnote{پوست خربزه‌ای} نهاد، پايش بلغزيد و بيفتاد، كارد بر كشيد و خربزه پوست را به كارد زد. چاكران او را گفتند: اى سرهنگ، مردى بدين عيارى و محتشمى كه تويى، شرم ندارى كه خربزه پوست را بكارد زنى‌؟ مهلب گفت: مرا خربزه پوست بيفگند من كه را به كارد زنم‌؟ هر كه را مرا بيفگند، من او را زنم كه دشمن من او بود. و دشمن را خوار نبايد داشت، اگر چه حقير دشمنى بود كه هر كه دشمن را خوار دارد زود خوار گردد. پس در تدبير هلاك دشمن باش از آن پيش\footnote{پیش از آن} كه وى تدبير هلاك تو كند.

اما با هر كس كه دشمنى كنى، چون بر وى چيره گشتى پيوسته آن دشمن را منكوه\footnote{نکوهش مکن} و به عاجزى به مردم منماى كه آنگه تو را فخرى نَبُوَد بدان چيرگى تو بر او\footnote{چیرگی تو بر او، برای تو فخر نباشد}، از عاجزى و نكوهيدگى او چيره شده باشى و اگر و العياذ باللّه وى بر تو چيره شَوَد، تو را عارى\footnote{عیب، ننگ، رسوایی} و عجزى عظيم باشد كه از عاجزى و نكوهيدگى افتاده باشى. نبينى كه چون پادشاهى فتحى كند، اگر چه خصمان\footnote{دشمنان، عدوان} پادشاه نه بس كسى بوده باشد، شاعران چون شعر فتح گويند و كاتبان چون فتح‌نامه نبيسند\footnote{بنویسند}، اول خصم را قادرى تمام خوانند و آن لشكر را بستايند و سواران و پيادگان را بشير\footnote{بشارت‌دهنده} و اژدها ماننده كنند و مصاف لشكر و قلب و جناح و سالار لشكر وى را هر چند بتوانند ستود بستايند. و آنگه گويند لشكرى بدين عظيمى، چون خداوند فلان با لشكر منصور خويش برسيد، هزيمت كرد و پشت بگردانيد تا بزرگى ممدوح خويش گفته باشند و قوّت لشكر خويش نموده چه [اگر] آن قوم منهزم\footnote{شکست خورده و ملغوب شده} را و آن پادشاه را به عاجزى نكوهند، اين پادشاه را كه مظفر باشد بس نامى و افتخارى نباشد به شكستن ضعيفى و عاجزى نه در فتح‌نامه و نه در شعرهاى فتح.

حكايت چنانكه وقتى به رِى‌\footnote{شهرِ رِی} زنى پادشاه بود، به لقب سيّده گفتندى. زنى بود ملك‌زاده و عفيفه و زاهده و كافيه\footnote{با کفایت} و دخترِ عمّ‌ِ مادرِ من بود، زن فخرالدوله بود. چون فخرالدوله فرمان يافت، وى را پسرى بود كوچك، مجدالدوله لقب دادندش و نام پادشاهى بر وى نهادند و خود پادشاهى همى‌راند سى و اند سال. چون مجدالدوله بزرگ شد، ناخلف بود، پادشاهى را نشايست همان نام ملك بر وى بود اما در خانه نشسته بود، با كنيزكان خلوت همى‌كرد و مادرش به رى و اصفهان و قهستان سى واند سال پادشاهى همى‌راند. مقصود من ازين سخن آنست كه جدّ تو، سلطان محمود رحمه اللّه، به رى رسولى فرستاد و گفت: بايد كه خطبه\footnote{کلام که در ستایش} بر من كنى و زَر به نام من زنى و خراج بپذيرى و اگر نه من بيايم و رِی بستانم و تو را نيست گردانم و تهديد بسيار كرد. و چون رسول بيامد و نامه بداد و پيام بگزارد، سيده گفت: بگوى سلطان محمود را تا شوى من فخرالدوله زنده بُوَد اين انديشه همى‌بود كه مگر تو را اين راى افتد و قصد رى كنى، چون وى فرمان يافت و شغل به من افتاد، انديشه از دل من برخاست، گفتم: محمود پادشاهى عاقلست داند كه چون او ملكى را به جنگ زنى نبايد آمدن، اكنون اگر بيايى خداى عزّوجل داند كه من نخواهم گريخت و جنگ را ايستاده‌ام، از بهر آنكه از دو بيرون نباشد: از دو لشكر يكى شكسته شود، اگر من تو را بشكنم به همه حال به همه عالم نامه نويسم كه سلطانى را شكستم كه صد پادشاه را شكسته است، و اگر تو مرا بشكنى چه توانى نبشت‌؟ گويى: زنى را شكستم، تو را نه فتح‌نامه رسد و نه شعر فتح كه شكستن زنى بس فتحى نباشد. بدين يك سخن تا وى زنده بود، سلطان محمود قصد وى نكرد.

و ازين گفتم كه دشمن خود را بسيار منكوه. و ديگر از دشمن به هيچ حال ايمن مباش، خاصّه از دشمن خانه و بيشتر از دشمن خانه ترس كه بيگانه را آن ديدار نيفتد در كار تو كه او را، چون از تو ترسيده گشت، دل وى هرگز از بد انديشيدن تو خالى نباشد و بر احوال تو مطّلع باشد و دشمن بيرونى آن نداند كه خانگى. پس با هيچ دشمن دوستى يك‌دل مكن و لكن دوست مجازى همى‌باش مگر آن مجازى حقيقى شود كه از دشمنى دوستى بسيار خيزد و ز دوستى دشمنى بسيار خيزد و آن دوستى و دشمنى كه چنين خيزد سخت‌تر باشد. و نزديكى با دشمنان از بيچارگى دان و دشمن را چنان گز كه از گزاييدن بر تو رنج نرسد. و جهد كن كه دوستانت اضعاف\footnote{جمع ضَعْف، به معنی دوچندان و زیاد‌تر} دشمنان باشند، بسيارْدوستِ كَم‌ْدشمن باش ولكن با صد هزار دوست يك دشمن مكن، زيرا كه آن هزار دوست از نگاه داشت تو غافل شوند و آن يك دشمن از بدسگاليدن تو غافل نباشد. و برداشتن سرد و گرم مردمان عار بين که هر كه مقدار خويش نداند اندر مردمى\footnote{انسانیت} او نقصان بود. و با دشمنى كه قوى‌تر از تو بود، آغاز دشمنى مكن و آن را كه ضعيف‌تر از تو بود از دشوارى نمودن مياساى. و لكن اگر دشمنى از تو زنهار\footnote{امان، پناه، مهلت} خواهد اگر[چه] سخت دشمن باشد و با تو بد كردار بود او را زنهار ده و آن غنيمتى بزرگ شناس كه گفته‌اند: دشمن چه مرده و چه گريخته و چه به زنهار آمده، ولكن چون زبون يابى يك‌بارگى نيز منشين. و اگر دشمن بر دست تو هلاك شود، روا بود اگر شادى كنى، اما اگر به مرگ خويش بميرد بس شادمانه مباش، آنگه شادى كن كه تو حقيقت دانى كه نخواهى مُرد، هر چند حكيمان گفته‌اند كه: هر كه به يك نفس از پس دشمن ميرد، آن مرگ را به غنيمت بايد داشت. اما چون دانيم كه همه بخواهيم مرد شادمانه نبايد بود به مرگ  كسى چنانكه در دو بيت من گويم:

\begin{quote}
گر مرگ برآورد ز بد خواه تو دود\quad \quad زان دود چنين شاد چرا گشتى زود؟ \\
چون مرگ تو را نيز بخواهد فرسود\quad \quad بر مرگ كسى چه شادمان بايد بود؟
\end{quote}
همه بر بسيج سفريم و توشه‌ی سفر جز كردار نيك هيچ‌چيز با خويش نشايد برد.

چنانكه شنيدم كه ذوالقرنين رحمه اللّه چون گرد عالم برگشت و همه جهان را مُسَخّر\footnote{رام} خويش گردانيد، بازگشت و قصد خانه‌ی خويش كرد. چون به دامغان رسيد فرمان يافت، در وصيّت گفت: مرا در تابوتى نهيد و تابوت را سوراخ كنيد و دست من از آن سوراخ بيرون كنيد كف‌گشاده، و همچنان همى‌بريد تا مردمان همى بينند كه اگر چه همه جهان بستديم دستِ تُهى همى‌رويم. دگر گفت: مادر مرا بگوييد كه اگر خواهى كه روانِ من از تو شادمانه باشد، غم من با كسى خور كه او را عزيزى نمرده باشد يا با كسى كه او نخواهد مرد.

و هر كسى را كه به دست بيندازى، به پاى همى‌گير از آنكه رسن اگر به حد و اندازه تابى، در يك ديگر همى پيوندد و چون بسيار تابى و از حد بيرون برى از هم بگسلد پس اندازه‌ی همه كارها نگاه‌دار، خواه در دوستى، خواه در دشمنى، كه اعتدال جزويست از عقل كلّى. و جهد كن در كار حاسدان خويش از بنمودن بديشان از چيزهايى كه ايشان را بدان خشم آيد، تا همى گدازند. و بر بدسگالان خويش بدسگال باش و لكن با افزونى جويان مچخ\footnote{چخیدن: کوشیدن، سعی کردن} و تغافل كن اندر كارِ ايشان كه آن افزونى جستن خود ايشان را افگند، كه همواره سبوى از آب درست نيايد\footnote{کنایه از اینکه هميشه كارها به كام ما بر نمي آيد}. و با سفيهان و جنگ‌جويان بردبارى كن و لكن با گردن‌كشان گردن‌كش باش. و هميشه در هر كارى كه باشى از طريق مردمى باز مگرد، و به وقت خشم بر خويشتن واجب كن خشم فروخوردن، و با دوست و دشمن گفتار آهسته دار و با آهستگى چرب‌گوى باش كه چرب‌گويى دوم جادويست. و هر چه بگويى از نيك و بد جواب چشم‌ دار، و هر چه نخواهى كه بشنوى مردمان را مشنوان، و هر چه از پيش مردم نتوانى گفت، از پس مردم مگوى. و بر خيره مردمان را تهديد مكن و لاف مزن بر كار ناكرده، مگوى كه: چنين كنم بلكه بگو كه: چون كردم، چنانكه من گويم:

\begin{quote}
از دل صنما مهر تو بيرون كردم\quad \quad و آن كوه غم تو را به هامون كردم \\
امروز نگويمت كه چون خواهم كرد \quad \quad فردا دانى كه گويمت چون كردم
\end{quote}
و كردار بيش از گفتار [شناس]. اما زبان خويش بر آن كس بسته دار كه اگر خواهد زبان خويش بر تو دراز نتواند كردن. و هرگز دورويى مكن و از مردم دوروى دور باش، و از اژدرهاى هفت سر مترس و از مردم نمّام\footnote{سخن‌چین} بترس كه هر چه او به ساعتى بشكافد به سالى نتوان دوخت. و هر چند بزرگ و محتشم باشى با قوى‌تر از خود مچخ، چنانكه آن حكيم گويد:

حكمت: ده خصلت پيشه كن تا از بلا رسته باشى: با كسى كه قوى‌تر از تو بود، پيكار مكن، و با كسى كه تند بود لجاج\footnote{عناد، یک‌دندگی، لجاجت} مكن، و با كسى كه حسود بود مجالست مكن، و با نادان مناظره مكن، و با مردم مرائى\footnote{ریا‌کار} دوستى مكن، و با دروغ‌زنان معامله مكن، و با بخيلان صحبت مكن، و با كسى كه معربد\footnote{آنکه عربده کند، بدمست} و غيور بود شراب مخور، و با زنان بسيار نشست و خاست مكن، و سرّ خويش با كسى مگوى كه آب بزرگى و حشمت خويش ببرى، و اگر كسى بر تو چيزى عيب گيرد آن عيب به جهد از خود دور كن، و خويشتن را به تكلّف بر مبر تا بى‌تكلّف فرو نيايى، و هيچ‌كس را چندان مستاى كه اگر وقتى ببايد نكوهيدن نتوانى و چندان منكوه كه اگر وقتى ببايد ستود نتوانى، و هر كه را بى‌ تو كار برآيد از خشم و گله‌ی خويش مترسان كه هر كه از تو مستغنى بود از خشم و گله‌ی تو نترسد، او [را] بترسانى هجاى خويش كرده باشى\footnote{خودت را مسخره کرده باشی}. و هر كه را بى‌تو كار برنيايد يك‌باره زبون مگير و بر وى چيره مشو و خشم ديگران بر وى مريز و اگر چه گناهى بزرگ بكند اندر گذار. و بر كهتران خويش بى‌بهانه بهانه مجوى، تو بدان آبادان باشى و ايشان از تو نفور\footnote{رمنده، دورشونده} نشوند. و كهتران را آبادان دار كه كهتران تو ضياع\footnote{زمین و آب و درخت، دارایی} تو اَند اگر ضياع خويش را آبادان دارى كار تو ساخته باشد و اگر ضياع را ويران دارى بى‌برگ و بى‌نوا باشى. و چاكر فرمان‌بردار دار، و چون شغلى فرمايى دو كس را مفرماى تا خلل از شغل و فرمان تو دور بُوَد كه گفته‌اند كه: ديگ به دو تن به جوش نيايد، چنانكه فرخى گويد:

\begin{quote}
خانه بدو كدبانو نارفته بماند
\end{quote}

و در مثل آمده است به تازى: «من كثرة الملاحين غرقت السفينة\footnote{کشتی به دلیلی تعدد ناخدایان غرق شد}.»


و اگر فرمان‌بردار باشى در آن فرمان شريك و انباز\footnote{شریک، دوست} مخواه، تا در آن كار با خلل و تقصير نباشى و دايم پيش خداوند سرخ‌روى باشى. امّا با دوست و دشمن كريم باش و اندر گناه مردم سخت مشو و هر سخنى را بر انگشت مپيچ و به هر حق و باطل دل در عقوبت مردم مبند و طريق كرم نگاه‌دار تا به هر زبانى ستوده باشى.




\newpage
\newpage
























