\addcontentsline{toc}{section}{باب هشتم - 
در ياد كردن پندهاى نوشين‌روان}
\section*{باب هشتم - 
در ياد كردن پندهاى نوشين‌روان}


اول گفت: تا روز و شب آينده‌ و‌ رونده است، از گردشِ حال‌ها شگفت مدار.

ديگر گفت: چرا مردمان از كارى پشيمانى خورند، كه از آن كار، ديگرى پشيمانى خورده باشد؟

ديگر گفت: چرا ايمن خسبد كسى كه با پادشاه آشنايى دارد؟

ديگر گفت: چرا زنده شِمُرَد كسى خويشتن را، كه زندگانى او جز به كام بُوَد؟

ديگر گفت: چرا نخوانى كسى را دشمن، كه جوانمردىِ خويش [را] در آزار مردمان داند؟

ديگر گفت: چرا دوست خوانى كسى را كه دشمنِ دوستانِ تو باشد؟

ديگر گفت: با مردمِ بى‌هنر دوستى مكن، كه مردمِ بى‌هنر نه دوستى را شايد و نه دشمنى را.

ديگر گفت و گفت: بپرهيز از نادانى كه خود را دانا شمرد.

ديگر گفت: داد از خويشتن بده، تا از داور مستغنى باشى.

ديگر گفت: حق گوى اگر چه طلخ\footnote{تلخ} باشد.

ديگر گفت: اگر خواهى رازِ تو دشمن نداند با دوست مگوى.

ديگر گفت: خُردْ نگرشِ بزرگْ‌زيان مباش.

ديگر گفت: بى‌قَدْرْ‌مردمْ را زنده مشمر.

ديگر گفت و گفت: اگر خواهى كه بى‌گنج توانگر باشى، بسند‌ْكار باش.

ديگر گفت: به گزاف مخر، تا بگزاف نبايد فروخت.

ديگر گفت: مرگ بِهْ، زآنكه نياز به هم‌سران خويش [بَرَد].

ديگر گفت: از گرسنگى بمردن بِهْ از آنكه به نانِ فرومايگان سير شدن.

ديگر گفت: به هر تخايلى كه تو را صورت بندد، بر نامعتمدان اعتماد مكن و از معتمدان اعتماد مَبُر.

ديگر گفت: به خويشاوندانِ کَم‌ از خويش محتاج بودن [را] مصيبتى عظيم دان، كه در آب مردن بِهْ كه از فزغ\footnote{بیم، ترس، هراس} زنهار\footnote{پناه و امان و مهلت} خواستن.

ديگر گفت: فاسقى متواضعِ اين جهان جوى، بهتر از قراى متكبر آن جهان جوى.

ديگر گفت و گفت: نادان‌تر از آن مردم نَبُوَد كه كهترى را به مهترى رسيده بيند، [ولی]همچنان ب وی به چشم كهترى نگرد.

ديگر گفت و گفت: شرمى نبود بزرگتر از آن كه به چيزى دعوى كند كه نداند و آنگه دروغ‌زَن باشد.

ديگر گفت: فريفته‌تر زان كسى نَبُوَد كه يافته به نايافته بدهد.

ديگر گفت: به جهان در فرومايه‌تر از آن كسى نيست كه كسى را بدو حاجت بُوَد و تواند اجابت كردن آن حاجت و او وفا نكند.

ديگر گفت: هركه را تو را [بى] گناهى زشت گويد، وى را تو معذورتر دار از آن كس كه آن سخن به تو رساند.

ديگر گفت: به خداوندِ مصيبتِ عزيز، آن دردسر نرسد كه بر آن كس كه بى‌فايده، گوش دارد.

ديگر گفت: از خداوند زيان بسيار آن زيان‌مندتر كه وى را ديدار چشم زيان‌مند بود.

ديگر گفت: هر بنده‌اى كه او را بخرند و بفروشند، آزادتر از آن كس بُوَد كه گلوبنده\footnote{پرخور، شکم‌خواره} بود.

ديگر گفت: هر چند دانا كسى بود كه با دانِشْ وِى را خِرَدْ نيست، آن دانش بر وى وبال بود.

ديگر گفت كه: هر كسى را كه روزگار او را دانا نكند، هيچ دانا را در آموزشِ او رنج نبايد بردن، كه رنج او ضايع بُوَد.

ديگر گفت: همه چيزى از نادان نگاه داشتن آسان‌تر كه ايشان را از تن خويش.

ديگر گفت: اگر خواهى كه مردمان نيكوگوىِ تو باشند، مردمان را نيكوگو باش.

 گفت: اگر خواهى كه رنج تو به جاى مردمان ضايع نشود، رنج مردمان به جاى خويش ضايع مكن.

ديگر گفت: اگر خواهى كه كَم‌ْدوست و كم‌يار نباشى، كينه مدار.

ديگر گفت: اگر خواهى كه بى‌اندازه اندهگن نباشى، حسود مباش.

ديگر گفت: اگر خواهى كه زندگانى به آسانى گذارى، رَوِشِ خويش را بر روى كاردار.

ديگر گفت: اگر خواهى كه از رنج دور باشى، آنچه نَرَوَد مَران.

ديگر گفت: اگر خواهى كه تو را ديوانه‌سار نشمارند، آنچه نايافتنى بُوَد مجوى.

ديگر گفت: اگر خواهى به آبِ‌رَوى\footnote{آب رونده} باشى، آزرم را پيشه كن.

ديگر گفت: اگر خواهى كه فريفته نباشى، كار ناكرده را به كرده مدار.

ديگر: اگر خواهى كه پرده‌ی تو دريده نشود، پرده‌ی كسان مدر.

گفت: اگر خواهى كه دَرْ پسِ قفاىِ تو نخندند، زيردستان را باك دار.

گفت: اگر خواهى كه از پشيمانىِ دراز ايمن گردى، به هواى دل كار مكن.

گفت: اگر خواهى كه از زيركان باشى، روى خويش در آينه‌ی كسان بين.

گفت: اگر خواهى كه قدرِ تو به‌ جاىْ باشد، قدرِ مردم بشناس.

گفت: اگر خواهى كه بر قول تو كار كنند، بر قول خويش كار كن.

گفت: اگر خواهى كه ستوده‌ی مردمان باشى، برآن كس كه خِرَدْ زو نهان باشد، نهانِ خويش آشكارا مكن.

گفت: اگر خواهى كه برتر از مردمان باشى، فراخ‌ْ‌‌نان‌‌ و‌‌ نمك باش.

گفت: اگر خواهى كه از شمار آزادان باشى، طمع را در دل خويش جاى مده.

گفت: اگر خواهى كه از شمار دادگران باشى، زيردستان خويش را به طاقتِ خويش نيكودار.

گفت: اگر خواهى كه از نكوهشِ عام دور باشى، اثرهاى ايشان را ستاينده باش.

گفت: اگر خواهى كه در هر دلى محبوب باشى و مردمان از تو نفور نباشند، سخن بر مرادِ مردمان گوى.

گفت: اگر خواهى كه تمامْ‌مردم باشى، آنچه به خويشتن نپسندى، به هيچ كس مپسند.

گفت: اگر خواهى كه بر دلت جراحتى نيوفتد كه به هيچ مرهم بهتر نشود، با هيچ نادان مناظره مكن.

گفت: اگر خواهى كه بهترين خلق باشى، چيز از خلق دريغ مدار.

گفت: اگر خواهى كه زبانت دراز بُوَد، كوتاه‌دست\footnote{کسی که دستش به مراد و مطلوب نرسد، ناکام، نامراد} باش.

اينست سخن‌ها و پندهاى نوشروانِ عادل؛ چون بخوانى اى پسر! اين لفظ‌ها را خوار مدار، كه ازين سخن‌ها [هم] بوى حكمت آيد و هم بوىِ مُلْك؛ زيرا كه هم سخنان ملكان است و هم سخن حكيمان؛ جمله معلوم خويش كن و اكنون آموز كه جوانى، چون پير گردى به انديشيدن حاجت نيايد، كه پيران چيزها دانند.


 	\newpage



























