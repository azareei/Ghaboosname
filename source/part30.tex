\addcontentsline{toc}{section}{باب سى‌ام - 
در عقوبت كردن و حاجت خواستن و روا كردن}
\section*{باب سى‌ام - 
در عقوبت كردن و حاجت خواستن و روا كردن}

و بهر گناهى، اى پسر، مردم را مستوجبِ عقوبتْ مدان و اگر كسى گناهى كند، از خويشتن اَندَرْ دِلْ عذرِ گناه او بخواه، كه او آدميست و نخستين گناهى آدم كرد، چنانكه من گويم:

\begin{quote}
گر من روزى ز خدمتت گشتم فرد\quad \quad صد بار دلم از آن پشيمانى خورد \\
جانا به يكى گناه از بنده مگرد\quad \quad من آدميم گنه نخست آدم كرد
\end{quote}

و خيره\footnote{باطل، بیهوده} عقوبت مكن، تا بى‌گناه سزاىِ عقوبت نگردى. و بهرِ چيزى خشم‌ْناك مشو، در وقتِ ضجرت\footnote{ملال، اندوه} خشم فروخوردن عادت كن، چون به گناهى از تو عفو خواهند، عفو كن؛ و بر خويشتن واجب‌دان اگرچه سخت گناهى بود، كه بنده اگر گناه‌كار نباشد، عفو خداوند پيدا نيايد و چون مكافات گناه كرده باشى آنگه تفضل تو كجا رسد. و چون عفو كردن واجب دانى، از شرف و بزرگى خالى نباشى و چون عفو كردى ديگر او را سرزنش مكن و از آن گناه ياد ميار، كه آنگه همچنان باشد كه آن عقوبت نكرده باشى. اما تو گناهى مكن كه تو را عذر بايد خواستن، پس اگر اتفاق افتد، كه تو را از كسى عذر بايد خواستن، از عذر خواستن ننگ مدار تا ستيزه منقطع شود.

اما اگر كسى گناهى كند كه مستوجبِ عقوبتْ بود، حدِ گناه او بنگر و اندر خور گناه او عقوبت فرماى كه خداوندان انصاف چنين گفته‌اند كه: عقوبت سزاى گناه بايد كرد، اما من چنين گويم كه: اگر كسى گناهى كند كه بدان گناه مستوجبِ عقوبت شود و تو سزاىِ آن گناه او را عقوبت كنى، طريق حِلم\footnote{شکیبایی،‌ صبر} و كرم و رحمت فراموش كرده باشى، چنان بايد كه درمى گناه را نيم درم عقوبت كنى، تا هم رسمِ سياست به جاى آورده باشى و هم شرط كرم نگه داشته باشى، تا هم از كريمان باشى و هم از سايسان\footnote{ادب کننده،‌ تربیت کننده}، كه نشايد كه كريمان كارِ بى‌رحمتان كنند.

چنانكه شنودم كه به روزگار معاويه قومى گناهى كردند كه كشتن بر ايشان واجب بود. معاويه پيش خويش فرمود، ايشان را گَردَن زدن. در ميانه يكى را پيش آوردند كه بِكُشَند. آن مرد گفت: يا امير! هر چه با ما كنى سزاىِ ماست و بر گناه خود مقرّم\foornote{متضاد منکر}، اما از بهرِ خداى را عزّوجل از من دو سخن بشنو و جواب دِه.معاويه گفت: بگوى. مرد مُجرِم گفت كه: همه عالم از حلم و كرم تو همى گويند و اگر اين گناه در پيش پادشاهى كردمى كه نه چون تو كريم و حليم و رحيم بودى، آن پادشاه با ما چه كردى‌؟ معاويه گفت: هم اين كه من همى كنم. مرد گفت: پس ازين كريمى و حليمى و رحيمى تو ما را چه سود دارد، كه تو همان كنى كه آن بى‌رحمت و بى‌حلم‌؟ معاويه گفت: اگر اين سخنْ آن مردِ نخستين گفته بودى، همه را عفو كردمى، اكنون اين‌ها كه مانده‌اند عفو كنيد. پس چون مجرمى عفو خواهد، اجابت كن و هيچ گناهى مدان كه آن به عذر نيرزد.

فصل، و اگر حاجتمندى را به تو حاجت افتد از ممكنات\footnote{جمع ممکنه، کلیه‌ی موجودات عالم را ممکنات گویند}، كه دين را در آن زيانى نبود و در مهمات دنيایى از بيشى خللى نبود، از بهرِ كم‌مايه دنيا، دلِ آن نيازمند باز مزن و آن‌كس را بى‌قضاىِ حاجت باز مگردان و ظن آن حاجتمند در خويشتن دروغ مكن، كه آن مرد تا در تو گمانِ نيكو نَبَرَد، از تو حاجت نخواهد.

و نيز آن مستمند در وقت حاجت خواستن اسير تو بود و گفته‌اند كه: حاجتمندى دوم اسيريست و بر اسيران رحمت بايد كرد، كه اسير كشتن ستوده ندارند كه كارى نكوهيده است. پس در اين معنى تقصير روا مدار تا محمدتِ\footnote{ستایش و مدح و ثنا و ذکر خیر و نیک نامی} دو جهانى بيابى.

فصل، و اگر تو را به كسى حاجت بود، بنگر كه آن مرد كريم‌ست يا لييم\footnote{بخیل، ناکس، فرومایه}، اگر مرد كريم باشد، حاجت بخواه اما فرصت نگاه‌دار، به وقتى كه دل‌تنگ باشد حاجت مخواه. و چون حاجت خواهى از ممكنات خواه تا به اجابت مقرون\footnote{نزدیک} بُوَد و نوميد نگردى. و نيز پيش از طعام بر گرسنگى حاجت مخواه. و در حاجت خواستن، سخنِ نيكو بينديش، وز پيش قاعده‌ی نيكو فرو نِهْ و آن‌گاه مَخلَصِ سخن بدان جاى رسان كه حاجت تو بيرون آيد. و اندر سخن گفتن تلطف بسيار نماى كه تلطف در حاجت خواستن دوم شفيعي‌ست كه اگر حاجت بدانى خواستنْ بى‌قضاىِ حاجتْ باز نگردى و حاجت تو روا شود. چنانكه من گويم:

\begin{quote}
اى دل خواهى كه زى دلارام رسى \quad \quad بى‌تيمارى بدان مه تام رسى \\
با او به مراد دل بِزى اى دل ازآنچ \quad \quad گر دانى خواست كامه در كام رسى
\end{quote}
و هر كه را بدو\footnote{به او} محتاج باشی، خويشتن چون چاكر و بنده‌ی او شناس، كه ما بندگیِ خداىِ تعالى را از آن همى‌كنيم كه ما را به وِى حاجت است، كه اگر به خداى تعالى حاجت نبودى هيچ‌كس روىْ سوىِ عبادتِ خداى تعالى نكردى. و چون اجابت يابى به هر جاىْ از آن شكرى بگوى، كه خداى عزّوجل مى‌گويد: «لَئِنْ شَكَرْتُمْ لَأَزِيدَنَّكُمْ » كه شاكران را خداوند سبحانه و تعالى دوست دارد و نيز شكر كردن به حاجت نخستين اميد اجابت حاجت دومين بُوَد. و اگر حاجت تو روانه كند، از بختِ خويش بين و از آن كس گله مكن، كه اگر وِى از گله‌كردنِ تو باك داشتى، خود حاجتِ تو روا كردى. پس اگر مرد بخيل و لييم باشد، به هشيارى از او هيچ چيز مخواه كه ندهد، به وقت مستى خواه كه بخيلان و لييمان به وقت مستى سخى باشند و كرم نمايند و اگرچه روزِ ديگر پشيمان شوند. و اگر حاجت به لييمى اُفتد، خويشتن را به جاىِ رحمت دان، كه گفته‌اند كه: سه تن به جاى رحمت‌اند: خردمندى كه زيرْدستِ بى‌خردى بُوَد، و ضعيفى كه قوى بر وِى مسلّط بُوَد، و كريمى كه محتاج لييمى بُوَد.

و بدان اى پسر! كه چون اين سخن‌ها كه در مقدمه گفتم بپرداختم، از هر نوعى فصلى بگويم، بر موجبِ طاقتِ خويش، خواستم كه دادِ سخنْ تمام بدهيم از پيش‌ها نيز ياد كنم تا آن نيز بخوانى و بدانى كه مگرت بدان حاجت افتد.

از آنچه خواستم كه علم اولين و آخرين من دانمى كه تو را بياموختمى و معلوم تو گردانيدمى، تا به وقتِ مرگْ، بى‌غمِ تو از اين جهان بيرون شدمى و لكن چه كنم كه من خود در دانش پياده‌ام، و اگر نيز چيزى دانم گفتار من نيز چه فايده دارد، اكنون تو از من هم چندان شنوى كه من از پدر خويش شنودم، پس تو را جاى ملامت نيست كه من خود داد از خويشتن بدهم تا به داور حاجت نباشد. اگر تو شنوى و اگرنه، در هر پيشه‌اى سخنى چند بگويم تا در سخن بخيلى نكرده باشم، كه آنچه مرا طبع دست داد بگفتم.




\newpage



























