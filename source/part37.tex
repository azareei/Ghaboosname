\addcontentsline{toc}{section}{باب سى و هفتم -
در خدمت كردن پادشاه}
\section*{باب سى و هفتم - 
در خدمت كردن پادشاه}

اگر اتفاق افتد كه از جمله‌ی حاشيه باشى و به خدمت پادشاه پيوندى،  هر چند كه پادشاه تو را به خود نزديك كند، تو بدان غرّه مشو؛ از نزديكى وى گريزان باش، اما از خدمت گريزان مباش، كه از نزديكى پادشاه دورى خيزد و از خدمت پادشاه نزديكى. و اگر تو را از خويشتن ايمن دارد آن‌روز ناايمن باش، كه از هر كه فربه شوى، نزار\footnote{زبون گشتن، اسیر شدن، گرفتار شدن} گشتن هم از وى بود. و هر چند عزيز باشى از خويشتن شناسى غافل مباش و سخن جز بر مراد خداوند مگوى و با وى لجاج مكن كه هر كه با خداوند خويش لجاج كند پيش از اجل بميرد، كه با درفش\footnote{پرچم، رایت،‌علم} مشت زدن احمقى بود. و خداوند خويش را جز به نيكى كردن راه منماى، تا با تو نيكى كند كه چون بد آموزى با تو جز بدى نكند.

چنانكه به روزگار فضلون كه پادشاه گنجه بود، ديلمى بود محتشم كه مشير او بود، پس هر كسى كه گناهى بكردى كه بند و زندان واجب شدى فضلون او را بگرفتى و به زندان كردى، اين ديلم كه مشير او بود او را گفتى: آزاده را ميازار و چون آزردى بيوژن\footnote{اوژنیدن:به خاک افکندن}، و چند كس به مشورت او هلاك شدند. تا به اتفاق اين ديلم گناهى بكرد، امير وى را بفرمود گرفتن و به زندان كردن. ديلم كس فرستاد كه: چندين مال بدهم مرا مكش. فضلون گفت: من از تو آموختم كه آزاده را ميازار و چون آزردى بيوژن، و اين ديلم جان در سر كار بد آموزى كرد.

امّا اگر از نيك نكوهيده شوى دوست‌تر دار كه از بد ستوده شوى. و آخر همه تمام‌ها نقصان‌شناس و بر دولت غرّه مباش و از كارِ سلطان حشمت طلب كن، كه نعمت خود از پس حشمت آيد كه عزّ خدمت سلطانى بِهْ از عزّ توانگريست. و اگر چه در عملِ پادشاه فربه شوى، خويشتن لاغر نماى تا ايمن باشى، نبينى كه گوسفند تا لاغر باشد از كشتن ايمن بُوَد و كس بكشتن او نكوشد و چون فربه گشت همه‌كس را بكشتن او طَمَعْ بُوَد. و از بهرِ دِرَمْ خداوندْ‌فروش مباش كه درم عمل سلطانى چون گل بود، نيكو و خوش‌بوى و مشهور و عزيز و لكن چون گُلِ عمر يابد، هر چند منافع خدمت سلطان پنهان نتوان داشت، هر درمى كه از عمل سلطان گرد كنى، از غبار عالم پراگنده‌تر شود. و حشمت خدمت خداوندان خود چون سرمايه است و درم كه ازو گرد آيد راست چون سودست، پس از بهر سود سرمايه از دست مده، كه تا سرمايه بر جاى بود هميشه اوميدِ سود بود، اگر سرمايه از دست بدهى، اندر سود نتوان رسيد. و هر كه درم را از خويشتن عزيزتر دارد زود از عزيزى به ذليلى اوفتد و رغبت جمع كردن مال اندر ميان عزِّ مرد هلاك مرد عزيز دان، مگر به حد و اندازه جمع همى كند و مردمان را نصيب مى‌كند، تا زبان خلق بر وى بسته گردد. و چون در خدمت سلطان بزرگ شدى و پايگاه يافتى هرگز با خداوند خويش خيانت مكن، اگر كنى از تعليم بخت بد تو باشد زيرا كه چون خداوندى كهترى را بزرگ گرداند، وى مكافات ولى نعمت خيانت كند دليل آن بود كه آن بزرگى خداوند ازو بازخواهد گرفت از آنكه تا محنتى بدان مرد نخواهد رسيد مكافات خداوند خويش بدى نكند.

چنانكه امير فضلون بوالسوار، ابوالبشر را با سفهسلاّرى\footnote{سپه‌سالار} بردع\footnote{شهری در جمهوری آذربایجان کنونی} همى‌فرستاد. وى گفت: تا زمستان در نيايد نروم، از آنچه آب و هواى بردع سخت بَدَسْت خاصه به تابستان و اندرين معنى سخن دراز گشت و امير فضلون وى را گفت: كسى بى‌اجل نميرد و نَمُرْدَسْت. بوالبشر گفت: چنانست كه خداوند همى‌گويد، كس بى‌اجل نميرد و لكن كسى را تا اجل نيامده باشد، به تابستان به بردع نرود.

و ديگر از كار دوست و دشمن ايمن مباش بايد، كه نفع و ضرّ تو به دوست و دشمن برسد، كه بزرگى بدان خوش باشد كه دوست و دشمن را به نيكى و بدى مكافات كنى. و مردم كه محتشم شد، نبايد كه درخت بى‌بر باشد و از بزرگى توانگرى خواهد و كس را از وى نفع و ضر نبود، كه جهود باشد كه صد هزار دينار دارد و به يك دانگ نفع و ضر او به مردم نرسد و ازو كمتر كس نباشد. پس منافع خويش را نعمت و كام‌رانى دان و مردمى از مردمان بازمگير، كه صاحب شريعت، صلوات‌اللّه‌عليه، گفته است: «خير الناس من ينفع الناس.» و خدمت مهترى كه به غايت به دولت رسيده باشد مجوى، كه به فرود آمدن نزديك بود و گرد دولت پير شده مگرد كه پير را اگر چه هنوز عمر مانده باشد، آخر مردمان او را به مرگ نزديك‌تر خوانند و نيز كم پيرى بود كه روزگار پيرى با وى وفا كند. و اگر خواهى كه در خدمت پادشاه بر جاى بمانى چنان باش كه عباس، رضى‌اللّه عنه، عمّ پيغامبر ما، صلى‌اللّه‌عليه‌و‌سلم، پسر خويش عبد اللّه را، رضى‌اللّه‌عنه،‌ گفت: بدان اى پسر كه اين مرد يعنى امير‌المؤمنين عمر، رضى‌اللّه‌عنه، تو را پيش خويش به شغل كرد و از همه خلق بر تو اعتماد كرد، اكنون اگر خواهى كه دشمنان تو بر تو چيره نگردند پنج خصلت نگاه دار تا هميشه ايمن باشى: اول بايد كه هرگز از تو دروغ نشنود، و دوم پيش او كس را عيب مجوى، سوم با وى به هيچ چيز خيانت مكن، و چهارم فرمان او را خلاف مكن و پنجم راز او با هيچ كس مگوى كه از مخلوق پرستيدن مقصود خويش بدين پنج خصلت حاصل توانى كرد. و ديگر هرگز تا بتوانى اندر خدمت ولى‌ْنعمتِ خويشْ تقصير مكن، پس اگر تقصير كنى، خويشتن را به مقصرى بدو منما و اندر آن تقصير خود را نادانى ساز بدين گونه خود را بدو نماى تا نداند كه تو قصد كرده‌اى و آن تقصير خدمت از تو به نادانى برگيرد نه به بى‌ادبى و نافرمانى، كه نادانى از تو به گناه نگيرند و بى‌ادبى و نافرمانى به گناه شمرند. ديگر پيوسته به خدمت مشغول باش، بى‌آنكه بفرمايد هر چه كسى ديگر خواهد كرد، بكوش كه تا تو كنى، چنان بايد كه هرکه تو را طلبد در خدمت از آن خويش يابد، مادام به درگاه حاضر باشى چنانكه هركه را طلب كند، تو را يابد، زيرا كه همت ملكانه پيوسته آنست كه دايم در آزمايش كهتران خويش باشند، تو را چون يك راه و دو راه بجويد، هر بارى در خدمتى يابد، مقيم به درگاه خويش بيند، به كارهاى بزرگ بر تو اعتماد كند چنانكه قمرى گرگانى گويد: 

\begin{quote}
پيش تو ما را سخن گفتن خطر كردن بود\\ 
به خطر كردن بر آرند از بن دريا گهر
\end{quote}
و تا رنج كهترى بر خويشتن ننهى، به آسايش مهترى نرسى، كه برگ نيل تا پوسيده نشود نيل نشود و آفريدگار پادشاه را چنان آفريده است كه همه را بيند كه به وِى حاجتمندند. و خود را به حسد به پادشاه منماى، كه اگر بعد از آن سخن كسى محسود پيش او بگويى نشنود و از جمله‌ی حسد شُمُرَد، اگر چه راست بُوَد. و هميشه از خشمِ پادشاه ترسان باش، كه دو چيز را هرگز خوار نتوان داشتن: يكى خشم پادشاه و ديگر پند حكيمان، كه هر كه اين دو چيز را خوار دارد خوار گردد. اينست شرط حاشيت پادشاهان، پس اگر چنانكه تو ازين درجه برگذرى و پايگاه بزرگتر يابى و به نديمى پادشاه افتى بايد كه شرط نديمى پادشاه تو را معلوم گردد واللّه‌اعلم.


















 
