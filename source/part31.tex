\addcontentsline{toc}{section}{باب سى و يكم -
در طالب علمى و فقيهى و فقها}
\section*{باب سى و يكم - 
در طالب علمى و فقيهى و فقها}

بدان اى پسر! كه گفتم كه اول سخن كه از پيش‌ها ياد كنم، غرض\footnote{هدف} پيشه\footnote{حرفه کار} نه دكان‌داريست\footnote{شغل و حرفه تنها دکان‌داری نیست}، هر كارى كه مردم كند و بَرْ\footnote{ثمر، ثمره، محصول} دست گيرد، آن چون پيشه است، بايد كه آن كار را نيك بدانى ورزيدن، تا از آن بَرْ توانى خوردن. اكنون چنانكه من همى‌بينم هيچ پيشه و كارى نيست كه [آدمى] آن بجويد كه آن پيشه را از داستان\footnote{نقل، حکایت} و نظام\footnote{نظم، آراستگی} مستغنى دانى، الاّ كه همه را ترتيب\footnote{نظم} دانستن بايد. و پيشه بسيارست، هر يكى را جدا شرح كردن ممكن نشود، قصه دراز گردد و كتاب من از نهاد\footnote{خلقت، سرشت} واصل بشود\footnote{به منتهی رسیدن و به حق متصل شدن} و لكن از هر صفت كه هست از سه وجه است: يا علمى است كه تعلّق به پيشه دارد، يا پيشه است كه تعلّق به علم دارد، يا خود [پيشه است] به سِرِّ خويش. اما علمى كه تعلّق به پيشه دارد چون طبيبى و منجمى و مهندسى و مسّاحى\footnote{اندازه‌گیری زمین} و شاعرى و مانند اين، و پيشه [اى] كه تعلّق به علم دارد چون خُنياگَرى\footnote{مطربی، نوازندگی، آوازخوانی} و بيطارى\footnote{دام‌پزشکی} و بنّائى و كاريز\footnote{قنات} كنى و مانند اين، و اين هر يكى را ساماني‌ست كه اگر تو رسم و سامان\footnote{نظم، اسباب، لوازم} اين ندانى، اگر چه استاد كسى باشى، در آن باب چون اسيرى باشى. پيش‌هاى خود معروفست، به شرح حاجت نيوفتد و لكن من چندان كه صورت بندد بگويم و سامان هر يك به تو نمايم، آز انچه از دو بيرون نَبْوَد: يا خود تو را بدين دانش نياز افتد، از اتّفاق روزگار و حوادث‌هاى زمانه، بارى به وقت نياز از اسرار هر يكى آگاه باشى. پس اگر نيازت نباشد و همچنين مهتر باشى كه هستى، مهتران را هم علم پيش‌ها دانستن لابدّ\footnote{ناگزیر} است.

بدان اى پسر! كه از هيچْ علم بَرْ نتوانى خورد، الاّ از علم آخرتى كه اگر خواهى كه از علم دنياوى بَرْ خورى، نتوانى خورد، مگر كه مخرقه\footnote{دروغ} درو آميزى، كه با علمِ شرع كه در كارِ قضا و قسامى و كرسى‌دارى\footnote{داشتن محضر و منبر} و مُذَكَرى\footnote{واعظی} در نرود نفع دنيا به عالم نرسد. و در نجوم تا تقويم‌گرى و مولودگرى و فال‌گويى به جد و هزل در نرود دنيا به منجم نرسد. و اندر طب تا دستِ كارى و رنگ آميزى و هليله\footnote{میوه‌ای بیضوی شکلی به اندازة یک سنجد ریز است که مصرف طبی دارد و خشک شده آن را به عنوان قابض به کار می برند، این گیاه خاص نواحی حاره است.} دهى به صواب و ناصواب در نرود هم مراد حاصل نشود. پس بزرگوارترين علمى علم دين‌ست كه اصول او نردبان توحيدست و فروع او احكامِ شرع است و مخرقه‌ی\footnote{دروغ و ریا} او نفع دنيا. پس اى پسر! تو نيز تا بتوانى گرد\footnote{گردآوری کردن} علمِ دين كرد تا دنيا و آخرت به دست آورى، اما اگر اين توفيق به دست آرى، نخست اصولِ دين راست كن، آنگه فروع، كه فروع بى‌اصول، تقليد بود.

فصل، پس اگر چنانكه از پيش‌ها چنين كه فرمودم طالب علم باشى، پرهيزگار و قانع باش و علم‌ْدوستْ و دنيا‌دشمنْ و بُردْبار و خفيف‌روح و دير‌خواب و زودْخيز و حريص به كتابت و درس و متواضع و آگاه از كار و حافظ و مكرّرِ كلام و متفحصِ سير و متجسّسِ اسرار و عالمْ‌دوست و متقرّب و با حرمت و در آموختن حريص و بى‌شرم و حق‌شناسِ استاد خود. و الفغده‌ی\footnote{اندوخته} تو بايد كه كتاب‌ها و اجزا و قلم و قلم‌دان و محبره\footnote{دوات و مرکب} و كارد قلم‌تراش و مانند اين چيزها بُوَد و جُز اين دِلِ تو به چيزى بسته نَبُوَد و هر چه بشنوى ياد گرفتن و باز گفتن. و كم‌ْسخنْ و دورْانديش باش و به تقليد راضى مباش، كه هر طالبِ علمى كه از اين سيرت\footnote{گونه، سرشت} بود، زود يگانه‌ی روزگار گردد.

فصل، و اگر عالمى مفتى\footnote{فتوا دهنده} باشى، با ديانت باش و بسيارْ‌حفظ و بسيارْدرس و در عبادت و نماز و روزه تجاوز مكن و دو روى مباش و پاكْ‌دين و پاكْ‌جامه باش و حاضرْجواب و هيچ مسئله را تا نينديشى بخ زودى جواب مده و بى‌حجتى به تقليدِ خويش قانع مباش و به تقليدِ كَسْ كار مكن و راىِ خود را عالى دان و بر وجهين\footnote{صورت} و قولين\foonote{گفتار} قناعت مكن. و جز بر خط معتمدان اعتماد مكن، هر كتابى را و هر جزوى را مقدّم مدار. اگر روايتى شنوى، به راويانِ سخن اندر نگر، سخن مجهول از راوى معروف مشنو و بر خَبْرِ\footnote{شایعه} آحاد اعتماد مكن، مگر از راويان معتمد، و از خبر متواتر\footnote{پی‌درپی، پیوسته تکرار شده} مگريز و مجتهد باش و به تعصب سخن مگوى و اگر مناظره كنى به خصم نِگَر، اگر قوّت او دارى و خواهى كه سخن بسيط\footnote{گسترده} گردد، مداخله كن به مسئل‌ها و اگر نه سخن را موقوف گردان و به يك مثال قناعت كن. و به يك حجّت طرد و عكس\footnote{طرد و عکس این است که سخنی را به ترتیبی بگویند و آنگاه آن را معکوس کنند و بازگویند،مثلا اینکه گفته شود: هر آتشی گوهری تابنده و سوزنده است و هر گوهری تابنده و سوزنده آتش است. این یه حجت طرد و عکس میباشد.} به هم\footnote{با هم} مگوى، نخستين را نگاه دار تا سخن پسين تباه نكند. اگر مناظره‌ی فقهى بود، آيت را بر خبر مقدم دار و خبر را بر قياس مقدّم دار و ممكنات گوى و در مناظره‌ی اصولى موجبات و ناموجبات و ممكنات و ناممكنات به هم عيب بُوَد، جهد كن تا غرض معلوم كنى. و سخن با زينت گوى، دُم‌ْبريده مگوى و نيز دُم‌ْدراز و بى‌معنى مگوى.

و اگر مذكر باشى، حافظ باش و ياد بسيار گير و هرگز بر زبر\footnote{بالا} كرسى جدل مكن و مناظره مكن الّا كه دانى كه خصم ضعيف است. و بر سر كرسى به هر چه خواهى دعوى بكن كه آنجا سايل\footnote{فقیر، گدا} باشد، مجيب\foonote{پاسخ‌دهنده} كس نبود و تو زفان فصيح كن و چنان دان كه آن مجلسيان تو همه بهايم‌\footnote{جمع بهیمه،‌چهارپایان}اند چنانكه خواهى همى‌گوى تا به سخن اندر نمانى. و لكن جامه پاك‌ْدار و مريدان نعره‌زن دار، چنانكه در مجلس تو باشند تا به هر نكته كه تو بگويى وى نعره بزند و مجلس گرم همى دارد و چون مردم بگريند، تو نيز وقت وقت همى‌گرى. و اگر به سخنى درمانى، باك مدار به صلوات و تهليل\footnote{لاالله الا الله گفتن} و گرم‌ْسخنى همى گذران و بر سَرِ كرسى گران‌ْجان و ترش‌ْروى  و سردْعبارت مباش، كه آنگه مجلسِ تو نيز همچون تو، گرانْ‌جان بود، از آنچه گفته‌اند: «كلّ شىء من الثقيل ثقيل».

و متحرك باش اندر سخن و در ميانِ گرمى زود سُسْت مشو. و مادام مستمع\footnote{شنونده} را نگر، اگر مستمع نكته خواهد، نكته گوى و اگر فسانه خواهد، فسانه گوى كه ندانى كه عام خريدارِ چه باشند و چون قبول افتاد باك مدار، بترين\footnote{بدترین} سخنى به بهترين همى فروش، كه به وقت قبول بخرند، لكن اندر قبول دايم با ترس باش كه خَصْمِ مذكر از در قبول پديد آيد. و جايى كه قبول نيابى قرار مگير و هر سؤالى كه از تو پرسند آن را كه دانى جواب ده و آن را كه ندانى بگوى كه: چنين مسئله نه سر كرسى را بود\footnote{در منبر و بر سر کرسی جای جواب چنین سوالی نیست}، به خانه آى تا به خانه جواب دهم\footnote{به خانه‌ من بیا تا آنجا جواب سوالت را بگویم}، كه خود كسى به خانه نيايد بدان سبب. و اگر تعمّد كنند و بسيار نويسند، رقعه را بِدَر\footnote{پاره کن} و بگوى كه: اين مسئله‌ی ملحدانست و زنديقان\footnote{دهری، کافر} است، سايل\footnote{پُرسَنْده} اين مسئله زنديقست، همه بگويند كه: لعنت بر ملحدان باد و زنديقان كه ديگر آن مسئله از تو كس نيارد پرسيدن. و سخنى كه در مجلس گفتى، حفظ‌ دار كه چه گفتى كه تا بدان اوقات ديگرباره آن را تكرار نكنى، هر وقت تازه‌روى باش. و در شهرها بسيار منشين، كه مذكران و فال‌گويان را روزى اندر پایْ بود و قبول در روى تازگى. و ناموس مذكرى نگاه‌دار، هميشه جامه و تن پاك‌ دار و ظاهر و باطن بمعامله‌ی شرعى آراسته‌دار، چون نماز و روزه‌ی تطوع\footnote{فرمان‌برداری}، و چرب‌زبان باش و در بازار در ميان عام بسيار مگرد تا به چشم عام عزيز باشى. و از قرينِ بد پرهيز كن و ادب كرسى نگاه دار و اين شرط جاى ديگر ياد كرده‌ام. و از تكبّر و دروغ و رشوت دورباش و خلق را آن فرماى كردن كه تو كنى كه تا عالم منصف باشى. و علم را نيكو بدان و آنچه بدانستى به نيكوترين عبارتى به كار دار، تا خجل نباشى به دعوى كردن بى‌معنى. و در سخن گفتن و موعظه دادن هر چه گويى با خوف و رجا گوى، يك‌باره خلق را از رحمت خداى تعالى نوميد مكن و نيز يك‌باره بى‌طاعت هيچ كس را به بهشت مفرست. بيشتر آن گوى كه بر آن ماهر باشى و نيك معلوم تو گشته باشد، تا در سخن دعوى بى‌حجت نكرده باشى كه عاقبت دعوى بى‌حجت شرم‌سارى بود.

فصل، پس اگر از دانشمندى بدرجه‌ی بزرگ‌تر اوفتى و قاضى شوى چون قاضيان حمول\footnote{حلیم و بردبار} و آهسته باش و زيرك و تيزْفهم، صاحبِ‌ تدبير و پيشْ‌بين و مردمْ‌شناس و صاحبْ‌سياست و دانا به علمِ دين و شناسنده‌ی طريقه‌هاى هر گروه و از احتيال\footnote{چاره‌گری،‌حیلت} هر گروه و ترتيب هر مذهبى و هر قومى آگاه باش. و بايد كه حيل قضاة تو را معلوم باشد تا اگر وقتى مظلومى به حكم آيد و وى را گوايى نباشد و بر وِى ظلم رود و حقى از آن وى بخواهد رفتن، از كارِ آن مظلوم بر رسى و به حيله و تدبير، آن مستحق را به حق خويش رسانى.

چنانكه به طبرستان قاضى‌القضاة ابو‌العباس رويانى بود. و وى مردى مستور بود و اعلم و ورع\footnote{پرهیزگار} و پيشْ‌بين و صاحبِ تدبير. و وقتى به مجلس او مردى پيش او به حكم آمد و بر مردى صد دينار دعوى كرد. قاضى از آن خصم پرسيد. آن مرد انكار كرد. قاضى اين مدعى را گفت: گواه دارى‌؟ گفت: ندارم. قاضى گفت: پس وى را سوگند دهم. مدعى بگريست زار زار و گفت: اى قاضى زينهار! وى را سوگند مده، كه وى بر سوگند خوردن دلير شده است و باك ندارد. قاضى گفت: من از شريعت بيرون نتوانم شد، يا تو را گواه بايد، يا وى را سوگند رسد. مرد در پيشِ قاضى در خاك همى گرديد و همى گفت: زينهار! اى قاضى مرا گواه نيست و وى سوگند بخورد و من مظلوم و مغبونم، زنهار بگردن تو تدبير بكن. قاضى چون زارى مرد بديد، بدانست كه راست همى‌گويد. گفت: اى خواجه وام دادن تو او را چگونه بوده است‌؟ از اصلِ كار مرا باز گوى، تا بدانم كه اين كار چون رفته است‌؟ اين مظلوم گفت: زندگانى قاضى دراز باد، اين مرد مردى بود چند ساله دوست من، اتفاق افتاد كه بر پرستارى عاشق شد، قيمت وى صد و پنجاه دينار و مايه‌ی اين مرد كم از صدوپنجاه دينار بود. و هيچ وجهى نمى‌دانست، شب و روز چون شيفتگان همى‌گشتى و همى‌گريستى و زارى همى كردى. روزى به تماشا رفته بوديم، من و وى در دشت تنها همى‌گرديديم، زمانى جايى بنشستيم. اين مرد با من سخن اين كنيزك همى‌گفت و زار همى‌گريست و دل من بر وى بسوخت كه بيست ساله دوست من بود. وى را گفتم اى فلان، تو را زينت تمام بهاى وى و مرا نيز نيست و هيچ كس دانى كه درين معنى فرياد تو نخواهد رسيد، اما مرا در همه جهان صد دينارست به سالهاى دراز جمع كرده‌ام. اين صد دينار تو را دهم و تو باقى بر سر نهى و اين كنيزك را بخرى و يك ماه مراد خويش از وى برگيرى و پس از ماهى بفروشى و زر من باز دهى. اين مرد پيش من به خاك بگرديد و سوگندان خورد كه: يك ماه بدارم و پس از آن اگر به زيان خرند بفروشم و زر تو باز دهم. من آن زر از ميان بگشادم و بدو دادم، من بودم و او و خداى عزّوجل، اكنون چهار ماه برآمد نه زر من باز همى‌دهد و نه كنيزك همى‌فروشد. قاضى گفت: كجا نشسته بودى، بدين وقت كه زر بدو دادى‌؟ گفت: به زير درختى. قاضى گفت: پس كه به زير درختى بودى، چرا مى‌گويى كه گواه ندارم‌؟ اين خصم را گفت: هم اينجا بنشين پيش من و مدعى را گفت: دل مشغول مدار برو و زير آن درخت دو ركعت نماز كن و صد بار بر پيغامبر صلى‌اللّه‌عليه‌و‌سلّم درود ده و آن درخت را بگوى كه: قاضى تو را همى‌خواند، بيا و گواهى من بده. خصم تبسّم كرد، قاضى بديد و بر خويشتن پوشيده كرد. مدعى گفت: اى قاضى ترسم كه آن درخت به فرمان من نيايد. قاضى گفت: اين مهر من ببر و درخت را گوى كه: اين مهر قاضى است، همى‌گويد: بيا و گواهى كه ببر توست بده اندرين باب. مرد مهر بستد و برفت و مرد ديگر پيش قاضى بنشست و قاضى به حكمهاى ديگر مشغول شد، خود بدين مرد نگاه نكرد تا يك‌بار در ميانه‌ی حكمى كه همى‌كرد رو سوى اين مرد كرد و گفت: فلان آنجاى رسيده باشد يا نه‌؟ اين مرد گفت: نه هنوز. قاضى به حكم مشغول شد. آن مرد مهر بدرخت نمود و گفت: قاضى تو را همى‌خواند، چون زمانى بنشست، از درخت جواب نيامد، غم‌ناك شد و بازگشت و پيش قاضى آمد و گفت: اى قاضى رفتم و مهر نمودم، نيامد. قاضى گفت: غلطى كه درخت آمد و گواهى داد و روى به خصم كرد و گفت: حق اين مرد بده يا كنيزك را بفروشم و زر به وِى دهم. مرد گفت: اى قاضى، تا من اينجا نشسته‌ام هيچ درخت نيامد. قاضى گفت: راست گويى، درخت نيامد امّا اگر تو اين زر از وى نگرفته‌اى زير آن درخت كه من از تو پرسيدم كه مرد رسيده باشد بدان درخت يا نه تو چرا نگفتى كه كدام درخت‌؟ من ندانم كه وى كجا رفته است‌؟ و مرد را الزام كرد و زر بستد و به خداوند حق داد.

پس همه حكم‌ها از كتاب نكنند، از خويشتن نيز بايد كه چنين استخراج‌ها كنند و تدبيرها سازند. و ديگر بايد كه در خانه‌ی خويش سخت متواضع باشى، اما در مجلس حكم به هيبت نشينى و ترش روى و بى‌خنده و باجاه و حشمت باشى، گران‌ْمايه و اندك‌ْگوى و بسيار‌ْنيوش و از شنيدن سخن و حكم كردن البته ملول نشوى و از خويشتن ضجرت\footnote{تنگدل،‌ ملال، اندوه} ننمايى و صابر باشى. و مسئله‌اى كه بيفتد، همه اعتماد بر راى خويش مكن و از مفتيان نيز مشورت خواه و راىِ خويش مادام روشن دار و پيوسته خالى مباش از درس مذهب و مسايل مذهب. و چنانكه گفتم تجربت‌ها نيز به كار دار كه در شريعت راىِ قاضى برابرِ راى شريعت است و بسيار حكم بُوَد كه از راىِ شرع گران آيد قاضى سبك بگيرد، چون قاضى مجتهد بود، روا بود. پس قاضى بايد كه مجتهد و دانا بود و فقيه و پارسا بود و بايد كه به چند وقت حكم نكند: يكى به گرسنگى و تشنگى، و از گرمابه برآمده، به وقت دل‌تنگى و انديشه‌ی دنيائى كه پيش آيد. و وكيلان جلد پيش دارد و نگذارد كه در وقتِ حكم كس قصه و سرگذشتِ خويش گويد و شرحِ حال خويش نمايد، بر قاضى شرطِ حكم كردنست، نه متفحصى، كه بسيار تفحص بود كه ناكرده بِهْ بُوَد، و سخن كوتاه كند و زود به سوى گواه و سوگند كشد. جايى كه داند كه مال بسيارست و مردم ناباكى بكند، هر تجربتى و تجسّسى كه بتواند بكند و هيچ تقصير نكند و سهل نگيرد. و مادام معدلان\footnote{تعدیل کننده} نيك را هم بر خود دارد و حكم كرده هرگز باز نشكافد و امر خويش را قوى و محكم دارد. و هرگز به دست خويش قباله و منشورى ننويسد، الّا كه ضرورتى بود و خطِ خويش را عزيز دارد و سخنِ خويش را تبجيل\footnote{بزرگ داشتن} كند. و بهترين هنرى قاضى را عمل است و ورع، پس اگر اين صناعت نورزى و اين توفيق نيابى و نيز لشكرى پيشه نباشى، بارى طريق تجارت بر دست گير تا مگر از آن نفعى يابى، كه هر چه از تجارت بدست آرى، حلال بود و به نزديك هر كسى پسنديده و ستوده بود. 


\newpage
