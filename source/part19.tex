\addcontentsline{toc}{section}{باب نوزدهم
در چوگان زدن}
\section*{باب نوزدهم
در چوگان زدن}

اگر نشاط چوگان زدن كنى، مادام چوگان زدن عادت مكن كه بسيار خلق را در چوگان زدن بد رسيده است.

% \HekaiatBegin
 چنانكه عمرو بن الليث را گويند كه يك چشم داشت. آنگه [كه] امير خوراسان گشت روزى به ميدان رفت كه گوى زند. وى را سپه سلارى \footnote{ سالار} بود، وى را ازهرِ\footnote{اسم طرف} خر گفتندى. اين ازهر خر بيامد و عنان او بگرفت و گفت: نگذارم كه تو گوى زنى. عمرو گفت: چون تو چوگان زنى روا نبود كه من چوگان زنم‌؟ گفت: نه. عمرو گفت: چرا؟ گفت: ازيرا كه ما را دو چشم است اگر گوى بر يك چشم آيد ما را، و به يك چشم كور شويم يك چشم ديگر داريم كه بدان ببينيم و تو يك چشم دارى اگر باتفاق گوى بر چشم تو آيد اميرى خوراسان بدرود بايد كرد. عمرو گفت: با همه خرىِ خود راست گفتى، پذيرفتم كه هرگز تا من باشم گوى نزنم.
% \HekaiatEnd

امّا اگر به سالى يك‌بار يا دو بار نشاط اوفتد [روا باشد]، اما سوار بسيار نبايد تا مخاطره‌ی سَمطه\footnote{زیان و خسران} نبايد. جمله سوار هشت بيش نبايد كه تو بر يك سر ميدان بپاى و يكى ديگر بر آخر ميدان و شش كس در ميانه‌ی ميدان تا گوى همى زنند.

هر گاه كه گوى به سوى تو آيد تو گوى همى باز گردان و اسب را به تقريب همى ران امّا اندر كرّ و فر  \footnote{جنگ و گریز} مباش تا از صدمت ايمن باشى و نيز مقصود تو به حاصل شود. طريق چوگان زدن محتشمان اينست كه ياد كرده آمد\footnote{که گفتم} تا معلوم گردد.




















\newpage
